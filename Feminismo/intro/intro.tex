\chapter*{Apresentação}
\section*{Por que Feminismo na Análise do Comportamento?}
\begin{flushright}
    \emph{Carolina Laurenti}
\end{flushright}

Ecos do ativismo do movimento feminista começaram a ressoar no contexto acadêmico na década de 1960, e foram ganhando mais projeção entre o fim dos anos 1970 e início da década de 1980, inclusive na psicologia. Desde então, as críticas feministas a determinadas formas de se produzir conhecimento científico têm desafiado a ciência de um modo geral, e a psicologia em particular, a se tornar uma prática cultural mais justa, igualitária e democrática, deixando de reproduzir, subscrever ou mesmo encorajar desigualdades entre gêneros verificadas em diferentes esferas da sociedade. 

A interlocução entre Análise do Comportamento e Feminismo no âmbito acadêmico foi principiada, sobretudo, pelos trabalhos de Maria del Rosario Ruiz (1950-2017), que explorou, do ponto de vista teórico-filosófico, algumas afinidades entre teses feministas e a filosofia do comportamentalismo radical. As produções dessa autora, entretanto, parecem não ter sido suficientes para inserir a Análise do Comportamento de modo mais expressivo nesse cenário, considerando, até o momento, a escassa produção que discute essa aliança na área \footnote{Ver Couto, A.,\& Dittrich, A. (2017). Feminismo e análise do comportamento: Caminhos para o diálogo. Revista: Perspectivas em Análise do Comportamento, 8(2), 147-158.}.

No Brasil, o diálogo entre Feminismo e Análise do Comportamento é recente e relativamente incipiente. A despeito do caráter embrionário desses estudos no país, tem acontecido uma série de iniciativas com o intuito de propiciar um contexto favorável à discussão de temas fomentados pelo Feminismo acadêmico. O livro “Debates sobre Feminismo e Análise do Comportamento” é uma expressão lídima desse esforço, e preenche uma lacuna importante no tocante à produção nacional acerca do Feminismo. Mas não se trata “apenas” de um livro sobre Feminismo. É um livro sobre a Análise do Comportamento – a sua história, teoria, ciência e profissão – discutida de uma perspectiva de gênero. Esse viés dá relevo a aspectos usualmente negligenciados na narrativa historiográfica dessa disciplina (e.g., as diferentes mulheres que participaram da institucionalização e consolidação dessa disciplina no Brasil), bem como permite perscrutar temas que pouco integram as agendas de pesquisa do campo (e.g., patriarcado, machismo, cultura do estupro, empoderamento feminino, práticas de gênero, assédio sexual, interseccionalidade, participação feminina na computação, psicoterapia feminista). É, antes de tudo, um livro sobre a práxis acadêmico-científica dos/as analistas do comportamento, o seu lugar social e suas implicações ético-políticas.

Um livro como este certamente enfrentará e, ao mesmo tempo, lançará muitos desafios para a Análise do Comportamento. Um deles é justamente discutir Feminismo. Isso porque existem conotações pejorativas do termo bastante difundidas no senso comum que podem tolher, logo de início, qualquer tentativa de debate, mesmo na esfera acadêmica. Em uma palestra proferida no TEDxEuston\footnote{Eis o link para acessar a palestra “Todos nós deveríamos ser feministas”: \url{https://goo.gl/q6u5zy}}, a escritora africana Chimamanda Ngozi Adichie mencionou alguns desses estereótipos: feministas são geralmente consideradas “mulheres infelizes que não arrumam marido” e que “odeiam os homens”. 

Para além dessas acepções burlescas do conceito, no domínio acadêmico o Feminismo dá visibilidade a uma “tensão” entre ciência e valores, opondo-se ao pensamento científico moderno. Nesse modelo epistemológico, a noção de objetividade científica era esclarecida pela ideia de neutralidade: o conhecimento científico seria objetivo, pois não estaria comprometido com qualquer perspectiva de valor particular, seja no plano ético, seja no político. Isso contribuiu para a construção da visão de que a ciência, orientada pelo método científico, seria regulada unicamente pela Razão; por conseguinte, os parâmetros para avaliar o seu desenvolvimento estariam circunscritos ao funcionamento interno da própria ciência, como as exigências de consistência lógica e apoio empírico, ao mesmo tempo em que fatores culturais, econômicos e políticos eram desconsiderados. 

Essa “tensão” entre ciência e política também encontra ressonâncias na Análise do Comportamento, tendo em vista que há algumas interpretações que identificam aspectos do pensamento científico moderno nas práticas científicas da área\footnote{Ver Moxley, R. A. (1999). Two Skinners, modern and postmodern. Behavior and Philosophy, 27, 97-125.}, como a tese da neutralidade científica. Considerando essas relações, situar o Feminismo, que é um movimento político, no escopo das discussões teórico-científicas da Análise do Comportamento, poderia, supostamente, comprometer o ideal de produção de um conhecimento objetivo, se objetividade ainda estiver sendo entendida como sinônimo de neutralidade.

Outro desafio que uma aliança com o Feminismo lança à Análise do Comportamento é estudar “gênero” – um termo que destaca as diferenças socialmente constituídas entre os diferentes sexos, estabelecendo o que seria entendido por masculino e feminino em uma dada cultura. Do ponto de vista epistemológico, o tema do gênero está tradicionalmente associado às ciências humanas. Isso reacende a já desgastada dicotomia entre ciências naturais e ciências humanas – uma oposição que afirma a superioridade científica das primeiras e desconfia do status científico das segundas. Reiterando essa dicotomia, o ensino de Análise do Comportamento por vezes filia essa proposta de psicologia científica ao campo das ciências naturais, com o intuito de revesti-la dos qualificativos e desideratos dessas ciências: rigor metodológico, operacionalização das variáveis, uso do método experimental e busca de regularidades nos fenômenos, com o fim último de previsão e controle. Tudo isso, à primeira vista, parece ser antitético aos temas, à epistemologia e às metodologias “qualitativas” das ciências humanas. Como, na visão tradicional, há um ceticismo sobre cientificidade dessas ciências, estudar gênero poderia conferir à Análise do Comportamento um status menos científico, afastando-a daquelas áreas de conhecimento que gozam de maior prestígio acadêmico. 

Uma vez enfrentados, e quiçá superados, esses desafios poderiam se transformar em valiosas contribuições à Análise do Comportamento. Distanciando-se dos estereótipos ilustrados por Chimamanda, o Feminismo, do ponto de vista político, pode ser entendido como um programa de ação que busca explicitar, enfrentar e superar práticas culturais opressivas, responsáveis por promover a desigualdade entre gêneros, que se manifesta em prejuízo das mulheres. O movimento feminista chama a atenção para o fato de que em algumas culturas as diferenças entre homens e mulheres, dentre elas as de natureza biológica, são transformadas em desigualdades. “Dominação masculina”, “patriarcado”, “machismo” são expressões utilizadas para dar visibilidade a essas práticas. De uma perspectiva analítico-comportamental, elas podem ser entendidas como um conjunto de contingências sociais, mantidas e transmitidas de geração a geração (práticas culturais), que controlam diferencialmente o comportamento de homens e mulheres, de modo que os homens teriam um acesso facilitado a reforçadores importantes (poder), que nem sempre são contingentes ao seu comportamento (privilégio). 
Como a ciência é parte e expressão da cultura, as desigualdades entre os gêneros, observadas em distintos contextos socioculturais (e.g., ambiente doméstico, trabalho, educação, religião etc.), podem também estar sendo reproduzidas pela própria atividade científica. Isso se evidencia, por exemplo, na exclusão histórica das mulheres da ciência, na disparidade entre gêneros verificada em diferentes campos científicos em favor dos homens (e.g., ciências matemáticas e tecnológicas) e na menor participação feminina na medida em que se avança para posições de mais notoriedade na hierarquia científica\footnote{Ver Nosik, M. R., Luke, M. L., \& Carr, J. E. (2018). Representation of women in behavior analysis: An empirical analysis. Behavior Analysis: Research and Practice. Advance online publication. \url{http://dx.doi.org/10.1037/bar0000118}}. 

Toda essa reflexão não parece ser inconsistente com os pressupostos teórico-filosóficos da Análise do Comportamento, de acordo com os quais ciência é comportamento do/a cientista, modelado e mantido por uma comunidade científica. De acordo com essa ótica, o comportamento científico é controlado não só por contingências relacionadas às regras do método científico, mas também por contingências da história de vida do cientista e da cultura na qual ele está inserido, e que não precisam ser tateadas para controlar o seu comportamento.

Uma aliança com o Feminismo poderia, então, promover uma mudança na identidade epistemológica da Análise do Comportamento, superando o pensamento binário que pauta a dicotomia entre ciências naturais e ciências humanas. O estudo do gênero na área poderia ser feito de acordo com procedimentos canonizados pelas práticas científicas dos/as analistas do comportamento; ele poderia, outrossim, ensejar novas e diferentes estratégias e procedimentos de investigação do assunto. De qualquer modo, não parece necessariamente haver ameaça à cientificidade da Análise do Comportamento estudar esse tipo de variável que, ao lado de outras, como poder, classe social, raça/etnia, são, não raro, desconsideradas nas análises funcionais, tanto do comportamento dos participantes das pesquisas e intervenções quanto do próprio comportamento do pesquisador e profissional analista do comportamento. Estudar gênero poderia, ainda, tornar a Análise do Comportamento mais objetiva, não na acepção de neutralidade científica, mas no sentido de que o processo de produção de conhecimento científico e os seus produtos possam ser avaliados igualmente entre homens e mulheres - algo que só será possível mediante a ``aplicação sistemática de métodos que permitam identificar os pressupostos, os preconceitos, os valores e os interesses que subjazem à investigação científica supostamente desprovida deles''\footnote{Ver Santos, B. S. (2000). A crítica da razão indolente: Contra o desperdício da experiência (vol. 1, p. 31). São Paulo: Cortez.}.

Além disso, o Feminismo destaca a relevância da discussão política na Análise do Comportamento, um aspecto do qual essa teoria é recorrentemente acusada de negligenciar, em função de sua ênfase em questões procedimentais e tecnológicas. De acordo com essas críticas, a falta de reflexões ético-políticas tem contribuído para que as intervenções dos/as analistas do comportamento, mesmo que amparadas em análises funcionais, fiquem centradas no indivíduo, desconsiderando o contexto mais amplo de contingências culturais e institucionais, das quais participam relações hierárquicas de poder. Deixar de reconhecer e examinar esses aspectos pode levar esses/as profissionais a serem “parte do problema, e não da solução”\footnote{Ver Holland, J. G. (1978). Behaviorism: Part of the problem or part of the solution? Journal of Applied Behavior Analysis, 11, 163-174.}.

O valor de sobrevivência das culturas, tão defendido no plano ético, se não for subsumido a uma discussão política, pode acabar subscrevendo a reprodução de culturas que simplesmente “sobreviveram”, ignorando que essas podem ser mais ou menos democráticas, mais ou menos justas, mais ou menos respeitosas\footnote{Ver Prileltensky, I. (1994). On the social legacy of B. F. Skinner: Rhetoric of change, philosophy of adjustment. Theory \& Psychology, 4(1), 125-137.}. Diante da questão “das culturas que sobrevivem, qual é a melhor, ou qual é a que deveria perecer?”, o Feminismo ajuda a dar uma resposta: uma cultura que fomenta a opressão, seja de que forma for, como a cultura da dominação masculina, deveria perecer. Em termos de projeto social, a pergunta ``qual cultura deveria sobreviver?'' também tem uma resposta feminista: uma cultura que promova relações mais igualitárias e justas entre gêneros é a melhor! Uma aproximação com o Feminismo poderia, então, instigar a potencial contribuição da Análise do Comportamento para mudar as formas opressivas de controle social, em direção à construção de um mundo melhor para todos e todas. 

Em suma, as reflexões feministas podem tornar a Análise do Comportamento uma ciência melhor, mais objetiva e mais engajada; e é isso que este livro vem mostrar. 