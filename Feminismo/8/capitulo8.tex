\setcounter{footnote}{0}
\setcounter{figure}{0}
\setcounter{table}{0}
\chapter*{Mulheres e tecnologia: aspectos culturais e intervenções comportamentais para aumento da participação feminina na computação}\sectionmark{Mulheres e tecnologia: aspectos culturais e intervenções \\comportamentais para aumento da participação feminina na computação}\blfootnote{A autora agradece e dedica este capítulo às mulheres responsáveis pelas iniciativas Pyladies São Paulo; Techladies Curitiba e Women Up Games, que além de inspirarem muito do conteúdo aqui apresentado, guiaram e apoiaram seus primeiros passos no mundo da programação.}
\addcontentsline{toc}{chapter}{Capítulo 8}
\addcontentsline{toc}{section}{Mulheres e tecnologia: aspectos culturais e intervenções comportamentais para aumento da participação feminina na computação}
\addcontentsline{toc}{subsection}{\textbf{Autora:} \textit{Izadora Ribeiro Perkoski}}
\begin{flushright}
\begin{small}
    Izadora Ribeiro Perkoski  
\end{small}
\vspace{1cm}
\end{flushright}

O desenvolvimento tecnológico é, assim como a ciência, produto do comportamento humano. Por isso, tanto o processo de desenvolvimento tecnológico quanto seus resultados estão embebidos em valores culturais. O caminho tecnológico que o conhecimento segue é determinado pelo contexto cultural, tanto quanto a inovação é produto da história de reforçamento do inventor. Para Lattal (2003), a tecnologia e a ciência estão intrinsecamente ligados, e o autor define a tecnologia como “a aplicação dos achados científicos a problemas da vida diária” (p. 943), uma definição bastante ampla que abrange a criação de novas ferramentas, métodos, processos e procedimentos para intervir na realidade humana e, evidentemente, abarca muito da produção em Análise do Comportamento. Há algumas características especialmente importantes da tecnologia, segundo Lattal: o fato de que ela não apenas se alimenta da pesquisa científica, mas também retroage sobre a forma como a ciência acontece e, assim como tem o poder de modificar a ciência, a tecnologia também modifica a si mesma – e exemplifica citando a evolução dos computadores.

A Computação é parte da vida diária das pessoas no século XXI, e aos poucos vem sendo adotada e incorporada à Ciência do Comportamento por seu potencial para permitir uma investigação mais refinada e precisa do nosso objeto de estudo. Analistas do comportamento começam a se dedicar a essa intersecção, e a publicação do livro “Introdução ao desenvolvimento de \textit{softwares} para analistas do comportamento” (Neves Filho, de Freitas, \& Quinta, 2018) representa um marco nesse movimento. Um dos capítulos apresenta um manifesto pelo ensino de programação nos cursos de Psicologia e, ao pontuar os desafios relacionados a essa mudança na formação do psicólogo, ressalta:

\begin{quote}
    Outra dificuldade para o ensino de programação nos cursos de Psicologia tem um forte aspecto cultural e tem sido amplamente debatido. As áreas de tecnologia continuam a serem dominadas por homens (este livro infelizmente não foge à regra, por exemplo). Logicamente, a ausência de mulheres nessas áreas ocorre pela falta de estímulo nos períodos escolares. A Psicologia (no Brasil, um curso caracterizado pela grande presença feminina) poderá ajudar a combater este viés ao incluir programação em sua grade curricular, pois estimulará que mulheres aprendam uma habilidade tecnológica e, desse modo, possam exercer melhor papéis de liderança e inovação. (Cardoso, Neves Filho, \& de Freitas, 2018, pp. 169-170).
\end{quote}

Como podemos observar, a baixa participação feminina na Computação se apresenta como um problema com implicações diretas para o desenvolvimento tecnológico da Análise do Comportamento, ciência feita majoritariamente por mulheres. 

Para abordar esse problema, este capítulo adota uma perspectiva feminista. Tal perspectiva pode parecer contraditória, já que muitas vezes o feminismo é pintado como tecnofóbico ou como avesso ao método científico de forma geral (Rothschild, 1981). Apesar disso, há diversas teóricas feministas que se debruçaram sobre o tema da tecnologia e suas intersecções com questões de gênero, tanto na segunda quanto na terceira onda do movimento feminista. A feminista Joan Rothschild (1981), por exemplo, sugere que uma perspectiva feminista da tecnologia pode contribuir tanto para explicitar os valores sexistas que guiam nosso desenvolvimento tecnológico quanto para redirecionar esse desenvolvimento para um caminho mais humanista. 

Portanto, este capítulo tem como objetivo oferecer uma interpretação de base feminista e behaviorista radical para o fenômeno da participação feminina no desenvolvimento tecnológico, mais precisamente na Computação, e oferecer possíveis caminhos de intervenção para modificar esse panorama. 

\section*{História das Mulheres na Tecnologia}\sectionmark{História das Mulheres na Tecnologia}

O apagamento das mulheres na história\footnote{A história aqui contada é incompleta, limitada e centrada na experiência das mulheres estadunidenses – porque assim são os registros disponíveis. A autora ressalta que a indisponibilidade de informações acerca das mulheres latino-americanas, africanas e asiáticas que certamente tiveram grandes contribuições em seus respectivos países de origem é parte do apagamento exposto, sendo evidência adicional dos problemas aqui relatados.} da produção cultural e tecnológica humana é uma realidade generalizada, seja nas artes, nas ciências ou na história das invenções. Nomes como Margaret Hamilton, Ada Lovelace e Carol Shaw são lembradas como felizes exceções de uma história dominada por homens. Essa é uma forma dramática, talvez inspiradora, talvez revoltante, de contar a história – mas é incompleta. Durante toda a história da Computação, desde seu nascimento com Babbage e Lovelace, as mulheres fizeram parte do desenvolvimento dessas tecnologias. 

No fim do século XIX, Edward Pickering era diretor do observatório astronômico de Harvard, e decidiu contratar mulheres para processarem os dados astronômicos coletados no observatório. A esse grupo de mulheres, foram dados dois nomes: “Computadoras de Harvard” e “harém de Pickering”. A princípio, o leitor desavisado pode encarar a contratação de uma equipe total ou majoritariamente feminina como um ato corajoso. A verdade é que o trabalho envolvido no processamento de dados, que consistia em identificar, medir e registrar dados de estrelas fotografadas pelos novos telescópios era considerado um trabalho extremamente tedioso e repetitivo. A mão de obra feminina responsável por catalogar um acervo com meio milhão de fotografias de estrelas era mal remunerada, e uma das “\textit{computadoras}” de Harvard poderia chegar a ganhar menos que alguém que trabalhasse em atividades rurais da região, metade do que era geralmente pago aos trabalhadores que realizavam trabalhos de cálculo (Nelson, 2008; Zarrelli, 2016). Apesar da péssima remuneração, o trabalho atraiu muitas alunas de pós-graduação e pesquisadoras interessadas em participar da catalogação e conduzir suas pesquisas paralelamente em seu tempo livre usando os dados do Observatório (Nelson, 2008). Do Observatório de Harvard saíram importantes nomes da Astronomia, como Williamina Fleming, Henrietta Swan Leavitt e Annie Jump Cannon. 

Alguns anos mais tarde, na década de 1930, o então \textit{National Advisory Committee for Aeronautics} (NACA – hoje chamado NASA), também contratou mulheres como “computadoras” para realizar os cálculos astronômicos (Blitz, 2017). Na década de 50, em meio à Guerra fria, corrida espacial e em um cenário de segregação racista, a NASA contava com mulheres negras como Kathryn Peddrew, Ophelia Taylor, e Sue Wilder, que foram peças indispensáveis para a exploração bem-sucedida do espaço. Apesar de sua enorme contribuição, o trabalho dessas mulheres só foi documentado, reconhecido e divulgado amplamente a partir de 2016, com o lançamento do livro “\textit{Estrelas além do tempo}”, da autora Margot Lee Shetterly, posteriormente adaptado para o cinema.

Paralelamente, nos anos 40, a necessidade de calcular trajetórias de mísseis durante a II Guerra Mundial exigia que o Exército empregasse diversas mulheres com formação em matemática para fazerem esses cálculos, resolvendo milhares de vezes as mesmas equações, já que os matemáticos do sexo masculino ou estavam servindo nas linhas de frente, ou fazendo outros trabalhos de inteligência. Quando dois engenheiros responsáveis por criar uma máquina que pudesse desempenhar os cálculos de forma mais eficiente precisaram de mão de obra para programá-la, Francis Snyder Holberton, Betty Jennings Bartik, Kathleen McNulty Mauchly Antonelli, Marlyn Wescoff Meltzer, Ruth Lichterman Teitelbaum, e Frances Bilas Spence foram designadas para a função. Após terminarem a programação do ENIAC (\textit{Electronic Numerical Integrator and Computer}), algumas das mulheres envolvidas no projeto migraram para o desenvolvimento de um computador comercial (UNIVAC - \textit{Universal Automatic Computer}), em que passaram a trabalhar com Grace Hopper. Nas descrições dos registros fotográficos da época, porém, elas foram por muito tempo confundidas com “modelos”, colocadas artificialmente nas cenas apenas por propósitos comerciais – narrativa que só foi modificada a partir dos anos 90. Quando o aniversário de 50 anos do ENIAC foi comemorado, as pioneiras da programação não foram convidadas (Sheppard, 2013).

Enquanto na Astronomia e Matemática a participação feminina era profusa, ainda que subestimada no reconhecimento de suas contribuições, na história dos \textit{videogames} há pouca documentação dessa participação. Embora projetos acadêmicos tenham dado origem a jogos já nos anos 40, o primeiro jogo digital creditado a uma mulher é o \textit{3D-Tic Tac Toe} de Carol Shaw, produzido pela Atari (1978) durante o que já é considerada a segunda geração dos jogos digitais. Talvez o cenário de jogos digitais pareça pouco relevante se comparado à criação de computadores, astronomia ou engenharia avançada, mas é importante considerar que os \textit{videogames} se tornaram produto cultural de enorme influência, além de movimentar grandes cifras e impulsionar avanços tecnológicos. Além disso, a indústria de \textit{games} é muito relevante para entendermos o que acontece com a participação feminina na tecnologia a partir dos anos 80.

Durante os anos 70 e até meados dos anos 80, a participação feminina nos cursos de computação era bastante significativa, mas caiu de forma brusca na metade da década de 80 (Stross, 2008). Se antigamente programar era trabalho de mulher, e hoje somos minoria na área de tecnologia, o que mudou a partir de então? Um dos fatores comumente creditados é a chegada do computador pessoal, nos anos 80. Como você deve imaginar (ou lembrar, caso já fosse nascida nessa época), os primeiros computadores pessoais não tinham tanta capacidade de processamento, o que os tornava úteis principalmente para entretenimento e jogos eletrônicos (Fessenden, 2014). Essas máquinas passaram a ser anunciadas como brinquedos para meninos. Além disso, na segunda metade dos anos 80, a indústria de jogos digitais passou por uma crise e empresas importantes do segmento como a Nintendo passaram a, estrategicamente, focar suas ações de publicidade no público masculino (Lien, 2013). Esses fatores, juntamente com as ótimas perspectivas de carreira em tecnologia na época, contribuíram para reacender o interesse desse grupo pela Computação, e assim meninos e meninas voltaram a ser expostos diferencialmente à tecnologia. 

A história das mulheres na Computação, no fim das contas, serve como um ótimo exemplo de como uma sociedade que estratifica e distribui privilégios por sexo opera: embora estivessem presentes em todas as etapas do desenvolvimento tecnológico em Computação, as mulheres tinham sua força de trabalho explorada, sendo contratações preferenciais justamente por “aceitarem” salários menores, e fazendo trabalhos geralmente recusados por serem considerados “braçais” ou “insignificantes” pelos seus pares masculinos. Precisamos situar o desenvolvimento tecnológico em uma moldura cultural que abriga em seu cerne questões relacionadas ao capitalismo e aos privilégios de gênero e raça como fatores que não apenas influenciam, mas determinam diretamente o caminho que tal desenvolvimento segue. Foi devido à exploração econômica que as mulheres entraram na área científico-tecnológica, e foi pela alternância de interesses dos homens que elas saíram.

\section*{Panorama Atual da Participação Feminina na Tecnologia}\sectionmark{Panorama Atual da Participação Feminina na Tecnologia}

Quando buscamos dados demográficos do perfil do trabalhador em tecnologia atualmente, encontramos alguns desafios relacionados principalmente ao método de coleta e tratamento dos dados. A pesquisa de perfil do usuário do StackOverflow é uma exceção a essa regra, e dá algumas pistas interessantes acerca das características da indústria a nível mundial em 2018: com cem mil participantes distribuídos por todo o planeta, a pesquisa foi realizada por meio de um questionário cuja resposta era voluntária. Dos participantes, 92.9\% identificaram-se como homens, 6.9\% como mulheres e 0.9\% como não binários, \textit{genderqueer} ou em não-conformidade de gênero. Segundo as estimativas do site, por volta de 10\% das visitas totais ao site são feitas por mulheres. Com relação à raça, 74.2\% da amostra se identifica como branco(a), e apenas 2.8\% se identificou como negro(a). 71\% dos entrevistados não tem filhos ou outros dependentes (StackOverflow, 2018).

Há outros pontos interessantes a serem debatidos na pesquisa do StackOverflow além do perfil do usuário. Por exemplo, as áreas com menor desigualdade na proporção entre homens e mulheres são: pesquisa acadêmica e/ou ensino, \textit{design, quality assurance} e ciência de dados, enquanto as com maior desigualdade são administração de sistemas e desenvolvimento e operação de \textit{softwares} (\textit{DevOps}), onde a chance de um profissional ser homem é de 25 a 30 vezes maior do que de ser uma mulher. Na mesma pesquisa pode-se observar que as funções com maior presença feminina são, também, as que têm maior número de profissionais ativamente procurando por emprego (17\% ou mais dos trabalhadores, dependendo da área) e estão entre as pior remuneradas (pesquisa/educação em terceiro lugar, \textit{design} em quarto, \textit{quality assurance} em sétimo. A exceção é a ciência de dados, terceira melhor remunerada, atrás de gerência de engenharia e DevOps).

A pesquisa investigou, ainda, os fatores de maior e menor prioridade na avaliação de uma oportunidade de trabalho na opinião dos participantes. Os fatores mais importantes foram, em ordem de prioridade: a remuneração e benefícios oferecidos (18\%), a linguagem ou \textit{framework} com o qual se trabalharia (17\%) e as oportunidades para desenvolvimento profissional (16\%). O fator considerado menos importante foi, com ampla margem de diferença, a diversidade da companhia ou organização (30\%), seguida pela performance financeira ou status de financiamento da companhia (14\%). Os benefícios mais valorizados pelos profissionais de ambos os sexos são o salário (70,2\%) e o plano de saúde (8,6\%). As opções mais votadas como “menor prioridade” foram auxílio-creche (21,7\%), e licença maternidade/paterni\-dade (14,1\%), sendo esses dois benefícios menos valorizados do que a oferta de refeições e lanches (12,3\%). Ao estratificar os resultados por gênero, temos o seguinte panorama: homens se preocupam majoritariamente com a remuneração (19\%) e com as linguagens e \textit{frameworks} com os quais irão trabalhar (17,6\%), enquanto mulheres se preocupam com o ambiente e cultura da companhia (16,9\%) e oportunidades para desenvolvimento profissional (16,8\%). 

Com relação aos anos de experiência, observamos aquilo que o próprio documento considera uma “evidência da mudança demográfica dos profissionais de programação”: 17\% das mulheres que responderam à pesquisa programam há dois anos ou menos, contra apenas 8\% dos homens. Enquanto 47,9\% das mulheres programam há menos de cinco anos, esse número é de 30\% para os homens. 

\section*{Variáveis Culturais Relevantes para a Baixa Representatividade Feminina em Tecnologia}\sectionmark{Variáveis Culturais Relevantes para a Baixa Representatividade Feminina em Tecnologia}

Como a desigualdade numérica entre homens e mulheres se mantém? E como podemos combatê-la? A historiadora feminista Londa Schiebinger (1991/2001) aponta dois modelos explicativos para a baixa participação das mulheres na ciência em geral adotados em diferentes momentos da história. Até os anos 70, era usada uma caracterização verticalizada e “de cima para baixo” do problema: as mulheres não se desenvolviam em suas carreiras científicas devido à adoção de práticas discriminatórias por parte dos níveis mais altos que impediam as mulheres de ascenderem. No fim dos anos 80, a interpretação do problema se inverte, e a baixa participação feminina passa a ser vista como produto de um baixo interesse feminino pela ciência, e a solução proposta baseava-se em propostas individuais de mudança de socialização por meio da exposição das meninas às mesmas condições dos meninos (Schiebinger, 1991/2001).

Essa abordagem entende que muitos dos interesses e aptidões dos indivíduos são modelados ainda na infância, passando desde as opções de brinquedos dadas às crianças, até diferenças na experiência educacional, como na atenção e didática adotada pelos professores ao explicarem matemática e ciências a meninos ou meninas, onde os professores tendem a incentivar mais a participação oral em sala e dar maior liberdade criativa na solução de problemas aos meninos do que às meninas, por exemplo. A disponibilidade de modelos femininos de cientistas e de uma comunidade de mulheres também são fatores relevantes, principalmente nos anos mais avançados da educação.

Nesse sentido, Schiebinger (1991/2001) ressalta que, com relação ao perfil das mulheres que se mantêm nas carreiras científicas, elas tendem a ter pais com formação científica e serem de famílias mais abastadas em comparação aos estudantes homens. Um dado adicional que a pesquisadora apresenta é que as egressas de faculdades femininas tendem a seguir carreiras científicas em proporção maior do que aquelas de instituições mistas. 

O modelo “de baixo para cima” parece insuficiente quando consideramos o dado apresentado por Schiebinger (1991/2001) acerca da evasão feminina nos campos científicos mesmo quando estas tem sucesso profissional: o número de mulheres que abandonam a carreira científica é alto (20\% segundo a autora), e as causas atribuídas pelas próprias mulheres estão relacionadas à discriminação no ambiente de trabalho e às demandas familiares. A autora ressalta que apenas intervenções pontuais jamais serão capazes de causar as mudanças estruturais profundas necessárias para o enfrentamento da discriminação contra mulheres e outras minorias, e que esse modelo “(...) não proporciona esclarecimento sobre como a estrutura das instituições ou as práticas correntes da ciência precisam mudar, antes que as mulheres possam ingressar comodamente nas fileiras dos cientistas.” (Schiebinger, 1991/2001, p.134).

Assim, uma análise adequada da questão precisa desvelar os fatores que estão por trás tanto da discriminação enquanto prática cultural quanto às origens das diferenças de repertório entre homens e mulheres que tornam a participação feminina menos frequente e menos proeminente. Precisamos estar atentas, por exemplo, ao fato de que essa desigualdade é mantida porque permite o acesso e controle de reforçadores por um grupo em detrimento do outro, ou seja, envolve questões de poder e privilégio (Terry, Bolling, Ruiz, \& Brown, 2010).

Se voltarmos à nossa pequena reconstrução histórica feita anteriormente, podemos observar que o fator principal para a entrada das mulheres no campo da tecnologia foi o fato de que os empregos disponíveis nesse setor eram pouco interessantes aos homens (baixo valor reforçador) ou não haviam homens disponíveis devido à demanda das duas Guerras Mundiais. Tão logo os meninos dos anos 80 começam a ser sistematicamente expostos ao computador pessoal, e as carreiras em computação voltam a chamar a atenção desse público (o valor reforçador aumenta devido à história de vida dos meninos que crescem com videogames), as mulheres começam a desaparecer desses setores. 

Além disso, dominar a tecnologia atualmente significa ser capaz de mediar grande parte das relações de outros indivíduos com o mundo. O programador competente é, muitas vezes, um planejador de contingências eficiente. Observe, por exemplo, a influência do \textit{Facebook} nas eleições americanas (Solon \& Siddiqui, 2017), em que o poder econômico e tecnológico de uma empresa fundamentada em tecnologia de interação entre humanos, mediada por computador, é capaz de influenciar eleições e gerir o conteúdo ao qual milhões de pessoas são expostas.

O ponto comum na exclusão ou baixa representatividade das mulheres em qualquer área da sociedade é o sexismo, que permeia todas as relações que mantemos com o nosso ambiente social. Apesar dos dados globais do StackOverflow (2018) nos darem um panorama geral da situação (mesmo considerando que a amostra dos usuários do site pode ser viciada), Galpin (2002) ressalta que, embora a baixa participação feminina em tecnologia seja um problema global (com exceções pontuais, como Singapura), os fatores relevantes para esse fenômeno variam entre culturas, e que tais diferenças precisam ser levadas em conta principalmente quando tentamos generalizar práticas bem sucedidas em um contexto para outros. A autora exemplifica falando sobre o modelo explicativo americano, muito centrado na percepção individual das garotas acerca de sua própria capacidade e do status social das disciplinas tecnológicas, que não se aplica adequadamente ao cenário indiano, em que o processo decisório é conduzido pelos objetivos coletivos da família, que em geral vê um bom casamento como mais importante para as meninas do que a carreira profissional, e pode, inclusive, considerar a carreira uma ameaça a esse plano maior.

Assim, os dados apresentados por Galpin (2002) fortalecem a necessidade de parcimônia ao transferir estratégias estadunidenses ou europeias para o contexto brasileiro. Por isso, a seguir, são apresentadas algumas iniciativas que tem alcançado resultados consideráveis em solo nacional.

Considerando a influência dos fatores culturais no afastamento das mulheres da tecnologia, é preciso que ações conscientes e planejadas especificamente para combater este quadro sejam tomadas. Como analistas do comportamento, sabemos que nem sempre há correspondência entre o que as pessoas dizem e como elas se comportam. Um exemplo ilustrativo disso é a participação de minorias nas comunidades de \textit{software} livre e código aberto (\textit{Free/Libre and Open Source Software} – FLOSS): mesmo com um discurso que liga diretamente o desenvolvimento das tecnologias livres com a inclusão e a diversidade, apenas 3\% dos colaboradores em projetos de \textit{software} livre ou de código aberto em 2017 eram mulheres (Open Source Survey, 2017). Isso significa que, embora a inclusão seja um valor declarado dessa comunidade e parte do comportamento verbal de seus membros, não estão sendo adotadas ações efetivas, já que o número de mulheres contribuindo em 2002 era de pouco menos de 2\% (Ghosh, Glott, Krieger, \& Robles, 2002). Lin (2005) destaca algumas possíveis práticas dessas comunidades que podem estar relacionadas com a baixa participação feminina, como a forma como o manejo do tempo é feito dentro dos projetos FLOSS (que seria especialmente desfavorável à participação feminina, visto que este público tem menos horas disponíveis para contribuir nos projetos, já que muitas vezes cumpre jornada dupla ou tripla), ou o uso de linguagem discriminatória na documentação produzida (usando apenas o masculino ao se referir aos usuários e/ou desenvolvedores). Essa hipótese é corroborada pelo levantamento (Open Source Survey, 2017) realizado com 5.500 colaboradores de projetos de \textit{software} livre, segundo o qual 25\% das mulheres e 15\% dos homens já encontraram conteúdos que causaram a sensação de não serem bem vindos e 12\% das mulheres já se sentiram estereotipadas. 

Para mudar esse quadro, há dois tipos de política que precisam andar juntas: o planejamento cuidadoso de intervenções culturais, que busquem modificar valores e implementar práticas mais inclusivas dentro dessa comunidade; e intervenções de foco individual, por meio de programas de ensino capazes de ampliar o repertório dessas mulheres, permitindo interações mais efetivas com o ambiente social e o aparato tecnológico ligado à Computação. Embora a Análise do Comportamento ofereça poderosas ferramentas para tais iniciativas, é indispensável citar que já existem projetos de altíssimo nível promovidos pelas próprias comunidades. Esses projetos têm se convertido em resultados bastante interessantes, embora ainda incipientes, para os quais os analistas do comportamento (sejam os estudiosos da cultura, sejam os desenvolvedores de tecnologia educacional) devem olhar com cuidado e atenção.

\section*{Ações Práticas com Impacto na Participação Feminina: o Caso da Comunidade Python}\sectionmark{Ações Práticas com Impacto na Participação Feminina: o Caso da Comunidade Python}
\subsection*{O que é Python?}

Python é uma linguagem de programação de código aberto (ou seja, é mantida pela participação da comunidade e pode ser usada de forma livre e gratuita), mantida pela Python Foundation. A Python Foundation tem como missão promover, proteger e avançar a linguagem de programação Python e apoiar e facilitar o crescimento de uma comunidade internacional e diversa de programadores Python (Python Fundation, 2018).

Por ser uma linguagem de código aberto, o desenvolvimento de Python enquanto linguagem é dependente da participação ativa da comunidade, e a diversidade é ativamente incentivada na declaração de missão da \textit{Python Foundation}, porque criar ambiente diverso e pautado no respeito mútuo tem efeitos diretos no número de contribuidores e variabilidade nas produções da comunidade. Além do propósito declarado de promover uma comunidade diversa, é fomentada na comunidade Python a ideia de que a presença de novatos é indispensável para a expansão e sobrevivência da linguagem, e o ensino de Python é fortemente incentivado e divulgado na comunidade (Python Fundation, 2018).

\subsection*{PyLadies no Brasil e no Mundo}

Na comunidade Python brasileira, por exemplo, tem aumentado consideravelmente o número de mulheres ativas, e, segundo relatos, em 2016 a proporção de mulheres palestrando na maior convenção brasileira da linguagem, a Python Brasil, chegou a 40\% (PyLadies Brasil, 2016). Tais resultados são frutos de ações de naturezas diversas, desde a publicação de um código de conduta, apoio massivo a comunidades femininas, como o PyLadies, à promoção de ações de financiamento coletivo buscando financiar a participação das mulheres palestrantes nos eventos, dentre outras.

O PyLadies é um grupo de mulheres que programam ou têm interesse em aprender a programar usando a linguagem Python. O foco do PyLadies é criar e fortalecer comunidades locais de mulheres envolvidas em tecnologia, com foco em escrita de código, seja profissionalmente ou como \textit{hobby} (PyLadies, 2018c) e promover a participação e liderança feminina na comunidade \textit{opensource} (PyLadies Brasil, 2016).

Fundado em Los Angeles no ano de 2011 por um grupo de sete desenvolvedoras, hoje, o PyLadies está presente em cinco continentes, com organização local autônoma e apoio global e atua promovendo cursos de curta duração para introdução à programação com Python, eventos sociais e conferências, além de produzir materiais educativos que permitam a replicação do projeto em outras comunidades e apoiem a formação de grupos locais do PyLadies. No Brasil, o PyLadies nasceu em 2013 em Natal, no Rio Grande do Norte, e hoje está espalhado por diversas regiões do Brasil.

As organizadoras locais têm autonomia para decidir se os eventos serão exclusivos para mulheres ou se serão abertos para a participação de homens, e duas estruturas para a organização de eventos mistos são sugeridas: ou se oferece a opção de que cada mulher possa levar um acompanhante independente do gênero, ou se faz eventos abertos à toda a comunidade. O documento de orientação principal alerta para o fato de que “a dinâmica do grupo tende a mudar quando a proporção entre homens e mulheres se inverte” (PyLadies, 2018b).

Por pretender ser uma comunidade acessível e acolhedora às mulheres, todos os grupos PyLadies têm um código de conduta e uma política de denúncias. O documento detalha comportamentos considerados como assédio (agressão verbal, perseguição, intimidação, uso de imagens sexuais, gravação, filmagem ou fotografia com intenção vexatória ou ameaçadora, assédio sexual – todos pessoalmente ou por meios digitais), as possíveis medidas a serem tomadas pelas organizadoras (de notificações verbais a expulsão dos grupos) e os meios de denúncia pela internet ou em eventos presenciais (PyLadies, 2018a). Há também orientações sobre como formalizar uma denúncia em outros ambientes que não os eventos e listas de discussão do próprio PyLadies, oferecendo uma descrição dos elementos relevantes na denúncia (descrição do fato, descrição das ações esperadas da organização do evento ou grupo e, opcionalmente, um pedido de confidencialidade), para quem encaminhar denúncias e um modelo para que a denúncia seja feita de forma efetiva, que diminua a probabilidade de os organizadores ou pessoas responsáveis por recebê-la se esquivem de alguma forma (PyLadies, 2018). Na conferência Python Brasil, o maior evento dedicado à linguagem no país, a primeira versão do código de conduta foi publicada em 2014 no \textit{github}, um repositório bastante utilizado por programadores para construção colaborativa de códigos e documentos, e uma nova versão foi publicada em 2017, com descrições mais específicas do que é considerado assédio, discriminação e humilhação e das ações tomadas pela organização nessas situações (PythonBrasil, 2014/2017). É interessante notar que a implementação do código de conduta coincide com a criação dos primeiros grupos PyLadies no Brasil.

Podemos retomar algumas das variáveis relevantes discutidas anteriormente e observar como as ações das PyLadies atacam muitos dos problemas que apontamos: ao promover cursos práticos e permitir que as próprias aprendizes se tornem também tutoras ou monitoras em outros cursos e repliquem o que aprenderam, a situação de ensino fica mais reforçadora; os encontros periódicos, com presença maciça de mulheres, oferecem uma comunidade acolhedora, uma rede de apoio e diversidade de modelos de sucesso; o código de conduta PyLadies, assim como os das convenções Python e da própria Python Foundation, especificam comportamentos discriminatórios e contingências a serem programadas para extinguir ou punir esses comportamentos e, pelo menos segundo os relatos informais das comunidades, colocam em prática o planejamento que descrevem. Yitbarek (2016), por exemplo, descreve sua experiência ao participar pela primeira vez de uma convenção de Django (um \textit{framework} para desenvolvimento \textit{web} em Python) da seguinte forma:

\begin{quote}
    Em vários pequenos detalhes, [a organização do evento] passava uma mensagem clara e coesa que enfatizava a importância de que todos se sentissem bem-vindos. Da escolha dos lanches à presença de placas, a conferência demonstrou inclusão e empatia de uma forma que me deixou orgulhosa de estar lá. Eu não sei o quanto outros participantes notaram esses detalhes, ou o quanto eles ligavam para isso. Mas esses detalhes comunicaram muito a mim, mais do que qualquer discurso pronto sobre o valor da diversidade poderia comunicar um dia. As pessoas conversaram, decidiram e agiram para perceber as pequenas coisas que mantém as pessoas fora, que as fazem sentir sozinhas, que dizem a elas que elas não pertencem e corrigiram essas coisas. Esses detalhes são onde o amor e a empatia vivem. Esses detalhes significam se importar. (para. 6, tradução da autora).
\end{quote}

A inclusão e fomento à diversidade são justificados pela comunidade pela importância dessas práticas para a sobrevivência da linguagem Python como prática cultural: apenas com um grande número de contribuidores e usuários da linguagem ela ganhará espaço e se manterá atualizada e competitiva em relação às outras linguagens de programação. E, nesse cenário, excluir mulheres e outras minorias é tratado como contraproducente.

\section*{Intervenções Comportamentais e Estratégias de Ensino para Aumentar a Probabilidade de Engajamento Feminino em Cursos de Programação}\sectionmark{Intervenções Comportamentais e Estratégias de Ensino para Aumentar \\a Probabilidade de Engajamento Feminino em Cursos de Programação}

Quando falamos sobre sexismo em um determinado segmento da sociedade, como é o caso do acesso e participação no desenvolvimento de tecnologias, precisamos lembrar que não estamos falando de uma ocorrência isolada. O sexismo nesse meio é produto das mesmas contingências que mantém a desigualdade entre homens e mulheres de forma geral. Sendo assim, meninos e meninas desenvolvem diferentes repertórios comportamentais, isso porque os comportamentos esperados (e reforçados) para esses dois grupos em uma mesma situação são diferentes. Além de fortalecer estereótipos (muitas vezes tomados, equivocadamente, como diferenças “naturais” ou inatas entre os sexos), essa dinâmica contribui para a manutenção de um ciclo vicioso de exclusão da mulher: porque quando uma menina se interessa por certas coisas ela é ou ignorada ou ativamente punida, temos poucas mulheres persistindo nesses interesses, e assim as áreas tecnológicas continuam dominadas por homens, são consideradas “coisas de homem” e as meninas da geração seguinte seguem não perseguindo esse campo. 

E é para quebrar esse ciclo, tornando as meninas e mulheres mais confiantes e persistentes, que o planejamento de situações de ensino adequadas é extremamente relevante. Para os analistas do comportamento é especialmente fácil cair na armadilha de considerarmos um bom, mas genérico, planejamento de contingências de reforço suficiente para garantir o sucesso dos aprendizes. Feliz ou infelizmente, a realidade é que a aprendizagem dos nossos alunos no contexto imediato em que os ensinamos é fortemente influenciada pelos fatores culturais aos quais eles foram expostos ao longo de sua história. Para planejar um programa de ensino bem sucedido, é preciso considerar a identidade dos aprendizes, ou seja, precisamos considerar que existem componentes do repertório comportamental comuns a membros de um mesmo grupo social, já que esses indivíduos foram expostos às mesmas contingências de reforçamento, e mesmas condições de poder e privilégio. Para não cair em estereotipias ou caricaturas, é importante notar, porém, que esse conjunto de comportamentos (no sentido amplo, englobando pensamentos, discursos, práticas sociais, etc.) comuns a um determinado grupo não representa a totalidade do repertório de cada indivíduo. 

Para conduzir um programa de ensino que seja sensível a questões culturais e capaz de conectar o uso da tecnologia com a realidade do aprendiz a partir dessa perspectiva, é necessário que o educador aprenda a perceber e adaptar seu próprio comportamento ao contexto apresentado pelos estudantes, mais do que saber fatos específicos sobre como as mulheres aprendem, ou como as pessoas de baixa renda usam a tecnologia. Mais uma vez podemos destacar as formas como as PyLadies têm encarado isso: os materiais usados nos cursos geralmente são disponibilizados livremente, e toda aprendiz é incentivada a repassar o que aprende a outras pessoas de seu contexto. Aqui no Brasil é interessante notar que as primeiras experiências do PyLadies foram iniciativas de mulheres sem formação em Computação e, em muitos casos, sem conhecimento anterior de nenhuma linguagem de programação.

Além de considerar e planejar as situações de ensino de forma culturalmente contextualizada, é preciso que os objetivos comportamentais definidos pela educadora sejam capazes de, pelo menos parcialmente, preencher as lacunas deixadas por uma história de vida marcada pela socialização feminina. Por exemplo, Denner e Bean (2006) conduziram um programa de ensino de programação a meninas por meio do desenvolvimento de jogos, chamado \textit{Girls Creating Games}. As autoras dividem os objetivos de ensino do programa em cinco classes: fluência com o computador, autoeficácia, resolução de problemas, curiosidade e criatividade, com o objetivo de cobrir tanto as questões técnicas necessárias para a interação bem sucedida com o computador quanto os comportamentos que garantissem a persistência das aprendizes por todo o programa e além dele. Esses comportamentos são classificados como repertórios de exploração intrépida. 

Margolis e Fisher (2002) definem a exploração intrépida como sendo “o desejo de explorar sem medo de quebrar o computador, e a confiança para resolver problemas e lidar com revezes” (p. 29, tradução da autora). Para analisarmos esse conceito comportamentalmente, podemos dividi-lo em alguns componentes:

\begin{enumerate}
    \item Desejo de explorar: para haver desejo de explorar, é preciso que as consequências da exploração sejam reforçadoras.
    \item Ausência de medo: para explorar sem medo, é preciso que o ambiente não sinalize consequências aversivas, ou seja, risco de punição, para uma determinada resposta ou que pelo menos as consequências aversivas sejam de baixa intensidade.
    \item Confiança: confiar na própria performance exige uma história de reforçamento já bem estabelecida para aquela resposta, preferivelmente por uma combinação de reforçadores naturais e sociais (ou seja, é preciso que a resposta seja emitida de forma bem sucedida por repetidas vezes).
    \item Resolução de problemas: resolver problemas e improvisar envolve recombinar repertórios e testar soluções, o que engloba todos os pontos anteriores
\end{enumerate}

A exploração intrépida refere-se à interação com um aparato ou ambiente digital na qual a falha e o erro não tem função aversiva. Além disso, significa que a interação com o aparato é resistente à extinção, sendo mantida mesmo com fracassos repetidos e, mais do que isso, esses fracassos servem como ocasião para acessar informações sobre como melhorar a performance. É um conjunto de repertórios que permite que, a partir de um certo ponto, a aprendiz interaja com o computador de forma autônoma, confiante e criativa.

Um bom programa de ensino com o objetivo de aumentar a frequência e a qualidade da interação feminina com a tecnologia (seja em programação de \textit{software}, seja em outras áreas tecnológicas) deve dar igual prioridade ao desenvolvimento da fluência (segundo Denner \& Bean, 2006, generalização de comportamentos entre diferentes ambientes virtuais, uso competente da linguagem técnica, dentre outros fatores) e ao desenvolvimento de comportamentos de exploração intrépida.

Outro ponto importante é que todo tipo de estudante, em todo tipo de contexto, pode se beneficiar de um programa de ensino que adote uma abordagem de desenvolvimento de repertórios de exploração intrépida. As cinco dimensões do repertório de exploração intrépida descritas por Margolis e Fisher (2002) podem ser um ponto de partida para o desenvolvimento de tecnologias comportamentais para o ensino de programação. Assim, este capítulo tem também a função de ser um convite à realização de pesquisas que investiguem o tema. 

\section*{Conclusão}\sectionmark{Conclusão}

Durante nossa formação em Análise do Comportamento, aprendemos a ser ambiciosos: ouvimos que nossa ciência pode ser aplicada nos mais diversos contextos, que temos grandes contribuições às mais diversas áreas do conhecimento, etc. Essa ambição vem, obviamente, com a responsabilidade de comprometer-se e participar ativamente da mudança das contingências às quais estamos expostos. Como, então, a Análise do Comportamento pode contribuir para tornar a participação feminina representativa na tecnologia? 

Em primeiro lugar, é preciso admitir que a análise feita nesse capítulo, de forma nenhuma substitui estudos empíricos acerca do tema. Muito do que foi apresentado nesse capítulo se baseia em extrapolações, e exige testagem empírica, como por exemplo no caso da análise da mudança nas contingências culturais conforme planejadas pela comunidade Python. Entender e caracterizar não apenas o fenômeno da exclusão feminina, mas determinar variáveis que têm sido manipuladas com sucesso para modificar este quadro é papel nosso. Nesse sentido, este capítulo mais instiga a curiosidade científica da leitora do que dá resultados e explicações definitivas.

Outra dimensão na qual a Análise do Comportamento pode ser útil é no desenvolvimento de tecnologia comportamental, especialmente por meio da criação de sistemas de ensino e delimitação de objetivos comportamentais que permitam neutralizar ou, pelo menos, diminuir os efeitos das contingências desfavoráveis às quais as mulheres estão expostas.

A base do pensamento feminista é a constatação de uma desigualdade no controle e acesso de reforçadores com base no sexo (Firestone, 1970): enquanto uma grande variedade de comportamentos emitidos por homens é reforçada socialmente, geralmente com alta magnitude, e esses comportamentos provavelmente também envolvem reforçadores naturais, uma mulher que se comporte da mesma maneira em muitos contextos terá menor probabilidade de ter seu comportamento reforçado, e maior probabilidade de ser punida. Assim, essa diferença nas contingências sociais das quais os comportamentos de homens e mulheres são função, é o que explica grande parte das diferenças comportamentais entre os sexos, e com base nela podemos entender a assimetria de ingresso e evasão entre homens e mulheres nos cursos de TI, por exemplo. 

Como analistas do comportamento, entendemos que \textit{gostar e ser bom} em alguma coisa não são características inatas do indivíduo. Interesse e talento englobam diversas classes de comportamento que só se estabelecem se modeladas por um longo período pelas contingências. Virginia Woolf (1928/1991) ilustra essa situação perfeitamente ao perguntar: e se Shakespeare tivesse uma irmã? E se essa irmã fosse igualmente talentosa? Teria ela alcançado a mesma fama e prestígio de seu irmão? Ou teria ela se casado e se ocupado dos afazeres domésticos? Será que a irmã de Shakespeare seria sequer alfabetizada?

Além de servir como uma fonte de observação e debate sobre práticas culturais para as analistas do comportamento interessadas em análise comportamental da cultura e sobre práticas de ensino para aquelas interessadas em métodos de ensino, entender a inclusão feminina na tecnologia tem implicações reais para o futuro da Análise do Comportamento: somos uma ciência feita, majoritariamente, por mulheres; e somos uma ciência que está muitos passos atrás no que se refere ao uso de procedimentos informatizados para pesquisa e intervenção. Garantir o sucesso da Análise do Comportamento como ciência e como tecnologia passa, portanto, por aplicar nossos métodos de ensino à nossa própria comunidade, de forma contextualizada com a realidade dessa comunidade – e, aqui, gênero é uma variável extremamente relevante para o sucesso dessa empreitada.
\vfill
\newpage

\section*{Referências Bibliográficas}\sectionmark{Referências Bibliográficas}

\hangindent=25pt
\hangafter=1
\noindent Atari. (1978). \textit{3D Tic Tac Toe}. Nova Iorque: Atari.

\hangindent=25pt
\hangafter=1
\noindent Blitz, M. (2017, Fevereiro 3). \textit{The True Story of “Hidden Figures” and the Women Who Crunched the Numbers for NASA}. Recuperado de \url{https://tinyurl.com/feminismo80}

\hangindent=25pt
\hangafter=1
\noindent Cardoso, R. M., Neves Filho, H. B., \& de Freitas, L. A. B. (2018). Ensino e pesquisa no século XXI: Um manifesto pelo ensino de programação na graduação em Psicologia. In H. B. Neves Filho, L. A. B. de Freitas, \& N. C. C. Quinta. (Orgs.) \textit{Introdução ao desenvolvimento de softwares para analistas do comportamento} (pp. 156–173). São Paulo: ABPMC.

\hangindent=25pt
\hangafter=1
\noindent Denner, J., \& Bean, S. (2006). Girls, games, and intrepid exploration on the computer. In E. M. Trauth (Ed.). \textit{Encyclopedia of gender and information on the computer}. (pp. 727–732). IGI Global. Recuperado de \url{https://tinyurl.com/feminismo81}

\hangindent=25pt
\hangafter=1
\noindent Fessenden, M. (2014, Outubro 22). \textit{What happened to all the women in Computer Science?} Recuperado de \url{https://tinyurl.com/feminismo83}

\hangindent=25pt
\hangafter=1
\noindent Firestone, S. (1970). \textit{A Dialética do Sexo}. Rio de Janeiro: Editorial Labor do Brasil.

\hangindent=25pt
\hangafter=1
\noindent Galpin, V. (2002). Women in computing around the world. \textit{ACM SIGCSE Bulletin, 34}(2), 94–100.

\hangindent=25pt
\hangafter=1
\noindent Ghosh, R. A., Glott, R., Krieger, B., \& Robles, G. (2002). \textit{Deliverable D18: Final Report.} Recuperado de \url{https://tinyurl.com/feminismo84}

\hangindent=25pt
\hangafter=1
\noindent Lattal, K. A. (2003). Some dimensions of behavioral technology. \textit{Estudos, 3}(5), 941–958.

\hangindent=25pt
\hangafter=1
\noindent Lien, T. (2013, Dezembro 2). \textit{No girls allowed.} Recuperado de \url{https://tinyurl.com/feminismo85}

\hangindent=25pt
\hangafter=1
\noindent Lin, Y. (2005). Inclusion, diversity and gender equality: Gender Dimensions of the Free/Libre Open Source Software Development. \textit{Mujeres En Red.} Recuperado de \url{https://tinyurl.com/feminismo86}

\hangindent=25pt
\hangafter=1
\noindent Margolis, J., \& Fisher, A. (2002). \textit{Unlocking the clubhouse: women in computing.} Cambridge, Mass: MIT Press.

\hangindent=25pt
\hangafter=1
\noindent Nelson, S. (2008). The Harvard computers. \textit{Nature, 455}(4), 36–37.

\hangindent=25pt
\hangafter=1
\noindent Neves Filho, H. B., de Freitas, L. A. B., \& Quinta, N. C. C. (2018). \textit{Introdução ao desenvolvimento de softwares para analistas do comportamento.} São Paulo: ABPMC. Recuperado de \url{https://tinyurl.com/feminismo87}

\hangindent=25pt
\hangafter=1
\noindent Open Source Survey. (2017). \textit{Open Source Survey.} Recuperado de \url{http://opensourcesurvey.org/2017/}

\hangindent=25pt
\hangafter=1
\noindent PyLadies. (2018a). \textit{Code of Conduct}. Recuperado de \url{http://www.pyladies.com/CodeOfConduct/}

\hangindent=25pt
\hangafter=1
\noindent PyLadies. (2018b). Prospective Organizers — PyLadies Organizer Handbook. Recuperado de \url{https://tinyurl.com/feminismo88}

\hangindent=25pt
\hangafter=1
\noindent PyLadies. (2018c). \textit{PyLadies – Women Who Love Coding in Python.} Recuperado de http://www.pyladies.com

\hangindent=25pt
\hangafter=1
\noindent PyLadies Brasil. (2016). Pyladies e Django Girls Brasil. \textit{In Speaker Deck.} Florianópolis. Recuperado de \url{https://tinyurl.com/fem89}

\hangindent=25pt
\hangafter=1
\noindent Python Fundation. (2018). \textit{Welcome to Python.org.} Recuperado de \url{https://www.python.org/about/}

\hangindent=25pt
\hangafter=1
\noindent PythonBrasil. (2017). \textit{Código de conduta do Evento Python Brasil.} Recuperado de \url{} (Trabalho original publicado em 2014) \url{https://tinyurl.com/feminismo80f}

\hangindent=25pt
\hangafter=1
\noindent Rothschild, J. A. (1981). A feminist perspective on technology and the future. \textit{Women’s Studies International Quarterly, 4}(1), 65–74.

\hangindent=25pt
\hangafter=1
\noindent Schiebinger, L. (2001). \textit{O feminismo mudou a ciência?} Bauru: EDUSC.

\hangindent=25pt
\hangafter=1
\noindent Sheppard, A. (2013, Outubro 13). \textit{Meet the “Refrigerator Ladies” Who Programmed the ENIAC.} Recuperado de \url{https://tinyurl.com/feminismo80a}

\hangindent=25pt
\hangafter=1
\noindent Solon, O., \& Siddiqui, S. (2017, Outubro 31). Russia-backed Facebook posts “reached 126m Americans” during US election. \textit{The Guardian.} Recuperado de \url{https://tinyurl.com/feminismo80b}

\hangindent=25pt
\hangafter=1
\noindent StackOverflow. (2018). \textit{Stack Overflow Developer Survey 2018.} Recuperado de \url{https://tinyurl.com/fem810}

\hangindent=25pt
\hangafter=1
\noindent Stross, R. (2008, Novembro 15). The Forces Driving Women Out of Computer Science. \textit{The New York Times.} Retrieved from \url{https://www.nytimes.com/2008/11/16/business/16digi.html}

\hangindent=25pt
\hangafter=1
\noindent Terry, C., Bolling, M., Ruiz, M., \& Brown, K. (2010). FAP and Feminist Therapies: confronting power and privilege in therapy. In J. W. Kanter, M. Tsai \& R. J. Kohlenberg (Eds.) \textit{The practice of Functional Analytic Psychotherapy} (pp. 97–122). Nova Iorque: Springer. Recuperado de \url{https://tinyurl.com/feminismo80c}

\hangindent=25pt
\hangafter=1
\noindent Woolf, V. (1991). \textit{Um teto todo seu.} São Paulo: Círculo do livro.

\hangindent=25pt
\hangafter=1
\noindent Yitbarek, S. (2016, Julho 31). \textit{That time I went to DjangoCon and fell in love with the community.} Recuperado de \url{https://tinyurl.com/feminismo80d}

\hangindent=25pt
\hangafter=1
\noindent Zarrelli, N. (2016, Março 4). \textit{How female computers mapped the universe and brought America to the moon.} Recuperado de \url{https://tinyurl.com/feminismo80e}

