\setcounter{footnote}{0}
\setcounter{figure}{0}
\setcounter{table}{0}
\chapter*{Poder e patriarcado: contribuições para uma análise comportamental da desigualdade de gênero}\sectionmark{Poder e patriarcado: contribuições para uma análise comportamental da desigualdade de gênero}\blfootnote{Partes deste Capítulo são discussões feitas durante o desenvolvimento da Dissertação de Mestrado da primeira autora no Programa de Pós-graduação em Psicologia Experimental do Instituto de Psicologia da Universidade de São Paulo. As autoras agradecem as contribuições da Profa. Dra. Maria Helena Leite Hünziker, orientadora de mestrado da primeira autora, para a argumentação perseguida neste texto.}
\addcontentsline{toc}{chapter}{Capítulo 3}
\addcontentsline{toc}{section}{Poder e patriarcado: contribuições para uma análise \\comportamental da desigualdade de gênero}
\addcontentsline{toc}{subsection}{\textbf{Autoras:} \textit{Laís Nicolodi \& Ana Arantes}}
\begin{flushright}
\begin{small}
Laís Nicolodi\\
Ana Arantes\\
\end{small}
\vspace{1cm}

\emph{``Gênero é um tópico sobre o qual a Análise do Comportamento, historicamente, se manteve em silêncio conspicuamente.''\\ (Maria del Rosario Ruiz, 2003)}
\end{flushright}

É com essa provocação que Maria Rosário Ruiz inicia, no primeiro parágrafo de um de seus artigos sobre gênero e Análise do Comportamento, uma dura e necessária crítica à desconsideração histórica da área analítico-comportamental pelas questões de gênero (Ruiz, 2003). Um breve levantamento da literatura sobre Feminismo e Análise do Comportamento realizada por Couto e Dittrich (2017) revela que, no período entre 1979 e 2016, apenas oito artigos sobre o tema foram publicados em periódicos científicos reconhecidamente analítico-compor\-tamentais. Rita Wolpert subscreve essa tese quando afirma que não só assuntos sobre gênero e feminismo estavam amplamente ausentes dos estudos analítico-comportamentais, como também estão incompletas as análises de contingências sociais de reforçamento que não levam em consideração os contextos de gênero, raça e classe (Wolpert, 2005). O próprio Skinner afirma que o uso da ciência analítico-comportamental poderia ser uma ferramenta importante para a construção de um mundo socialmente mais justo, tal como no livro Beyond Freedom and Dignity (Skinner, 1971), em que ele aborda questões para se repensar o planejamento de culturas mais igualitárias, inclusive no que diz respeito à igualdade entre gêneros.

Segundo a Análise do Comportamento, o comportamento dos indivíduos é função de relações do organismo com eventos e contextos do ambiente imediato e histórico. Quando este ambiente é social, isto é, formado pelas outras pessoas com quem o indivíduo se relaciona, o controle do comportamento se dá pela interação complexa entre os comportamentos de todos os indivíduos envolvidos (Skinner, 1953; 1957). Ao definirmos operacionalmente os fenômenos observados no controle do comportamento humano em termos de comportamentos sociais e socialmente determinados (ou seja, ao descrevermos como os comportamentos individuais selecionam e mantêm práticas culturais e são ao mesmo tempo selecionados e mantidos por essas práticas culturais), o que pretendemos é aproximar a Análise do Comportamento das teorias e movimentos sociais que procuram analisar e intervir em práticas culturais que oprimem grupos de indivíduos. Essa estratégia de intervenção social foi prevista por Holland (1973) quando preconizou como o analista do comportamento poderia atuar para a modificação de práticas sociais injustas e opressoras:

\begin{quote}
    (...) temos que explorar as formas de modificação do comportamento que sejam compatíveis com um sistema igualitário, não materialista e não elitista, mas, ao contrário, construtivo, pelo menos no tocante aos meios para uma inadiável mudança revolucionária do homem (Holland, 1973, p. 280).
\end{quote}

Dado que os analistas do comportamento estão inseridos em uma cultura e se comportam também sob controle das mesmas variáveis que afetam práticas culturais vigentes, não seria surpresa que, na sua prática científica, reproduzam padrões de comportamento cultural-\linebreak mente selecionados. Ao entendermos que a ciência (enquanto conjunto de relatos verbais acerca das relações entre eventos do mundo) é o produto do comportamento do cientista, conforme proposto por Pennypacker e Johnston (1993), uma revisão crítica do conhecimento produzido em uma disciplina deve passar pela análise das variáveis das quais o comportamento que o produziu é função, o que inclui, obrigatoriamente, a análise das práticas culturais que selecionam o comportamento do cientista. Assim, como indicado por Ruiz e Roche (2007), uma atuação ética do analista do comportamento – tanto na sua prática científica quanto na prestação de serviços – deve levar em conta não só a honesta proposição dos valores que guiam o seu fazer científico e a aplicação das tecnologias cientificamente construídas, mas também a compreensão acurada das variáveis que controlam o comportamento ético e as tomadas de decisão sobre o fazer científico e prático. Para isso é necessário buscar compreender o contexto mais amplo dentro do qual o comportamento do cientista está inserido: como as contingências sociais estão organizadas para a manutenção dessas práticas culturais e como elas definem uma dada cultura (Fernandes, Carrara, \& Zilio, 2017).

A partir da ideia de que a ciência é, portanto, um fenômeno também cultural, pretendemos neste capítulo fazer um diálogo entre os conceitos analítico-comportamentais e os conceitos feministas com o objetivo de permitir uma análise mais comportamentalmente contextualizada das contingências sociais diferenciadas para homens e mulheres e das práticas culturais que mantêm as desigualdades e assimetrias de poder entre os gêneros na cultura patriarcal. Para isso, apresentaremos os conceitos que julgamos tornar mais visíveis as contingências culturais de gênero que são selecionadoras desses padrões de invisibilização e silenciamento das questões feministas dentro da Análise do Comportamento, e recomendamos que essas variáveis e contextos culturalmente relevantes sejam parte das análises de fenômenos sociais e das relações entre eventos que a nossa ciência descreve.

Antes de começarmos a falar do sistema social e de suas contingências mais amplas, precisamos descrever o conceito de \textit{gênero}. Embora cada autora feminista enfatize aspectos distintos da noção de gênero, existe um campo de consenso – ainda que limitado às correntes feministas que rejeitam as noções de ``natureza feminina'' ou destino biológico – de que \textit{gênero} é a \textit{construção social} das categorias \textit{feminino} e \textit{masculino}. Rubin (1975) é uma das autoras que trabalha o conceito de gênero a partir da distinção entre sexo e gênero. A autora desenvolve em seu estudo a noção do sistema sexo-gênero, dentro do qual ela analisa que existe um conjunto de arranjos sociais através dos quais a matéria prima (o sexo fêmea) é transformada em produto social (o gênero feminino). Já para Lerner (1986), a construção social dos gêneros pode ser entendida enquanto diferentes comportamentos (respostas diante de determinados contextos, padrões de estética, funções e papéis sociais, discriminação de estímulos, etc.) para homens e mulheres, o que, numa leitura analítico-comportamental, pode dizer respeito a conjuntos de contingências sociais diferenciadas para os indivíduos do sexo feminino e do sexo masculino. Em outras palavras, a depender de serem emitidas por homens ou mulheres, as mesmas classes de respostas podem ser reforçadas diferencialmente para cada grupo. A literatura não comportamental sobre o fenômeno da desigualdade entre os gêneros é vasta, e uma leitura analítico-comportamental das contingências de comportamento descritas nessa literatura leva a confirmação da hipótese de reforçamento diferencial de respostas a depender do gênero de quem se comporta. Por exemplo, Lavy e Sand (2015) relatam que professores e professoras dos níveis iniciais de ensino tendem a dar mais ênfase a acertos feitos por meninos em testes de matemática e ciências do que aos acertos feitos por meninas, além de darem mais \textit{feedbacks} para comentários em sala durante as aulas quando eram feitos por meninos, em comparação com os comentários feitos por meninas. Os autores verificaram que, ao longo do tempo, essas práticas tinham efeito direto sobre o desempenho escolar das alunas e dos alunos ao longo das séries escolares, de modo que, nos níveis médios de ensino, o desempenho dos meninos era significativamente melhor nos testes de matemática e ciências e havia significantemente mais participação em sala de aula por parte dos meninos do que das meninas durante as discussões nesses temas. Jafee (1989) examinou as diferenças nos níveis de autonomia e autoridade em ambientes de trabalho entre homens e mulheres e concluiu que, para um mesmo nível dentro das organizações, mulheres têm menos autonomia (por exemplo, para tomar decisões sem consulta ou chancela de superiores) do que homens, e que a única variável que afetava essa medida era a quantidade de anos de estudo ou treinamento, enquanto para homens, além de maiores níveis de autonomia dentro da mesma função organizacional, outras variáveis como confiança dos pares e o nível ocupacional eram mais importantes do que a escolaridade. Quanto à autoridade (o quanto as decisões tomadas são seguidas e apoiadas pelos subordinados), o autor observou que mulheres tinham menores níveis de autoridade e participavam menos de tomadas de decisão do que homens, ainda que os homens ocupassem cargos inferiores ou tivessem menos escolaridade. Os exemplos descritos são pequenas amostras de como as contingências sociais a que mulheres e homens estão submetidos geram consequências diferentes a depender do gênero do indivíduo que se comporta e, em longo prazo, os produtos dessas práticas levam a diferenças materiais comprovadas pelo fenômeno chamado \textit{pay gap}, a diferença de acesso à riqueza. Dados de levantamentos internacionais mostram que globalmente, em média, mulheres ganham 24\% a menos do que os homens (ONU, 2015, \textit{Minimum Set of Gender Indicators}).

Para Ruiz (1998), muitas das fontes de controle sobre as práticas culturais de gênero são ainda invisíveis, não no sentido de que não se consideram essas diferenças de repertórios comportamentais advindas do reforçamento diferencial de comportamentos de homens e mulheres, mas sim que essas diferenças ainda são lidas como naturais, pois seriam fruto de uma \textit{essência} inata das mulheres e dos homens. Silva e Laurenti (2016), ao enfatizar as semelhanças entre Beauvoir (1970) e o modelo de seleção por consequências de Skinner (1981), rompem com essa lógica ao apontar a visão comum dos autores sobre o antiessencialismo, isto é, o sexo biológico das mulheres e dos homens influencia pouco nas distinções de gênero. Além disso, esse reforçamento diferencial de respostas entre mulheres e homens deve ser visto como determinante na manutenção das práticas culturais de gênero, ou seja, os comportamentos que são adquiridos e mantidos de formas distintas a depender dos indivíduos serem mulheres ou homens levam a repertórios de respostas, de controle de estímulos e de reforçadores diferentes. E é a partir dessa diferença que as analistas do comportamento feministas passaram a se debruçar sobre como a questão de gênero é central no estudo das desigualdades sociais observadas entre as consequências individuais e sociais dos comportamentos de homens e de mulheres e, principalmente, sobre quais são as variáveis implicadas nas práticas culturais que oprimem mulheres em detrimento dos homens.

Dados sobre essas diferenças podem ser vistos, por exemplo, no índice criado pela Organização das Nações Unidas (ONU, 2015, citado por Couto \& Dittrich, 2017), chamado \textit{Gender Inequality Index} (GII), cujo objetivo é analisar as desigualdades entre os gêneros (Couto \& Dittrich, 2017). Esse índice avaliou, em cada país, três principais indicadores sociais: saúde reprodutiva, empoderamento e participação no mercado de trabalho\footnote{Saúde reprodutiva é medida pela taxa de mortalidade materna e taxas de natalidade de adolescentes; empoderamento é medido pela proporção de assentos parlamentares ocupados por mulheres e pela proporção de mulheres adultas e homens com idade igual ou superior a 25 anos com pelo menos algum ensino secundário; e status econômico é medido pela participação no mercado de trabalho e pela taxa de participação da população feminina e masculina com idade igual ou superior a 15 anos. O índice de desigualdade entre os gêneros é construído no mesmos moldes que o Índice de Desenvolvimento Humano (IDH) para expor melhor as diferenças na distribuição de conquistas entre mulheres e homens.}.

O valor desse índice pode variar de 0 a 1, sendo que quanto mais próximo de zero, mais igualdade entre os gêneros, e quanto mais próximo de um, mais desigualdade. De todos os 155 países avaliados, nenhum chegou ao índice de igualdade plena, ou seja, zero. O índice mais próximo da igualdade foi obtido pela Eslovênia, com 0,016 pontos, e o de maior desigualdade foi o obtido pelo Iêmen, com 0,744 pontos. A criação desse índice pela ONU permitiu a constatação de que ainda são muitos os países que apresentam um alto índice de desigualdade entre os gêneros, bem como é possível dizer que as mulheres não atingiram a igualdade plena de direitos com os homens em absolutamente nenhum dos 155 países avaliados.

No contexto brasileiro, outro índice assombroso e que evidencia esse quadro assimétrico e forte de desigualdade entre os gêneros são os números sobre feminicídio\footnote{De acordo com a Lei 13.104 de 9 de março de 2015, que modifica o Decreto-Lei no 2.848, de 7 de dezembro de 1940 do Código Penal, o art. 121, § 2° passou a ter o inciso VI que trata o ``Feminicídio enquanto o ato de homicídio contra a mulher por razões da condição de sexo feminino''. Ainda, ``considera-se que há razões de condição de sexo feminino quando o crime envolve (I) violência doméstica e familiar; (II) menosprezo ou discriminação à condição de mulher''.}  encontrados no \textit{Mapa da Violência sobre Homicídios de Mulheres no Brasil} (Waiselfisz, 2015). Os registros mostram que o número de mortes de mulheres por feminicídio no país passou de 1.353, no ano de 1980, para 4.762, em 2013, o que representa um crescimento de 252\% no número de mulheres assassinadas no Brasil. Para efeitos de comparação, com sua taxa de 4,8 homicídios por 100 mil mulheres, o Brasil, num grupo de 83 países analisados com dados homogêneos, fornecidos pela Organização Mundial da Saúde, ocupa a quinta posição entre os países onde mais mulheres são assassinadas (Waiselfisz, 2015).

Esses são alguns dos dados que evidenciam a desigualdade de direitos e de poder entre os gêneros. Em outros termos, não se pode negar que mulheres dispõem de menor acesso a reforçadores sociais em comparação ao acesso masculino a esses mesmos reforçadores. Existem, no entanto, duas questões a serem levantadas no que tange a essa notória desigualdade: 1) afinal, ao que se deve essa desigualdade entre os gêneros? E, 2) usar o conceito de \textit{gênero}, sozinho, abarca essas contingências de desequilíbrio de acesso a reforçadores?

Para autoras como Heleieth Saffioti (2004), falar apenas sobre gênero, deixando à margem uma investigação mais cuidadosa sobre a forma de organização social que sustenta a assimetria de poder entre homens e mulheres, e a consequente opressão dessas últimas, é uma forma de distrair a atenção sobre o real problema da desigualdade existente que invisibiliza as contingências e práticas culturais que estão por trás desse funcionamento. Isso se dá porque o conceito de gênero, por si só, não explicita necessariamente a desigualdade de poder e a diferença de privilégios entre homens e mulheres. Por exemplo: falar em violência de gênero, apenas, poderia englobar falar tanto sobre a violência de homens contra as mulheres quanto de mulheres contra homens, e isso se deve ao caráter neutro do termo \textit{gênero} quando esse é utilizado isoladamente. ``Não se trata de abolir o uso do conceito de gênero, mas de eliminar sua utilização exclusiva. Gênero é um conceito por demais palatável, porque é excessivamente geral, ahistórico, apolítico e pretensamente neutro'' (Saffioti, 2004, p. 148). A autora defende que a desigualdade entre gêneros, com notório prejuízo das mulheres, é melhor expressa pelo termo \textit{patriarcado}, que significa ao pé da letra: \textit{o poder do patriarca}. O uso exclusivo do conceito de gênero não evidencia o agente deste poder, manifesto em especial como homem/marido, e neutraliza a relação de dominação-exploração masculina sobre a mulher. Daí ser necessário adotar o conceito de patriarcado nos estudos sobre a desigualdade de poder e a diferença de privilégios entre homens e mulheres, conforme sugerido por Pateman: ``Abandonar o termo patriarcado significaria abandonar o único conceito que se refere especificamente à sujeição da mulher, e que singulariza a forma de direito político que todos os homens exercem pelo fato de serem homens'' (Pateman, 1993, pp.39-40).

Derivado do grego, \textit{pater} refere-se a pai, enquanto \textit{arkhe} refere-se à origem e comando, de forma que o conceito de patriarcado inicialmente apresentava o sentido de \textit{autoridade do pai}. No início do século XIX, antes das denúncias de autores socialistas, o termo era muito usado como adjetivo, de maneira elogiosa, em expressões como ``as virtudes patriarcais'', para se referir às sabedorias e costumes da vida no campo (Delphy, 2009). O sentido feminista foi inaugurado por Kate Millet (1970), que usou o termo para designar \textit{o(s) sistema(s) que oprime(m) as mulheres}. Esse significado foi rapidamente adotado pelos movimentos feministas nos anos 1970 para designar o conjunto do sistema a ser combatido, ou seja, uma formação social na qual homens detêm o poder, sendo assim sinônimo de \textit{dominação masculina} ou de \textit{opressão das mulheres} (Delphy, 2009).

Na proposta de Saffioti (2004), o patriarcado pode ser definido enquanto uma \textit{hierarquia entre homens e mulheres} que existe há milênios, com \textit{primazia masculina}, e que estabelece uma \textit{estrutura de poder que situa as mulheres muito abaixo dos homens em todas as áreas da convivência humana}. Na mesma direção, Lerner (1986) afirma:

\begin{quote}
    O patriarcado se refere à manifestação e à institucionalização da dominação masculina sobre as mulheres e as crianças na família, e na extensão da dominação masculina sobre as mulheres na sociedade em geral. Isso implica que homens detêm poder em todas as instituições importantes da sociedade e que as mulheres estão desprovidas de acesso a tal poder (p.239, tradução das autoras)\footnote{Trecho original retirado da obra ``The Creation of Patriarchy'' de G. Lerner (1986): ``In its wider definition means the manifestation and institutionalization of male dominance over women and children in the family and the extension of male dominance over women in society in general. It implies that men hold power in all the important institutions of society and that women are deprived of access to such power.'' (p. 239)}.
\end{quote}

É nesse sentido que entendemos o conceito de patriarcado como mais específico que o conceito de gênero, pois ele ressalta a \textit{desigualdade de poder} e a \textit{diferença de privilégios} entre os gêneros, incluindo a dominação exercida pelos homens sobre as mulheres (Saffioti, 2004). Além disso, é importante para sua análise o fato de que o patriarcado não opera somente na esfera privada, pois suas hierarquias e suas estruturas de poder estão presentes também na cultura como práticas culturais. Dessa forma, a não utilização desse conceito permite que essa relação de exploração-dominação encontre formas e meios de continuar invisível, existindo sem ser facilmente percebida (Saffioti, 2004). 

Para que seja possível propor uma leitura analítico-comportamental do patriarcado e das relações sociais patriarcais é importante que aprofundemos o conceito analítico-comportamental de \textit{controle}, e suas implicações para os fenômenos do poder e da dominação. Saffioti (2004) enfatiza que o valor central da cultura gerada pela dominação-explo-\linebreak ração patriarcal é o controle. O conceito de controle, para a Análise do Comportamento, carrega uma noção distinta da noção do senso comum sobre o termo, que normalmente está vinculada à ideia de autoritarismo, dominação, tiranismo, etc. (Hunziker, 2011). Para a análise do comportamento, controle está vinculado à lógica das relações funcionais, por exemplo, se o comportamento de A é função do comportamento de B, então o comportamento de B controla o comportamento de A. O termo \textit{controle} é usado para descrever a relação em que um evento tem a sua probabilidade de ocorrência alterada por outro (Hunziker, 2011). Dessa forma, podemos considerar que, nas relações sociais que envolvem interação entre homens e mulheres, o controle está sendo exercido sempre de forma bidirecional. Dentro de uma análise funcional é correto dizer que tanto o comportamento da mulher está sob controle do comportamento do homem, seja por meio de qualquer esquema que envolva a probabilidade de ocorrência de algum comportamento da mulher; quanto que o comportamento do homem está sob controle do comportamento da mulher. Portanto, afirmar apenas que o patriarcado envolve \textit{relações de controle} não é suficiente, nos termos analítico-comportamentais, porque não evidencia as nuances desse tipo de controle e o desequilíbrio e assimetria entre as partes da relação em análise. Por esse motivo é necessário categorizar de que forma o controle está sendo exercido, qual o grau desse controle, quais as condições que o mantém e de que maneira as partes são afetadas pelas consequências desse controle. Baum (2006) afirma que o grau de controle que um indivíduo exerce sobre o comportamento do outro está vinculado ao tipo de relação estabelecida entre os indivíduos, equitativa ou de desigualdade:

\begin{quote}
    Duas pessoas podem ser chamadas de parceiras equitativas quando suas interações incluem atos e reforçadores que, de ambas as partes, são do mesmo tipo. Se dois irmãos são afetuosos um com o outro, pedem e dão dinheiro um para o outro, e emprestam brinquedos ou ferramentas um para o outro, então podemos dizer que são parceiros equitativos. (...) Em relações de desigualdade, a sobreposição entre ações e reforçadores de ambas as partes inexiste ou é pequena. Ana, a supervisora, distribui as tarefas, paga salários e recebe parte do lucro das vendas; Pedro, o empregado, trabalha e recebe salário (Baum, 2006, p. 218).
\end{quote}

Nas relações de desigualdade, o grau de controle que o indivíduo que obtém mais benefícios exerce sobre o comportamento do outro é superior ao grau de controle exercido pelo indivíduo que se beneficia menos na relação. Embora ambos exerçam controle um sobre o comportamento do outro, ou seja, ambos estão se comportando em função do comportamento do outro, quando existe um desequilíbrio de acesso aos reforçadores entre as partes é possível dizer que esse controle não está sendo exercido de maneira igual entre os indivíduos. A essas diferenças de grau de controle e de acesso a benefícios na relação, Baum irá chamar de \textit{poder}:

\begin{quote}
    A discussão sobre poder diz respeito ao grau de controle que cada parceiro exerce sobre o comportamento do outro. Quando os parceiros se beneficiam de forma desigual do relacionamento, aquele que obtém mais benefícios também tem mais poder. Esse maior poder, tanto quanto o maior benefício, é o que nos leva a denominar esse parceiro de controlador (Baum, 2006, p. 231). 
\end{quote}

A partir dessa definição, é possível incluir a variável poder nas análises sobre as relações de desigualdade. Deter mais poder na relação não envolve apenas concentrar mais reforçadores positivos em relação ao outro, aquele que detém mais poder é também quem gerencia as relações de reforço vigentes, isto é, dispõe as contingências para o comportamento do outro, limitando as possibilidades de escolha e reduzindo a possibilidade de formas alternativas de comportamento. Para Goldiamond (1976), o número de respostas alternativas disponíveis em uma dada contingência social se relaciona com a noção de liberdade. Uma pessoa pode ser considerada mais livre do que outra na medida em que ela pode fazer escolhas genuínas. Para que uma escolha possa ser considerada genuína, deve haver um conjunto de contingências alternativas igualmente possíveis que proporcionem acesso aos mesmos reforçadores ou a reforçadores de mesmo valor. Desse modo, quanto maior é o número de contingências alternativas disponíveis em uma dada relação social, maior é o grau de liberdade de escolha que o indivíduo possui (Fernandes \& Dittrich, 2018).

Para ilustrar uma relação de desigualdade pela diferença de poder, diferença de acesso a reforçadores e restrição de liberdade, podemos pensar em uma situação de violência doméstica na qual o marido agride a esposa quando ela se recusa a fazer algo por ele, e após a agressão ela acaba fazendo o que ele havia pedido. Em termos de análise funcional, o marido pedir por algo é um estímulo discriminativo diante do qual a mulher responde verbalizando o ``não'', e a resposta negativa da mulher é punida com a agressão física do marido. Em ocasiões futuras, há a diminuição da probabilidade de respostas negativas da esposa diante de solicitações do marido, e aumento da probabilidade de emissão de respostas de esquiva dessa punição pela emissão de respostas da classe ``atender às solicitações do marido''. Nesse caso, é correto afirmar que, na relação entre o homem agressor e a mulher agredida, ambos emitem respostas sob controle do comportamento um do outro, da mesma forma que o homem responde em função do comportamento da esposa de dizer não (agredindo a esposa), a mulher passa a responder em função do comportamento do marido de pedir por algo (atendendo a pedidos do marido). 

Como mencionado anteriormente, embora o controle seja bidirecional, é evidente que ele não está sendo exercido com o mesmo grau entre as partes na medida em que o marido é beneficiado e a mulher não é beneficiada na relação em questão. Não se trata de uma relação equitativa, pois a interação não inclui reforçadores do mesmo tipo para ambas as partes, da mesma forma que não inclui contingências que proporcionam igualmente as mesmas possibilidades de escolha. Dado que o homem, nessa relação, após fazer uso da violência física tem como consequência um benefício que é o de ter seu pedido atendido pela esposa, é correto afirmar que ele se beneficia de forma desigual com relação à mulher, e promove um grau de controle distinto do controle exercido pela esposa. Portanto, ele detém o poder na relação. O controle exercido pela esposa não é um controle que garante benefício para ela na relação e podemos afirmar que o grau de controle que ela exerce sobre o marido é diferente do grau de controle que ele exerce sobre ela, porque ela não obtém acesso aos mesmos reforçadores e não se beneficia de forma equitativa em relação a ele. Portanto, é correto afirmar que ela não detém o poder na relação. Além disso, é visível que ela está com suas possibilidades de escolher genuinamente restringidas pela contingência em questão, pois quem dispõe a contingência dentro da qual ela responde é quem detém mais poder na relação, ou seja, o marido. E quem está com a sua liberdade restringida é a mulher.

Baum (2006) usa o conceito de \textit{exploração} para se referir ao tipo de interação social que não serve aos interesses dos parceiros de forma equitativa, sendo que nem sempre a parte explorada – que é a que possui sua liberdade restrita e possui menos poder na relação – se sente lesada. Essa seria a definição do \textit{escravo feliz} que trabalha para o dono de engenho e se sente contente porque seu comportamento de trabalhar em excesso está sendo negativamente reforçado na medida em que ele não é agredido fisicamente pelo dono de engenho; ou, por exemplo, o pai que explora a filha recompensando-a com cuidados e afetos, desde que ela trabalhe pedindo dinheiro na rua ou participe de atos sexuais, contingência de reforçamento positivo na qual a filha está com liberdade restrita e detém menos poder na relação com o pai. 

O exemplo sobre violência doméstica mencionado anteriormente é fatídico porque não ilustra um caso isolado, mas sim reflete uma realidade bastante comum no cenário nacional. Sistematicamente, as maiores vítimas de violência doméstica são mulheres que sofrem a violência pela mão de seus parceiros. De acordo com o dossiê ``Violência e Assassinatos de Mulheres'' (Data Popular/Instituto Patrícia Galvão, 2013), entre os entrevistados de ambos os sexos e de todas as classes sociais, 54\% conhecem uma mulher que já foi agredida por um parceiro e 56\% conhecem um homem que já agrediu uma parceira. O ``Relógio da Violência'' do Instituto Maria da Penha (IMP, 2017) registra que, em média, a cada 7,2 segundos, no Brasil, uma mulher é vítima de violência física e a cada 6,3 segundos uma mulher é vítima de ameaça de violência pelos seus parceiros. É a partir de dados como esses que se torna possível visualizar com mais clareza a diferença de poder existente nas relações desiguais entre homens e mulheres. Quando é possível observar uma repetição de padrões de comportamento sistemáticos de violência contra as mulheres, é possível também perceber que as mulheres estão inseridas em contingências que reduzem suas possibilidades de escolhas genuínas, ou seja, elas estão com suas liberdades restritas. No vocabulário feminista, isso se trata da \textit{opressão feminina} (Lerner, 1986). São comuns argumentos que questionam o porquê das mulheres não abandonarem seus parceiros, já que elas não estão se beneficiando de forma equitativa na relação, em comparação a eles. Argumentos como esses podem ser problemáticos porque eles não consideram a forma como as contingências estão dispostas na relação de desigualdade que a mulher tem estabelecida com o parceiro. Nem sempre são facilmente percebidas quais as outras variáveis que mantêm essas mulheres nessas relações de desigualdade: Elas têm para onde ir? Elas têm condições físicas e emocionais de enfrentar o marido? Elas têm filhos para criar? A análise dessa situação deve levar em consideração as contingências que estão garantindo a manutenção desse padrão sistemático de relações desiguais entre homens e mulheres com diferença de poder e restrição de liberdade. Diferentes condições materiais entre os gêneros (advindas, por exemplo, do \textit{pay gap}) e a restrição de escolhas ao longo da história de relações das mulheres com os homens dentro de práticas culturais de dominação masculina podem gerar repertórios comportamentais restritos, e o contexto social imediato em que as mulheres estão inseridas pode impedir que comportamentos sejam emitidos de maneira adequada e efetiva.

Não pretendemos dar conta de nomear e descrever cada uma das contingências sociais que mantêm esse padrão de funcionamento, mas compreendemos que essas contingências sociais podem ser reunidas em uma mesma categoria a partir de um ponto em comum a todas elas: essas contingências sociais estão organizadas de tal modo a garantir a manutenção de práticas culturais e padrões de comportamento que fortalecem a supremacia masculina e estabelecem uma hierarquia de poder entre os gêneros, dentro da qual as mulheres são sistematicamente oprimidas e não beneficiadas. A esse conjunto de contingências sociais damos o nome de \textit{contingências patriarcais}. Patriarcado seria, dessa forma, um sistema social em que um conjunto de contingências socialmente organizadas estabelece práticas culturais e padrões de comportamento de interações desiguais entre os gêneros, e dentro das quais os homens, sistematicamente, se beneficiam e concentram mais poder do que as mulheres, o que, por sua vez, as colocam em posições de inferioridade e com liberdade restrita. 

Autoras feministas como Lerner (1986) argumentam que o patriarcado também pode ser entendido como a institucionalização dessas práticas culturais da dominação masculina sobre as mulheres. Skinner (1953) denominou essas práticas de controle social sobre um grupo ou cultura de \textit{agências de controle} e descreveu algumas delas: governo e lei, educação, religião, poder econômico e psicoterapia. Uma agência de controle é formada por um conjunto de indivíduos que se comportam, em determinados contextos, de acordo com práticas culturais e regras e que mantêm e replicam essas mesmas práticas, ensinando-as às próximas gerações. Desse modo, a agência de controle tem a função tanto de controlar o comportamento dos indivíduos quanto de manter essas práticas na cultura (Skinner, 1953). As maneiras pelas quais uma agência controladora estabelece controle e influência sobre o comportamento individual podem variar desde o estabelecimento de regras (como leis, regulamentos e regimentos, por exemplo), uso de controle aversivo e punição, poder exclusivo sobre determinados reforçadores e até o estabelecimento dos valores reforçadores dos estímulos. Agências de controle muitas vezes atuam por meio de um conjunto de regras que pode ser descrito como \textit{ideologia}, ou seja, um conjunto de regras (nem sempre explícitas) que estabelecem determinadas práticas sociais como naturais ou habituais a um determinado contexto social. Dessa maneira, pode-se dizer que agências de controle usam ideologias para estabelecer os critérios de verdade (na forma de regras) que vão determinar os comportamentos dos membros daquela cultura. Quando usamos declarações de intenção do tipo ``porque eu quis'' ou ``porque eu gosto'' podemos estar apenas enunciando a ideologia (o conjunto não explícito de regras) que seleciona os comportamentos de gostar e estabelece os valores reforçadores das consequências que mantêm esse comportamento, em detrimento de uma análise das contingências ambientais e das variáveis das quais o comportamento individual é função direta. As agências controladoras agem usando estratégias de \textit{naturalização} do \textit{status quo}, mantendo práticas sociais que permitem que os controlados evitem punição ao se comportarem de determinadas maneiras que são, então, reforçadas diferencialmente. 

Esta forma sistemática de controle do comportamento social praticada pelas agências de controle também aproxima esse conceito daquilo que a teoria marxista define como \textit{sistemas de dominação}\footnote{Para Marx (1971), a dominação é uma situação que gera relações assimétricas de comando entre os indivíduos.} (Marx, 1971). Neste ponto, podemos começar a definir operacionalmente o que o feminismo vai conceituar como dominação masculina, ou seja, as contingências estabelecidas e mantidas pela agência de controle patriarcal para controlar o comportamento de mulheres, visando à perpetuação de práticas opressoras para as mulheres e vantajosas para os homens que mantêm e transmitem a ideologia patriarcal. A Secretaria Geral das Nações Unidas (ONU, 1979), no documento da ``Convenção para Eliminação de Todas as Formas de Discriminação contra a Mulher'', leva em conta o caráter institucional e sistemático da dominação masculina quando apresenta o problema da violência contra as mulheres:

\begin{quote}
    A violência contra mulheres e meninas não se limita a nenhum sistema político ou econômico em particular, mas é prevalente em todas as sociedades do mundo. Ele atravessa fronteiras de riqueza, raça e cultura. É uma expressão de valores e padrões histórica e culturalmente específicos que ainda hoje são executados por meio de muitas instituições sociais e políticas que promovem a subserviência das mulheres e a discriminação contra mulheres e meninas (ONU, 1979, sp., tradução das autoras )\footnote{Texto original: ``Violence against women and girls is not confined to any particular political or economic system, but it is prevalent in every society in the world. It cuts across boundaries of wealth, race and culture. It is an expression of historically and culturally specific values and standards which are today still executed through many social and political institutions that foster women’s subservience and discrimination against women and girls''.}.
\end{quote}

Dominação masculina é um conjunto de sentidos e valores provenientes de práticas e instituições (Thompson, 2001) das quais os indivíduos fazem parte, mas que pode se perpetuar ainda que alguns desses indivíduos se recusem a participar explicitamente na manutenção dessas práticas. A autora também destaca que as diferenças entre os papéis sociais exercidos por cada gênero é tanto aprendida e mantida pelo ambiente social, como parte das práticas que determinam o comportamento sob o controle desta agência (Thompson, 2001). Quando Saffioti (2004) afirma que o patriarcado opera não só na esfera pública, como também na esfera privada, ela estava fazendo referência aos comportamentos intrafamiliares de violência contra a mulher, mas podemos estender sua análise para outras relações de controle que apresentam desigualdade de poder e que são aprendidas em e mantidas por contigências patriarcais amplamente reproduzidas na cultura, como o que se convencionou chamar \textit{cultura do estupro}\footnote{Para mais análises sobre a cultura de estupro, ver o Capítulo 04 deste livro.}, por exemplo.

Nossa proposta de operacionalização dos conceitos de poder e de patriarcado tem como objetivo explicitar a necessidade da inclusão das variáveis desses contextos nas análises não só de comportamentos individuais, como de fenômenos sociais (como o machismo), contribuindo para a proposição de intervenções individuais e de larga escala que tenham probabilidade de produzir mudanças culturais duradouras. Considerar a Análise do Comportamento como uma ciência neutra em questões de gênero compromete nossa percepção dos controles aos quais nossos comportamentos estão submetidos ao praticar essa ciência (Holland, 1973; Ruiz \& Roche, 2007). Ao nos comprometermos ética e politicamente com a prática feminista, o que pretendemos é instrumentalizar os analistas do comportamento para estar sob controle das contingências sociais amplas e complexas de onde os problemas humanos individuais surgem. Ao reconhecer que a descrição das contingências sociais em vigor deve incluir as práticas culturais de gênero e as variáveis de controle dispostas pelo patriarcado, damos ao analista do comportamento outras ferramentas e uma nova visão sobre as variáveis das quais o comportamento é função, evidenciando os controles desiguais e as relações de poder a que tanto os cientistas comportamentais quanto os indivíduos e a sociedade que eles se dispõem a descrever estão submetidos. Continuar negligenciando as variáveis ambientais históricas e imediatas provenientes do contexto das relações entre os gêneros impede que o conhecimento produzido pela Análise do Comportamento possa ser traduzido em intervenções efetivas e em mudança cultural que leve à melhora efetiva da qualidade de vida das pessoas, sobretudo à preservação da vida de mulheres.
\vfill
\pagebreak
\section*{Referências Bibliográficas}\sectionmark{Referências Bibliográficas}

\hangindent=25pt
\hangafter=1
\noindent Baum, W. M. (2006). \textit{Compreender o behaviorismo}. Porto Alegre: Artmed Editora. (Trabalho original publicado em 1994).

\hangindent=25pt
\hangafter=1
\noindent Beauvoir, S. (1970). \textit{O segundo sexo: Fatos e mitos}. São Paulo: Difusão Européia do Livro. (Trabalho original publicado em 1949).

\hangindent=25pt
\hangafter=1
\noindent Couto, A. G., \& Dittrich, A. (2017). Feminismo e análise do comportamento: Caminhos para o diálogo. \textit{Perspectivas em Análise do Comportamento, 8}(2), 147-158.

\hangindent=25pt
\hangafter=1
\noindent Data Popular/Instituto Patrícia Galvão. (2013). Percepção da Sociedade sobre Violência e Assassinatos de Mulheres. Recuperado de: 
\url{https://tinyurl.com/feminismo30}

\hangindent=25pt
\hangafter=1
\noindent Delphy, C. (2009). Patriarcado. Em D. Senotier, F. Laborie, H. Hirata, \& H. Doare (Orgs.), \textit{Dicionário Crítico do Feminismo} (pp. 173-179). São Paulo: Editora Unesp.

\hangindent=25pt
\hangafter=1
\noindent Fernandes, R. C., \& Dittrich, A. (2018). Expanding the behavior-analytic meanings of ``freedom'': The contributions of Israel Goldiamond. \textit{Behavior and Social Issues, 27}, 4-19.

\hangindent=25pt
\hangafter=1
\noindent Fernandes, D. M., Carrara, K., \& Zilio, D. (2017). Apontamentos para uma definição comportamentalista de cultura. \textit{Acta Comportamentalia, 25}(2), 265-280.

\hangindent=25pt
\hangafter=1
\noindent Goldiamond, I. (1976). Protection of human subjects and patients: A social contingency analysis of distinctions between research and practice, and its implications. \textit{Behaviorism, 4}, 1–41.

\hangindent=25pt
\hangafter=1
\noindent Holland, J. (1973). Servirán los princípios conductuales para los revolucionarios? Em F. S. Keller \& E. R. Iñesta (Orgs.), \textit{Modificación de la conduta: Aplicaciones a la educación}. México: Trillas.

\hangindent=25pt
\hangafter=1
\noindent Hunziker, M. H. L. (2011). Afinal, o que é controle aversivo? \textit{Acta Comportamentalia, 19}, 9-19.

\hangindent=25pt
\hangafter=1
\noindent Instituto Maria da Penha. (2017). \textit{Relógio da Violência}. Recuperado de: \url{https://www.relogiosdaviolencia.com.br}

\hangindent=25pt
\hangafter=1
\noindent Jaffee, D. (1989). Gender inequality in workplace: Autonomy and\linebreak authority. \textit{Social Science Quarterly, 70}(2), 375-392.

\hangindent=25pt
\hangafter=1
\noindent Pennypacker, H. S., \& Johnston, J. M. (1993). \textit{Strategies and tactics of behavioral research}. Hillsdale, NJ: Lawrence Erlbaum Associates.

\hangindent=25pt
\hangafter=1
\noindent Lavy, V., \& Sand, E. (2015). \textit{On the origins of gender human capital gaps: Short and long term consequences of teachers’ stereotypical biases}. National Bureau of Economic Research, Working Paper No. 20909. Recuperado de: \url{http://www.nber.org/papers/w20909}

\hangindent=25pt
\hangafter=1
\noindent Lerner, G. (1986). \textit{The Creation of Patriarchy}. New York: Oxford University Press.

\hangindent=25pt
\hangafter=1
\noindent Marx, K. (1971). \textit{Elementos fundamentales para la critica de la economia política, Vol. I}. Buenos Aires: Editora Siglo XXI.

\hangindent=25pt
\hangafter=1
\noindent Organização das Nações Unidas. (1979). \textit{Convention on the Elimination of All Forms of Discrimination Against Women}. Recuperado de: \url{https://tinyurl.com/feminismo31}

\hangindent=25pt
\hangafter=1
\noindent Organização das Nações Unidas. (2015). \textit{Minimum Set of Gender Indicators}. Recuperado de: \url{https://genderstats.un.org}

\hangindent=25pt
\hangafter=1
\noindent Pateman, C. (1993). \textit{O contrato sexual}. São Paulo: Paz e Terra. (Trabalho original publicado em 1988.)

\hangindent=25pt
\hangafter=1
\noindent Rubin, G. (1975). The traffic in women. Em R. Reiter (Org.), \textit{Towards an Antropology of Women} (pp. 160-207). New York: Monthly Review Press.

\hangindent=25pt
\hangafter=1
\noindent Ruiz, M. R. (1998). Personal agency in feminist theory: Evicting the illusive dweller. \textit{The Behavior Analyst, 21}(2), 179–192.

\hangindent=25pt
\hangafter=1
\noindent Ruiz, M. R. (2003). Inconspicuous sources of behavioral control: The case of gendered practices. \textit{The Behavior Analyst Today, 4}, 12–16.

\hangindent=25pt
\hangafter=1
\noindent Ruiz, M. R., \& Roche, B. (2007). Values and the scientific culture of behavior analysis. \textit{The Behavior Analyst, 30}(1), 1-16.

\hangindent=25pt
\hangafter=1
\noindent Saffioti, H. B. (2004). \textit{Gênero, patriarcado e violência}. São Paulo, SP: Fundação Perseu Abramo.

\hangindent=25pt
\hangafter=1
\noindent Silva, E. C., \& Laurenti, C. (2017). B. F. Skinner e Simone de Beauvoir: ``A mulher'' à luz do modelo de seleção pelas consequências. \textit{Perspectivas em Análise do Comportamento, 7}(2), 197-211. 

\hangindent=25pt
\hangafter=1
\noindent Skinner, B. F. (1953). \textit{Ciência e comportamento humano}. São Paulo: Martins Fontes.

\hangindent=25pt
\hangafter=1
\noindent Skinner, B.F. (1957). \textit{Verbal behavior}. Acton, Massachusetts: Copley.

\hangindent=25pt
\hangafter=1
\noindent Skinner, B.F. (1971). \textit{Beyond freedom and dignity}. New York: Alfred A. Knopf.

\hangindent=25pt
\hangafter=1
\noindent Skinner, B. F. (1981). Selection by consequences. \textit{Science, 213}(4507), 501-504.

\hangindent=25pt
\hangafter=1
\noindent Thompson, D. (2001). \textit{Radical feminism today}. Londres: SAGE Publications.

\hangindent=25pt
\hangafter=1
\noindent Waiselfisz, J. J. (2013). \textit{Mapa da violência}. UNESCO Brasil.

\hangindent=25pt
\hangafter=1
\noindent Wolpert, R. S. (2005). A multicultural feminist analysis of Walden Two. \textit{The Behavior Analyst Today, 6}, 186–190