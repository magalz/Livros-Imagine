\chapter{Poder e patriarcado: contribuições para uma análise comportamental da desigualdade de gênero}\sectionmark{Poder e patriarcado: contribuições para uma análise comportamental da desigualdade de gênero}
\begin{flushright}
\begin{scriptsize}
Laís Nicolodi \& Ana Arantes
\end{scriptsize}
\vspace{1cm}

\emph{``Gênero é um tópico sobre o qual a Análise do Comportamento, historicamente, se manteve em silêncio conspicuamente.''\\ (Maria del Rosario Ruiz, 2003)}
\end{flushright}

É com essa provocação que Maria Rosário Ruiz inicia, no primeiro parágrafo de um de seus artigos sobre gênero e Análise do Comportamento, uma dura e necessária crítica a desconsideração histórica da área analítico-comportamental pelas questões de gênero (Ruiz, 2003). Um breve levantamento da literatura sobre Feminismo e Análise do Comportamento realizada por Couto e Dittrich (2017) revela que, no período entre 1979 e 2016, apenas oito artigos sobre o tema foram publicados em periódicos científicos reconhecidamente analítico-comportamentais. Rita Wolpert subscreve essa tese quando afirma que não só assuntos sobre gênero e feminismo estavam amplamente ausentes dos estudos analítico-comportamentais, como também estão incompletas as análises de contingências sociais de reforçamento que não levam em consideração os contextos de gênero, raça e classe (Wolpert, 2005). O próprio Skinner afirma que o uso da ciência analítico-comportamental poderia ser uma ferramenta importante para a construção de um mundo socialmente mais justo, tal como no livro Beyond Freedom and Dignity (Skinner, 1971), em que ele aborda questões para se repensar o planejamento de culturas mais igualitárias, inclusive no que diz respeito à igualdade entre gêneros.

Segundo a Análise do Comportamento, o comportamento dos indivíduos é função de relações do organismo com eventos e contextos do ambiente imediato e histórico. Quando este ambiente é social, isto é, formado pelas outras pessoas com quem o indivíduo se relaciona, o controle do comportamento se dá pela interação complexa entre os comportamentos de todos os indivíduos envolvidos (Skinner, 1953; 1957). Ao definirmos operacionalmente os fenômenos observados no controle do comportamento humano em termos de comportamentos sociais e socialmente determinados (ou seja, ao descrevermos como os comportamentos individuais selecionam e mantêm práticas culturais e são ao mesmo tempo selecionados e mantidos por essas práticas culturais), o que pretendemos é aproximar a Análise do Comportamento das teorias e movimentos sociais que procuram analisar e intervir em práticas culturais que oprimem grupos de indivíduos. Essa estratégia de intervenção social foi prevista por Holland (1973) quando preconizou como o analista do comportamento poderia atuar para a modificação de práticas sociais injustas e opressoras:

\begin{quote}
    (...) temos que explorar as formas de modificação do comportamento que sejam compatíveis com um sistema igualitário, não materialista e não elitista, mas, ao contrário, construtivo, pelo menos no tocante aos meios para uma inadiável mudança revolucionária do homem (Holland, 1973, p. 280).
\end{quote}

Dado que os analistas do comportamento estão inseridos em uma cultura e se comportam também sob controle das mesmas variáveis que afetam práticas culturais vigentes, não seria surpresa que, na sua prática científica, reproduzam padrões de comportamento culturalmente selecionados. Ao entendermos que a ciência (enquanto conjunto de relatos verbais acerca das relações entre eventos do mundo) é o produto do comportamento do cientista, conforme proposto por Pennypacker e Johnston (1993), uma revisão crítica do conhecimento produzido em uma disciplina deve passar pela análise das variáveis das quais o comportamento que o produziu é função, o que inclui, obrigatoriamente, a análise das práticas culturais que selecionam o comportamento do cientista. Assim, como indicado por Ruiz e Roche (2007), uma atuação ética do analista do comportamento – tanto na sua prática científica quanto na prestação de serviços – deve levar em conta não só a honesta proposição dos valores que guiam o seu fazer científico e a aplicação das tecnologias cientificamente construídas, mas também a compreensão acurada das variáveis que controlam o comportamento ético e as tomadas de decisão sobre o fazer científico e prático. Para isso é necessário buscar compreender o contexto mais amplo dentro do qual o comportamento do cientista está inserido: como as contingências sociais estão organizadas para a manutenção dessas práticas culturais e como elas definem uma dada cultura (Fernandes, Carrara, \& Zilio, 2017).

A partir da ideia de que a ciência é, portanto, um fenômeno também cultural, pretendemos neste capítulo fazer um diálogo entre os conceitos analítico-comportamentais e os conceitos feministas com o objetivo de permitir uma análise mais comportamentalmente contextualizada das contingências sociais diferenciadas para homens e mulheres e das práticas culturais que mantêm as desigualdades e assimetrias de poder entre os gêneros na cultura patriarcal. Para isso, apresentaremos os conceitos que julgamos tornar mais visíveis as contingências culturais de gênero que são selecionadoras desses padrões de invisibilização e silenciamento das questões feministas dentro da Análise do Comportamento, e recomendamos que essas variáveis e contextos culturalmente relevantes sejam parte das análises de fenômenos sociais e das relações entre eventos que a nossa ciência descreve.

