\setcounter{footnote}{0}
\setcounter{figure}{0}
\setcounter{table}{0}
\chapter*{Métodos de investigação sobre cultura do estupro: o que a Análise do Comportamento tem a aprender com as contribuições de outras áreas do conhecimento}\sectionmark{Métodos de investigação sobre cultura do estupro}
\addcontentsline{toc}{chapter}{Capítulo 4}
\addcontentsline{toc}{section}{Métodos de investigação sobre cultura do estupro: o que a \\Análise do Comportamento tem a aprender com as \\contribuições de outras áreas do conhecimento}
\addcontentsline{toc}{subsection}{\textbf{Autora:} \textit{Amanda Oliveira de Morais \& Júlia Castro de \\Carvalho Freitas}}
\begin{flushright}
\begin{small}
    Amanda Oliveira de Morais \& Júlia Castro de Carvalho Freitas\blfootnote{O presente trabalho foi realizado com apoio da Coordenação de Aperfeiçoamento de Pessoal de Nível Superior - Brasil (CAPES) - Código de Financiamento 001.}
\end{small}
\vspace{1cm}
\end{flushright}

A expressão cultura do estupro surgiu durante a segunda onda do feminismo, na década de 1970, e foi inicialmente cunhada para descrever a cultura americana (Smith, 2004). Em uma cultura do estupro, o estupro e outras violências sexuais contra mulheres são prevalentes e consideradas norma ou inevitáveis, e não são desafiadas (Buchwald, Fletcher, \& Roth, 1993/2005). A partir do reconhecimento da alta prevalência de violência sexual, iniciaram-se os questionamentos sobre as culturas que toleram o estupro e outros abusos. Quando o número de agressões sexuais é alto e a taxa de prisões, processos e condenações de agressores é baixa, frequentemente indica-se que há práticas culturais mantendo este tipo de violência. As práticas culturais indicadas envolvem: pareamento entre sexo e violência; culpabilização da vítima após sofrer a agressão; responsabilização da mulher em evitar crimes sexuais; perpetuação de mitos sobre estupro pela comunidade verbal, especialmente pela mídia em propagandas, programas de televisão, filmes e músicas; explicações psicopatológicas ou naturalizantes sobre o comportamento do agressor, negligenciando fatores socioculturais; papéis de gênero atribuídos a cada sexo e privilégios legais, sociais e econômicos aos homens (Freitas \& Morais, no prelo). Estas duas últimas práticas não se relacionam apenas com a denominada cultura do estupro, referem-se a práticas contidas em um cenário mais amplo: o patriarcado\footnote{Saffioti (2004) define o patriarcado como uma forma de organização e de dominação social fundamentada na exploração dos homens sobre as mulheres. Portanto, neste capítulo, contexto patriarcal se refere a dominação masculina presente em toda dinâmica social, incluindo as esferas familiar, trabalhista, midiática, política e os controles sutis de variáveis desconhecidas por homens e mulheres ao comportarem-se privada ou publicamente. 
Para mais discussões recomenda-se a leitura do capítulo 02 deste livro.}. Assim, algumas das práticas apontadas ensejam, além de violência sexual, diversas formas de opressão das mulheres.

Em termos comportamentais, a cultura do estupro pode ser caracterizada por um conjunto de contingências que encorajam e/ou permitem práticas sexuais violentas e por um conjunto de classes de comportamentos sexualmente abusivos, dos mais sutis (como assédio verbal) ao estupro, que ocorrem no contexto patriarcal (Freitas \& Morais, no prelo). Portanto, pesquisas e métodos que investiguem características de classes de comportamentos abusivos, bem como a presença de contingências que promovem e mantêm essas classes de comportamentos, podem ser consideradas pesquisas que estudam o que chamamos de cultura do estupro. Há, ainda, estudos que investigam a violência sexual com enfoque nas variáveis ontogenéticas do comportamento do agressor para explicar a ocorrência desse fenômeno e possibilidades de intervenção. Tais pesquisas não estão necessariamente implicadas com o conceito de cultura do estupro, entretanto, não representam a exclusão das 
variáveis culturais, podendo fornecer pistas sobre possíveis análises mais amplas.

Demonstrada a amplitude do campo de investigação fornecido pelo conceito de cultura do estupro, o presente capítulo tem o objetivo de apresentar e discutir diferentes métodos de estudo sobre o tema. Para tanto, (a) apresentaremos pesquisas realizadas por feministas do cam\-po das ciências sociais e psicologia, (b) apresentaremos pesquisas realizadas por analistas do comportamento que investigam comportamentos que estariam presentes na cultura do estupro; e (c) discutiremos de que forma analistas do comportamento podem aprender com o estado da arte e produzir uma ciência comportamental comprometida com a perspectiva feminista radical sobre a violência sexual.

\section*{Pesquisas sobre cultura do estupro em diversas áreas}

A discussão sobre a alta frequência de violência sexual identificada em diversos países abriu as portas para a investigação das variáveis que afetam o comportamento de agredir sexualmente em diferentes culturas e contextos sociais. Feministas da segunda geração começaram a desmistificar o estupro, antes considerado raro e praticado por um desconhecido, utilizando como método entrevistas com mulheres vítimas de violência sexual. Os relatos de estupros foram apresentados em primeira pessoa pela primeira vez em livros como ``\textit{Rape: The first sourcebook for women}'', produzido pela \textit{New York Radical Feminists} e editado por Noreen Connel e Cassandra Wilson (1974) e ``\textit{Against our will: Men, women, and rape}'' de Susan Brownmiller (1975). O uso da transcrição de relatos verbais vocais de pessoas que vivenciaram crimes, como estupros, permite investigar as características do crime relatado, como quais comportamentos foram emitidos, o local, identificar os autores, averiguar a gravidade da violência perpetrada e indicar os danos causados, quando utilizadas técnicas de entrevistas forenses (Williams et al., 2014). A utilização de entrevistas para o estudo desse fenômeno é ainda justificada conceitualmente por coletar dados de fontes geralmente negligenciadas em uma cultura que avalia o relato de vítimas com desconfiança (Williams, 2002). 

O uso do relato verbal, todavia, remete a outras preocupações metodológicas quando aplicado para apurar as características de um fenômeno ou as variáveis que o afetam. A forma de realizar perguntas em uma entrevista pode afetar as declarações da entrevistada ou entrevistado. Por exemplo, questões fechadas, que limitam as respostas da(o) participante em sim e não, apresentam índices maiores de sugestionabilidade, sendo preferível o uso de questões abertas e escuta livre sobre o fato (Williams et al., 2014). Além disso, quando se trata de estudos que pretendem investigar a prevalência de comportamentos ou variáveis relacionadas a eles, as preocupações sobre validade estatística dos dados devem ser consideradas, como a representatividade da amostra e seleção das informações. Davis (1981/2016) argumenta, por exemplo, que no livro ``\textit{The politics of rape}'', de Diana Russell, dos 22 casos de estupro relatados por mulheres da área da baía de São Francisco, 12 (mais da metade) envolviam mulheres brancas que foram estupradas por homens negros, de origem mexicana ou indígena. Porém, originalmente, foram realizadas 95 entrevistas, nas quais apenas 26\% dos autores de agressão sexual eram de minorias étnicas. O processo de seleção de quais relatos seriam publicados implicou na indução de uma falsa conclusão relacionada aos autores desse tipo de violência, além de produzir consequências políticas, como fortalecer o mito do estuprador negro.

Os estudos que empregam o uso de entrevistas podem basear-se em métodos qualitativos ou quantitativos. O método qualitativo geralmente é apoiado por procedimentos de análise de discurso, enquanto o método quantitativo adota procedimentos de estatística descritiva e inferencial (Baptista \& Campos, 2015). Além disso, as pesquisas podem correlacionar fontes de dados de entrevistas com outras fontes, como documentos, leis e registros oficiais. Por exemplo, ao se questionarem se estadunidenses vivem mesmo em uma cultura do estupro, Buchwald, Fletcher e Roth (2005) apresentaram dados de dois programas administrados pelo Departamento de Justiça dos Estados Unidos que empregam dois métodos de coleta de dados diferentes entre si, porém, para a análise dos dados coletados, ambos utilizam procedimentos estatísticos em maior ou menor grau. São eles o \textit{Uniform Crime Report} (URC) e o \textit{National Crime Victimization Survey} (NCVS). O primeiro programa coleta dados de denúncias realizadas, cobrindo 96\% da população, enquanto o segundo realiza pesquisas estatísticas com amostras representativas (por ano são entrevistadas 42.000 famílias, aproximadamente 76.000 pessoas) com o objetivo de estudar a vitimização da população como um todo. Comparando os dados desses programas, as autoras concluem que apenas 34\% dos estupros e 26\% das agressões sexuais são reportadas à polícia. As razões citadas por mulheres na pesquisa do NCVS para não reportarem tais crimes incluem: manter a agressão como um assunto pessoal, medo de represálias e proteção do agressor. Quanto maior a relação de proximidade entre as mulheres que foram vítimas e os agressores, maior a probabilidade de as violações não serem reportadas à polícia. 

Diversos relatórios produzidos por órgãos governamentais, ONGs e empresas são baseados em pesquisas que utilizam o método \textit{Survey} (questionários). Um exemplo é o \textit{National Violence Against Women Survey} (NVAWS), realizado de 1995 a 1996 com o objetivo de levantar informações sobre a prevalência de vítimas de estupro por gênero, idade e raça/etnia; características das vítimas de estupro, dos autores de estupro e sobre a ocorrência do estupro; a relação entre a vitimização de crianças, adolescentes em comparação com a de adultos; consequências físicas, sociais e psicológicas da vitimização do estupro; e satisfação com o sistema de justiça (Tjaden \& Thoennes, 2006). No Brasil, um programa que emprega o uso de questionários é o Sistema de Indicadores de Percepção Social (SIPS) administrado pelo Instituto de Pesquisa Econômica Aplicada (IPEA). O SIPS realiza estudos domiciliares e presenciais em 3.775 domicílios, em 212 municípios, abrangendo todas as unidades da federação, utilizando o método de amostragem probabilística, garantindo uma margem de erro de 5\% a um nível de significância de 95\% para o país como um todo e para as cinco grandes regiões. No Brasil, entrevistas em pesquisas especializadas no tema de violência sexual no âmbito nacional ainda não foram realizadas, porém, em 2013, um questionário sobre vitimização dessa violência foi integrado ao SIPS. Estimou-se que 527.000 tentativas de estupro ou estupros acontecem no país por ano (IPEA, 2014). Essa estimativa deve ser observada com cautela, pois, como foi destacado, é possível que o método realizado pelo SIPS não seja o mais adequado para se estimar a prevalência do estupro, sendo alertado - em nota técnica pelo IPEA (2014) - que este número deve servir apenas como uma estimativa para o limite inferior de prevalência do fenômeno no país. Comparando esse dado com os registros do Anuário Brasileiro de Segurança Pública (FBSP, 2013), calcula-se que, em 2012, apenas 9,6\% (50.617 casos) das violações foram reportadas à polícia, aproximadamente. O registro de agressões sexuais, os estudos que estimam a prevalência da violência sexual e as correlações entre esses dois métodos de produção de dados sobre o fenômeno, evidenciam o baixo índice de denúncias. Sendo a denúncia um indicativo social de reprovação de determinado comportamento, a pequena frequência de queixas formais destaca que devem existir variáveis que afetam o comportamento da vítima de denunciar ou não, bem como aponta que os comportamentos dos agressores raramente são punidos formalmente.

Para uma melhor tipologia dos crimes sexuais, desde 2011 o Sistema de Informações de Agravo de Notificação do Ministério da Saúde (Sinan) tem apontado as principais características de vítimas e autores da violência no Brasil. A base de dados do Sinan é alimentada por notificações e investigações de profissionais da área da saúde, nos setores público e privado, sobre casos de doenças e agravos que constam na lista nacional de doenças de notificação compulsória, o que inclui violência sexual. Dos 12.087 casos registrados pelo Sinan em 2011, as vítimas eram principalmente do sexo feminino em todas as faixas etárias (88,5\%) e os autores das agressões eram majoritariamente do sexo masculino - acima de 90\% em todas as faixas etárias (Cerqueira, Coelho, \& Ferreira, 2017). Utilizar métodos que identifiquem as principais características dos autores de agressões, bem como quais pessoas geralmente são atacadas, auxilia na descrição do fenômeno e promove a investigação das variáveis que afetam a incidência das agressões. Os dados de pesquisas tipológicas sobre o estupro e abuso sexual, como aqui exemplificada com os dados do Sinan, têm demonstrado que a violência sexual se caracteriza como uma violência perpetrada por homens.

As diferenças de conceituação sobre violências sexuais, a forma como são registrados os dados, e os fatores sociais que afetam o comportamento de denunciar ou relatar a violência sexual afetam as pesquisas estatísticas, mesmo nos casos de registros oficiais. Por exemplo, em diversas legislações, o uso da força física é prerrogativa para que haja violência sexual, enquanto em outras, como na Lei Maria da Penha, o uso de coação e manipulação ampliam o que pode ser considerado este tipo de violência. Portanto, ao comparar a prevalência desses crimes em diferentes países é preciso levar em consideração qual critério de inclusão de agressões foi utilizado nas pesquisas. Outro fator relevante é a forma de registro de ocorrências. Quando uma vítima denuncia uma situação de violência sexual, pode-se contabilizar todos os episódios de agressões sexuais ou apenas um caso a partir da emissão de um único boletim de ocorrência. Exemplificando, a estimativa de vítimas de violência sexual poderia ser diferente da apontada nos estudos descritos anteriormente se a forma como as perguntas são feitas e a especificação das ocorrências fossem consideradas. Um dos itens de uma pesquisa realizada com 2.300 mulheres de 14 a 24 anos, das classes econômicas C, D e E demonstrou que 47\% das entrevistadas afirmaram já terem sido forçadas a ter relações sexuais com o parceiro (Énois Inteligência Jovem, Instituto Vladimir Herzog \& Instituto Patrícia Galvão, 2015). Se fosse possível detalhar aproximadamente quantas vezes cada uma dessas mulheres já foi forçada a ter relações com o parceiro, teríamos uma prevalência mais apurada da ocorrência de violência sexual nas relações íntimas de jovens brasileiras dessas classes econômicas. Os procedimentos descritos até aqui são amplamente utilizados em pesquisas nas áreas das ciências sociais e ainda para o reconhecimento de problemas sociais e úteis para a formulação de ações do Estado. 

Um estudo transcultural realizado em 1981, por Sanday, exemplifica outra forma de estudar o estupro, para além dos dados epidemiológicos e da tipologia do crime: olhando para o contexto sociocultural do estupro. Uma amostra transcultural padrão publicada por Murdock e White em 1969, consistia em descrever diversas características de 186 sociedades em um período de tempo que varia de 1750 a.C. até o final dos anos 1960. As sociedades incluídas na amostra padrão são distribuídas de forma relativamente igual entre as seis principais regiões do mundo. Sanday (1981) selecionou 156 sociedades da amostra que continham dados sobre estupro, segundo suas definições. O objetivo do estudo foi fornecer um perfil descritivo de sociedades ``propensas ao estupro'' e ``sem estupro'', e apresentar uma análise das atitudes, motivações e fatores socioculturais relacionados à incidência de estupro. Para isso, 21 variáveis foram codificadas com base em códigos publicados na revista Ethnology; materiais de bibliotecas; e os arquivos da área de relações humanas. A autora discute a variação da incidência de estupro de acordo com a cultura, verificando que as altas taxas de estupro ocorrem em configurações culturais distintas das quais há baixas taxas. Os dados sugerem que o estupro faz parte de uma configuração cultural que inclui violência interpessoal, dominância masculina (poder e a autoridade femininos são menores nas sociedades propensas ao estupro e mulheres quase não participam da tomada de decisão pública enquanto os homens expressam desprezo pelas mulheres que o fazem), e separação sexual (presença de estruturas ou locais onde as pessoas se reúnem em grupos do mesmo sexo).

Outra forma de investigação importante refere-se a verificar se modelos presentes na cultura afetam o comportamento de agredir sexualmente a partir de estudos que avaliam a correlação entre pornografia e violência de gênero. Bridges, Wosnitzer, Scharrer, Sun e Liberman (2010) avaliaram e classificaram cenas de vídeos pornográficos populares, para indicar a exibição (ou não) de violência. Foram avaliadas cenas selecionadas a partir de listas dos vídeos mais alugados, publicada mensalmente pela \textit{Adult Video News} (AVN). Foram selecionados os 30 vídeos que foram classificados como melhores, em cada lista, de dezembro de 2004 a junho de 2005. Após a exclusão de duplicações, a amostra consistiu em 275 títulos, sendo 304 cenas extraídas de 50 vídeos selecionados aleatoriamente. Para a análise das cenas foram operacionalizadas as variáveis referentes à cena (personagens, atos sexuais, posição de ejaculação, presença de agressão física e verbal e presença de comportamentos positivos, como abraçar, beijar e elogiar). Quando houve agressão, os atos foram descritos em: tipo de ato agressivo, perpetradores e alvos de agressão, e resposta do alvo à agressão. Das cenas analisadas, 88,2\% continham agressão física, principalmente palmadas, amordaçamento e bofetadas, enquanto 48,7\% das cenas continham agressão verbal, principalmente xingamentos. Os autores da agressão eram geralmente do sexo masculino (70,3\%), enquanto os alvos da agressão eram, frequentemente, do sexo feminino (94,4\%). As respostas mais frequentes à agressão, emitidas pelos alvos, foram demonstração de prazer ou respostas classificadas como neutra. Tal resultado implica na demonstração de que a pornografia inclui violência contra atrizes. O impacto desses modelos no comportamento de quem assiste foi investigado pelo estudo relatado a seguir.

A relação entre visualização regular de pornografia \textit{online}, coerção e abuso sexual, e o envio e recebimento de imagens sexuais (conhecidas como \textit{sexting}) foi avaliada por Stanley et al. (2016). A pesquisa foi realizada em 45 escolas de cinco países europeus, com o uso de questionários que eram respondidos individualmente. As categorias de análise (abuso e coerção sexual, \textit{sexting}, igualdade de gênero e visualização de pornografia) foram abordadas em grupos de questões detalhadas do estudo (Stanley et al. 2016). A pesquisa foi complementada por entrevistas qualitativas. Para a análise dos dados foi utilizado o programa \textit{Statistical Package for Social Science} (SPSS), executando estatísticas descritivas, incluindo tabulações cruzadas para cada país. Foram realizados testes do qui-quadrado para identificar os efeitos de gênero em cada país. Também se utilizou análise multivariada para explorar ainda mais a relação entre o uso da coerção sexual por parte dos meninos e sua relação com a visualização regular de pornografia. A análise de dados limitou-se ao uso da coerção sexual pelos meninos, pois neste estudo, menos de 8\% de meninas relataram usar comportamento sexualmente coercitivo. Em suma, as taxas de visualização regular de pornografia \textit{online} foram maiores entre os meninos, e foram significativamente associadas à perpetração de coerção, abuso sexual e \textit{sexting}. Além disso, os meninos que assistiram regularmente a pornografia \textit{online} eram significativamente mais propensos a apresentar atitudes negativas em relação à igualdade de gênero. Os dados apresentados são relevantes, pois se trata da primeira pesquisa comparativa em larga escala sobre o assunto, porém seus resultados não podem ser generalizáveis para todos os jovens dos cinco países, pois não foi possível construir uma amostra aleatória. 

Outro método utilizado nas ciências sociais refere-se a estudos de casos e pode ser classificado como exploratório, descritivo ou analítico. Este método pode elucidar questões sobre fenômenos que são difíceis de estudar em ambiente laboratorial. Um exemplo desse tipo de estudo é a etnografia do ciberespaço após uma notícia de violência sexual, realizada por Rost e Vieira (2015). As autoras realizaram uma análise das narrativas dos participantes de um debate online sobre o episódio no qual o diretor teatral Gerald Thomas enfiou a mão por debaixo do vestido da repórter Nicole Bahls sem consentimento e publicamente. A partir da análise de 172 comentários publicados em um texto sobre o ocorrido, as pesquisadoras verificaram quais convenções de gênero os participantes estavam defendendo quando discutiam a violência sexual e como elas ganhavam sentido, dando ênfase à dimensão simbólica da disputa sobre os direitos das mulheres (Rost \& Vieira, 2015). As pesquisadoras concluem que ``a noção de violência sexual é atravessada por moralidades relativas a convenções de gênero e sexualidade que interferem na percepção dos direitos individuais das mulheres.'' (Rost \& Vieira, 2015 p. 261). Em termos comportamentais, precisaríamos explicar o que são moralidades relativas a convenções de gênero e como elas interferem na percepção dos direitos individuais das mulheres. Uma forma de investigar isso, na Análise do Comportamento, seria utilizar métodos experimentais que abordam os processos envolvidos no comportamento simbólico, como será abordado mais adiante.

\section*{Pesquisas em Psicologia}

No campo da Psicologia, diversos pesquisadores também vêm se dedicando a estudar questões presentes em uma cultura do estupro. As questões vão desde estereótipos de gênero, que contribuem para a manutenção de acesso desigual de homens e mulheres a reforçadores, passando por variáveis envolvidas na naturalização da violência sexual (e.g., Kahn, Jackson, Kully, Badger, \& Halvorsen, 2003), até ``perfis'' de agressores e busca por fatores que mantêm a violência sexual (e.g., Nunes, Hermann, \& Ratcliffe, 2013). A produção acadêmica em Psicologia emprega métodos experimentais ou quase-experimentais, além de realizar pesquisas do tipo \textit{Survey}, como em outros campos citados neste capítulo. Grande parte dessas pesquisas foi conduzida por psicólogas de abordagem cognitivista, como será apresentado a seguir.

Uma questão que tem sido bastante estudada é o que as pessoas entendem por estupro. Para tanto, Ryan (1988) utilizou o método de pedir a estudantes universitários que descrevessem um estupro típico e uma situação de sedução típica, dando o máximo de detalhes possível sobre o que teria acontecido, as circunstâncias e pessoas envolvidas. Na abordagem cognitivista, a descrição feita por cada participante é entendida como um \textit{script}, ou seja, um protótipo de como estas situações costumam acontecer. Para análise, foram identificados os elementos presentes nos \textit{scripts} e cada um destes foi codificado em uma escala de frequência. Os resultados apontaram que a maioria dos participantes via estupro como um ato violento, acontecido à noite, em local de acesso público, em que um homem com problemas mentais ou sociais e desconhecido por uma mulher a ataca e usa força física para penetrá-la, enquanto a vítima apresenta resistência.

Outros métodos têm sido utilizados para identificar quais variáveis influenciam a nomeação de um ato como estupro. Um deles se dá através da comparação entre as respostas do participante a uma lista de comportamentos que ele pode ter praticado (ou, no de caso de mulheres, aos quais ela pode ter sido submetida) e sua resposta sobre ter praticado (ou sofrido) estupro. Na lista, constam comportamentos como ``ter fisicamente forçado uma mulher a ter relações sexuais'' e ``ter feito sexo com uma mulher que estava inconsciente'', os quais são considerados estupro pela lei. Então, os participantes que afirmam ter praticado (ou sofrido) pelo menos um desses comportamentos, mas dizem que nunca estupraram (ou foram estupradas), são agrupados separadamente daqueles que dizem expressamente ter praticado (ou sofrido) estupro. Em seguida, os grupos são comparados em termos de características dos participantes e/ou da situação que vivenciaram. Este método foi usado tanto com participantes homens (e.g., Edwards, Bradshaw, \& Hinsz, 2014\footnote{Neste estudo, o relato coletado não foi sobre o comportamento passado, mas sobre a intenção de comportamento futuro.}) como com participantes mulheres (e.g., Koss, Dinero, Seibel, \& Cox, 1988; Kahn \& Mathie, 1994; Kahn et al., 2003; Layman, Gidycz \& Lynn, 1996; Peterson \& Muehlenhard, 2004). Alguns resultados encontrados nesses estudos apontam que é menos comum que a vítima nomeie o que sofreu como estupro caso tenha uma relação com o agressor, não tenha lutado contra este ou considere ter tido um comportamento sexualmente provocativo.

Um terceiro método usado para investigar o que seria considerado estupro é através da apresentação de histórias de estupro (chamadas de vinhetas) aos participantes, como no estudo de Sasson e Paul (2014). Estas autoras escreveram uma história fictícia, com variações, que narrava um estupro e pediram a trabalhadores de uma empresa que escolhessem, dentre as opções ``agressão sexual'', ``abuso sexual'', ``estupro'' ou ``nenhuma dessas'', a expressão que melhor descrevia a situação da história que cada um leu. Em seguida, as respostas dos participantes foram analisadas em busca de correlações entre rotular a vinheta lida como estupro e algumas variáveis, como gênero da(o) participante e ter sido vítima de violência sexual. As autoras encontraram que, independentemente das variações na história, não nomear a situação como ``estupro'' estava positivamente correlacionado a altos escores em um instrumento que mede o grau de concordância com mitos sobre violência sexual, a Escala de mitos modernos sobre violência sexual (AMMSA, em inglês), desenvolvida por Gerger, Kley, Bohner e Siebler (2007), que será aborda mais adiante.

Vinhetas desse tipo também foram usadas em uma série de estudos que pretendiam verificar quais características presentes nas histórias poderiam implicar que os participantes culpabilizassem mais as vítimas (para uma revisão, ver Grubb \& Harrower, 2008; para uma meta-análise, ver Whatley, 1996). O método comumente utilizado nesses estudos foi de apresentar, para os participantes, vinhetas que narravam situações de estupro e pedir que estes dissessem a quem deveria ser atribuída a responsabilidade por aquela situação. Para verificar se havia diferença na culpabilização de acordo com as variáveis escolhidas, eram apresentadas, a dois ou mais grupos de participantes, vinhetas similares que variavam apenas em termos de características da vítima, do agressor ou das circunstâncias (e.g., se a vítima usava roupas que mostravam ou que cobriam mais o seu corpo). Em geral, um grupo de participantes (agrupados de maneira aleatória) lia e respondia a apenas uma vinheta. Uma possibilidade de análise de dados nesses estudos (por exemplo, quando era utilizada uma escala do tipo \textit{Likert}) é através da comparação das médias das respostas entre os grupos que responderam a um tipo ou outro de vinheta. Se a média da responsabilidade atribuída em uma vinheta fosse significativamente maior do que a média em outra, entender-se-ia que as vítimas teriam sido mais culpabilizadas quando a característica analisada estivesse presente. Tomados em conjunto, esses estudos apontam que mulheres que (a) estavam usando roupas que mostram mais seus corpos, (b) apresentavam comportamento moral ``menos respeitável'' (como ingerir bebidas alcoólicas ou trabalhar em boates) e/ou (c) conheciam o seu agressor, tendem a ser mais culpabilizadas pelo estupro que sofreram em comparação com as que não apresentavam essas características nas vinhetas.

Instrumentos como a AMMSA têm sido usados desde 1980 para acessar crenças que negam, justificam ou minimizam a violência sexual (e.g., Burt, 1980; Payne, Lonsway, \& Fitzgerald, 1999; Gerger et al., 2007). Esses autores defendem que essas crenças contribuem para ocorrência de estupros. O primeiro instrumento desse tipo foi a Escala de aceitação de mitos sobre estupro (Burt, 1980) que conta com itens como ``na maioria dos estupros, a vítima é promíscua ou tem uma reputação ruim'' e ``uma mulher que pega carona merece ser estuprada''. Cada um dos 19 itens deve ser respondido em uma escala de sete pontos que vai de ``concordo totalmente'' a ``discordo totalmente''. Devido às crescentes manifestações feministas na sociedade, concordar com tais mitos têm se tornado cada vez mais reprovável pelas pessoas, de forma que o instrumento desenvolvido por Burt (assim como outros mais antigos) pode ser atualmente menos sensível a acessar atitudes sobre violência sexual (Gerger, et al., 2007). Foi pensando nisso que Gerger e colaboradores (2007) desenvolveram a AMMSA. Este instrumento conta com frases mais sutis a respeito desse tipo de violência, mas que também contribuem para a sua manutenção. Dessa forma, os autores encontraram que tanto em uma população de estudantes americanos, como de estudantes alemães, as respostas a esta escala obedecem à curva normal da estatística. Em outras palavras, esse instrumento se mostrou mais sensível a identificar variações no grau de aceitação dos mitos entre indivíduos de um grupo em comparação com as outras escalas aqui citadas. Exemplos de itens da AMMSA são: ``a discussão sobre assédio sexual no trabalho resulta principalmente da má interpretação de comportamentos inofensivos como sendo assédio'' e ``em um encontro, a expectativa comum é que a mulher ‘puxe o freio’ e o homem ‘pise no acelerador’'' (Gerger, et al., 2007, tradução da autora).

Embora a AMMSA tenha se mostrado um bom instrumento para acessar a concordância com mitos sobre estupro, é importante notar que este é um instrumento de autorrelato e, como tal, está sujeito ao viés de desejabilidade social. Com o intuito de evitar esse tipo de viés, foram criados instrumentos que acessam atitudes chamadas pela literatura cognitivista de ``implícitas'' (e.g., Nunes et al., 2013)\footnote{Para uma discussão mais aprofundada sobre atitudes implícitas neste livro, consulte o capítulo 05}. Parafraseando Nunes et al. (2013), atitudes implícitas são associações automáticas com caráter avaliativo a respeito de um dado objeto. Esses autores usaram um instrumento denominado \textit{Implicit Association Test} (IAT) para acessar atitudes diante da apresentação da palavra-estímulo ``estupro''. Além disso, foi aplicado um instrumento no qual os participantes, todos homens, deveriam responder se e quantas vezes tinham praticado determinados comportamentos que se caracterizam como violência sexual contra uma mulher (e.g., pressioná-la ou forçá-la fisicamente para fazer sexo oral). Com base nessas respostas, os participantes foram separados em grupos de agressores e não agressores. Cada participante foi solicitado ainda a pensar em três consequências que poderiam acontecer para ele caso forçasse uma mulher a fazer sexo com ele. O participante deveria avaliar cada uma das consequências numa escala que ia de -3 (muito negativo) a +3 (muito positivo) e, então, era feito um cálculo final somando as três consequências, numa tentativa de determinar se o participante considerava que este comportamento ``valia a pena''.

Para verificar a automaticidade das associações entre a palavra-estímulo e coisas positivas e entre aquela e coisas negativas, Nunes e colaboradores (2013) mensuraram a diferença da latência da resposta dos participantes diante dos diferentes blocos de apresentações. No bloco tipo 1, a palavra ``estupro'' se encontrava no mesmo lado da tela que a palavra ``mau'', enquanto ``não-estupro'' aparecia juntamente com a palavra ``bom'', já no bloco tipo 2, o contrário acontecia. No centro da tela, apareciam outras palavras comumente vistas como positivas ou negativas. Nos blocos tipo 1, os participantes deveriam pressionar uma tecla à direita se a palavra combinasse com ``estupro'' ou ``mau'', e também deveriam pressionar uma tecla à esquerda se a palavra combinasse com ``não-estupro'' ou ``bom''. A ideia por trás disso é que a latência dessa resposta seria menor para pessoas que associam estupro com algo ruim em comparação com pessoas que não associam. Por outro lado, no bloco tipo 2, haveria um conflito para essas pessoas, uma vez que tanto ``estupro'' como ``bom'' estão do mesmo lado da tela, então ao ver uma palavra no centro que deve ser associada com um dos lados, a latência da resposta deve ser maior do que no bloco tipo 1. De fato, este foi o resultado encontrado para o grupo de homens não agressores. No entanto, ao contrário do que os autores esperavam, o resultado foi similar para os agressores, ou seja, eles também responderam mais rápido no bloco tipo 1 do que no bloco tipo 2. Porém, para o grupo de agressores a diferença na rapidez foi menor do que para o grupo não agressor. Em outras palavras, para homens não agressores há uma associação mais forte entre ``estupro'' e ``mau'' do que para agressores.

Os resultados desse IAT podem ser interpretados a partir do que foi constatado em outros estudos aqui citados. A palavra estupro frequentemente é usada para descrever um tipo específico de penetração forçada (em geral, com o agressor sendo um desconhecido, usando força, etc.), o que mantém uma visão estereotipada sobre estupro, de forma a ignorar uma série de características que podem se fazer presentes nestes crimes. Dessa forma, os participantes do IAT em Nunes et al. (2013) parecem, de fato, associar a palavra estupro a coisas negativas, mas os participantes agressores possivelmente não relacionam os atos que cometeram com a palavra estupro. Portanto, a forma como são rotulados os atos sexuais violentos parece contribuir para a naturalização destes, o que, juntamente com a culpabilização da vítima, são contingências encorajadoras e/ou permissivas com comportamentos abusivos, ou seja, estão presentes em uma cultura do estupro.

Nesta seção, vimos que a psicologia tem usado métodos diversos para investigar a questão da violência sexual e que os resultados encontrados contribuem para a concepção de que vivemos em uma cultura do estupro. Como citamos anteriormente, na perspectiva cognitivista, crenças e atitudes são, em grande medida, a explicação do porquê um homem se comporta de maneira a agredir uma mulher sexualmente. Para a Análise do Comportamento, no entanto, o comportamento de agredir (bem como as crenças e atitudes) é determinado pelas contingências de reforçamento e punição as quais os indivíduos estão submetidos. Além disso, os estudos aqui citados focam na topografia, e não na função dos comportamentos investigados. Assim, entendemos que, para estudos analítico-comportamentais a respeito da cultura do estupro, faz-se necessária a aplicação e/ou o desenvolvimento de métodos compatíveis com esta perspectiva, ou ainda uma proposta diferente de interpretação dos dados encontrados a partir de métodos utilizados por outros campos de conhecimento.

\section*{Pesquisas em análise do comportamento}

A análise do comportamento carece, ainda, de uma discussão sobre os fatores culturais envolvidos com a violência sexual. De fato, em busca sobre esse tema em revistas especializadas (\textit{Acta Comportamentalia, Behavior and Social Issues, European Journal of Behavior Analysis} (EJOBA), Perspectivas em Análise do Comportamento, Revista Brasileira de Análise do Comportamento (REBAC), Revista Brasileira de Terapia Comportamental e Cognitiva (RBTCC), \textit{The Behavior Analyst, The Behavior Analyst Today e Psychological Record}) não foi encontrado nenhum artigo A que versasse especificamente sobre o tema da violência sexual relacionada a aspectos culturais, de gênero ou a fatores de risco e proteção. Apesar disso, um estudo encontrado versava sobre o tratamento psicoterapêutico de mulheres vítimas de violência, com enfoque em estratégias de atendimento quando estas apresentam estresse pós-traumático (Garcia \& Bolsoni-Silva, 2015). Nesta busca, realizada para os fins deste capítulo, foram utilizadas as palavras-chave ``\textit{rape}'', ``estupro'', ``\textit{sexual violence}'', ``violência sexual'', ``\textit{sexual abuse}'', ``abuso sexual'', ``\textit{sexual aggression}'', ``agressão sexual'', ``\textit{sexual harassment}'', ``assédio sexual'', e ``\textit{sexual assault}''. No entanto, alguns dos artigos encontrados abordam brevemente a questão, embora esta não seja o assunto principal (e.g., Roche \& Barnes, 1998; Roche \& Barnes-Holmes, 2002).

Além de ser escassa a discussão sobre cultura e violência sexual na análise do comportamento, até o presente momento, não há estudos publicados que discutam cultura do estupro. O conceito de cultura do estupro auxiliou a ampliação do campo de investigação sobre violência sexual ao (a) descrever um espectro de comportamentos abusivos dos mais sutis aos mais graves, incluindo violências negligenciadas pela investigação científica até o início da década de 70, como o estupro no casamento; e (b) investigar as variáveis culturais que afetam a prevalência dessas violências. Além disso, como apontado na revisão feita por Couto e Dittrich (2017), há poucos estudos em análise do comportamento que versem sobre qualquer questão feminista. No entanto, algumas pesquisas sobre estereótipos de gênero (e.g., Roche \& Barnes, 1996; Drake et al., 2010; Cartwright, Roche, Gogarty, O'Reilly, \& Stewart, 2016) podem contribuir para uma discussão feminista e analítico-comportamental sobre comportamentos e contingências presentes em uma cultura do estupro.

Um experimento desenvolvido por Roche e Barnes (1996) investigou o responder dos participantes sobre categorias sexuais no que diz respeito à dominância e submissão. Este estudo se embasa na Teoria das Molduras Relacionais (RFT, em inglês), a qual propõe que o comportamento verbal pode ser entendido como responder relacional arbitrariamente aplicável (RRAA) (ver Hayes, Barnes-Holmes, \& Roche, 2001). De maneira resumida, isso quer dizer que aprendemos relações arbitrárias entre coisas no mundo (o que inclui objetos, palavras e nós mesmos) e, mais do que isso, aprendemos um operante verbal de relacionar coisas de maneira arbitrária (ou seja, muitas relações não precisam ser diretamente ensinadas, pois temos a capacidade de derivar uma relação a partir de outras que aprendemos anteriormente) e diferentes tipos de relações (como de similaridade, oposição, comparação). No estudo de 1996, Roche e Barnes usaram categorias sexuais para demonstrar em laboratório como é possível aprender o significado de uma dica contextual (a qual determina o tipo de relação entre dois estímulos). Inicialmente, os participantes foram expostos a um pré-treino para aprender que determinados símbolos (dicas contextuais) tinham os significados de ``igual'', ``oposto'' ou ``diferente'', através de um treino de relações entre estímulos diante desses símbolos. Em seguida, em uma série de tentativas, uma das palavras-estímulo ``vagina'', ``pênis'' e ``amnésia'' aparecia no centro de uma tela como estímulos-modelo, juntamente com um dos símbolos aprendidos no pré-treino, este na parte de cima da tela, e as palavras ``submeter-se'', ``dominar'' e ``esquecer'', que eram os estímulos de comparação. Em cada tentativa, os participantes eram solicitados a escolher uma das palavras de comparação, que dependia da dica contextual, e nunca recebiam feedback pela escolha. Foi verificado que, diante da dica contextual de igualdade, todos os participantes relacionavam a palavra ``vagina'' com a palavra ``submeter-se'' e a palavra ``pênis'' com a palavra ``dominar'', bem como, diante da dica contextual de oposição, relacionavam a palavra ``vagina'' com ``dominar'' e a palavra ``pênis'' com ``submeter-se'', em pelo menos nove de 10 tentativas. Isso quer dizer que os participantes de alguma forma aprenderam na sua história de vida que o órgão sexual das mulheres teria um papel de submissão, enquanto o dos homens teria um papel de dominação. Essa aprendizagem provavelmente foi verbal e indireta (no sentido de que o participante não teria ouvido especificamente que vagina é igual submeter-se).

Em uma publicação posterior (Barnes \& Roche, 1997), esses autores retomaram os resultados encontrados em 1996 para levantar uma breve discussão sobre a influência que o aprendizado cultural de relações verbais entre estímulos pode ter sobre o comportamento de estuprar. Nesta perspectiva, o estupro adquire função reforçadora para homens devido ao ensino verbal de que mulheres seriam submissas e deveriam ser controladas, e de que os homens seriam o oposto disso (dentre outras relações verbais que os membros da cultura produzem e reproduzem). Segundo esses autores, isso explicaria a dificuldade em eliminar o comportamento de estuprar usando técnicas tradicionais na terapia comportamental (e.g., extinção), já que há uma ampla aprendizagem verbal que vai além do pareamento direto entre sexo e violência. Embora essas sejam considerações importantes, esse artigo trata de sexualidade humana em geral, de forma a não enfocar na questão da violência sexual. Em dois artigos posteriores (Roche \& Barnes, 1998; Roche \& Barnes-Holmes, 2002), esses autores retomaram essas considerações, ainda mantendo o mesmo enfoque.

O estudo de Roche e Barnes (1996) não tinha o objetivo de verificar, muito menos de discutir, a existência de papéis sexuais estereotipados. Do contrário, relações estereotipadas foram usadas para verificar a existência de processos descritos pela RFT (ou seja, a partir da pressuposição de que há um papel sexual atribuído às mulheres oposto ao atribuído aos homens, os autores usaram as relações entre os sexos e respectivos papéis para verificar em condições de laboratório a existência do responder relacional a partir de relações de coordenação, oposição e diferença). Isso tem sido feito através de outros procedimentos utilizados pela análise do comportamento, dentre eles, o \textit{Implicit Relational Assessment Procedure} (IRAP) (Barnes-Holmes et al., 2006). Este procedimento é similar ao IAT, porém foi desenvolvido por analistas do comportamento e investiga o RRAA. Segundo Barnes-Holmes, Barnes-Holmes, Luciano e McEnteggart (2017), respostas relacionais se encontram num \textit{continuum} entre respostas relacionais breves e imediatas (RRBI) e respostas relacionais elaboradas e estendidas (RREE), a partir do grau de derivação, coerência e complexidade da relação. Nesse sentido, uma RRBI tende a acontecer de maneira muito rápida (são as chamadas atitudes implícitas pela literatura cognitivista), enquanto uma RREE tende a acontecer quando o sujeito tem um certo tempo para pensar e elaborar sua resposta. O IRAP tem sido usado como um procedimento para identificar, através de respostas relacionais, vieses no que tange a estímulos socialmente relevantes, como é o caso de estereótipos de gênero. Para tanto, os participantes são instruídos e solicitados a responder, em determinados blocos, que uma relação entre dois estímulos (por exemplo, ``homem'' e ``ciências exatas'') é verdadeira e, em blocos alternados, que esta relação é falsa. Além disso, devem responder o mais rápido possível, tendo um tempo máximo para cada tentativa (em geral, dois segundos), o que garantiria que a resposta dada seja uma RRBI e não uma RREE. Assim como no IAT, a ideia é evitar respostas de acordo com a desejabilidade social ou outros vieses possíveis em instrumentos de autorrelato. Os resultados são analisados através da comparação da latência de resposta dos participantes entre os dois tipos de blocos para cada par de estímulos. Nos blocos onde há relações consistentes com a história de vida do participante, a latência da resposta tende a ser menor do que nos blocos onde há relações inconsistentes. Uma vantagem do IRAP em relação ao IAT é permitir uma análise do resultado de cada par de estímulos em separado.

Um levantamento feito por Freitas (2017) encontrou sete estudos que utilizaram estímulos relacionados a gênero em IRAPs. Dentre esses, destacamos o estudo de Hussey et al. (2016), que versou sobre a objetificação e desumanização das mulheres, uma questão de grande importância na cultura do estupro. No entanto, esses autores estavam mais interessados em usar os resultados para discutir características do IRAP do que para discutir a forma como as mulheres são tratadas. Um resumo dos resultados deste e dos outros estudos pode ser encontrado no capítulo 05 deste livro, assim como uma discussão sobre o uso do \textit{Function Aquisition Speed Test} (FAST) para investigações similares. O FAST (Cartwright, Roche, Gogarty, O'Reilly, \& Stewart, 2016) é um procedimento parecido com o IRAP, porém mede-se o tempo que os participantes levam para aprender a responder em cada bloco. Dessa forma, os resultados são analisados através da comparação das curvas de aprendizagem dos diferentes blocos.

Dentre os estudos publicados em análise do comportamento, aque\-le que mais tenta se aproximar de uma concepção cultural sobre violência sexual é o de Hertzog, Wright e Beat (2008). Foi investigada uma série de variáveis presentes em empresas, visando identificar possíveis fatores que estariam envolvidos na emissão ou não de comportamentos de assédio sexual em contexto de trabalho. Dentre as variáveis, constavam a presença de workshops de treinamento, o tamanho da organização e a proporção de mulheres empregadas. Os dados foram extraídos a partir de entrevistas com funcionários de diversas empresas e foram analisados estatisticamente os dados de 303 delas, de forma a apontar correlações entre a presença de determinada variável organizacional, a presença ou não de assédio, e o registro de denúncia sobre assédio. Os resultados encontrados não foram conclusivos a respeito das correlações entre essas variáveis e a presença ou não de assédio. Os autores apontaram, principalmente, que a presença de políticas formais antiassédio não foi suficiente para evitar esse tipo de violência. Sugerem, portanto, intervenções comportamentais. Para tanto, seria necessário intervir sobre comportamentos, por exemplo, a partir de regras e mudanças nos sistemas de recompensa.

Um último estudo que merece destaque nesta seção é o de Guerin e Ortolan (2017). Embora este estudo não trate explicitamente sobre a violência sexual, aborda a violência entre parceiros e, conforme apresentamos mais cedo, dados demonstram que a violência sexual acontece nos relacionamentos com frequência. Ainda, é nesse contexto que as mulheres menos denunciam o ocorrido. Guerin e Ortolan (2017) realizaram análises sobre a violência doméstica que são úteis para a compreensão das estratégias de controle dos homens em relação às companheiras, além de apresentar um modelo de estudo analítico-comportamental que inclui contextos sociais mais amplos nas análises individualistas. Guerin e Ortolan utilizaram materiais já publicados sobre violência contra mulher para identificar algumas estratégias gerais de controle e posteriormente apresentaram possíveis contingências das quais essas estratégias são produto. Essa análise baseou-se no método desenvolvido por Guerin (2016). Foram encontrados diversos comportamentos que sugerem cinco padrões funcionais comuns na violência doméstica: ações físicas diretas e ameaças à mulher; manipular o contexto para controlar o comportamento da mulher; estratégias para manter segredos dentro dos relacionamentos; estratégias para monitorar ou descobrir sobre a mulher, suas atividades e contatos sociais; e construções verbais para ameaçar ou persuadir a mulher sobre a visão do homem para o mundo. Também foram analisados como homens estabelecem que seus comportamentos sejam consequenciados pela mulher, através de exigências, sutis ou não, sobre: obtenção de recursos (sexo, controle financeiro, serviços domésticos); atenção (pré-condição para outras estratégias); conformidade verbal (tornar a mulher complacente com grande número de demandas e comandos); concordância com os ``fatos'' (mulher, além de seguir, concorda com regras patriarcais); e evitar o controle de outros (mulher evita o controle de outras pessoas que não o companheiro). Como estratégias de controle sobre as parceiras, os homens podem usar violência física e psicológica, no entanto, posteriormente, basta um lembrete (pré-aversivo) da violência para exercer controle.

Esses padrões comportamentais podem começar com frequência e intensidade menores, porém, com o tempo se tornam mais violentos. Por fim, os autores incluíram em suas análises o papel dos contextos sociais e políticos. Argumentaram que a maioria das estratégias não funcionaria sem a aceitação social de tais contextos e privilégios diferenciais, identificados como parte do patriarcado. Embora o estudo não trate especificamente sobre a violência sexual, nos instrumentaliza para identificar que ela pode ocorrer nesses contextos. Por exemplo, é aceito socialmente que homens exerçam direito sobre a sexualidade da mulher. Utilizando estratégias classificadas como conformidade com os ``fatos'' e manutenção de segredos dentro do relacionamento, é provável que sexo forçado não seja discriminado, pela mulher, como estupro, bem como, esta apresente resistência a tornar tal assunto público através da denúncia. 

A partir dos artigos citados nesta seção, observa-se que quase todas as pesquisas feitas por analistas do comportamento sobre gênero, sexualidade e/ou violência apenas tangenciam questões sobre cultura do estupro, sem discuti-las ou mesmo usar essa expressão. Faz-se necessário uma análise feminista comportamental desses resultados, de forma a enriquecer o debate já existente em outras correntes da psicologia e, sobretudo, outros campos do conhecimento.

\section*{Por uma Análise do Comportamento comprometida com a perspectiva feminista sobre a violência sexual}

A produção acadêmica advinda de outras áreas do conhecimento apon\-ta a presença de topografias de comportamentos sexualmente abusivos por parte de homens em direção a mulheres e formula explicações para a sua existência: crenças, contexto macro, etc. Análises históricas, dados estatísticos, comparações entre diferentes sociedades, grupos e indivíduos, bem como estudos correlacionais, contribuíram para a identificação de prováveis variáveis sociais, culturais e características pessoais que favorecem a ocorrência de estupros. A análise do comportamento, no entanto, tradicionalmente investiga fenômenos através de diferentes delineamentos experimentais, mas principalmente, do delineamento de caso único (embora outros métodos também sejam empregados). Esta seria uma primeira proposta de investigação comportamental sobre a cultura do estupro. No entanto, quando se trata de determinar quais variáveis aumentam ou diminuem a probabilidade da emissão de comportamentos sexualmente abusivos, nota-se que pesquisas experimentais com manipulação de variável podem ser eticamente problemáticas. Por exemplo, uma forma de comprovar que a pornografia influencia a ocorrência de estupros seria utilizar como sujeitos garotos que nunca tivessem tido contato com pornografia. Inicialmente, deveríamos estabelecer uma linha de base sobre a frequência de comportamentos abusivos e, posteriormente, expor esses garotos a filmes pornográficos por certo período de tempo, enquanto avaliamos se houve mudança na frequência dos comportamentos de interesse. Ainda que fosse um dado importante, tratar-se-ia de um resultado indesejado, uma vez que não queremos que um indivíduo adquira um novo comportamento deste tipo.

Outra forma de manipular variáveis envolveria a redução de práticas culturais e comportamentos relacionados à dominação masculina, apresentados pelos participantes previamente, que afetam a cultura do estupro. Por exemplo, a pesquisa desenvolvida por Moxon, Keenan e Hine (1993) tinha o objetivo de reduzir estereótipos de gênero através de um procedimento de emparelhamento com o modelo. Os resultados, no entanto, foram pouco animadores, provavelmente porque se tratam de relações que fazem parte de uma rede relacional, e, nela, as relações entre os estímulos podem ter sido reforçadas por muitos anos nas vidas dos sujeitos, em diversos contextos.

Dentre os métodos não experimentais utilizados na área, destaca-se o uso dos fundamentos do Behaviorismo Radical e princípios da Análise do Comportamento para produção de interpretação e levantamento de hipóteses funcionais para analisar fenômenos culturais. Citado anteriormente, o estudo realizado por Guerin e Ortolan (2017) é um bom exemplo de utilização desse tipo de procedimento. Além disso, pode-se utilizar métodos tradicionalmente empregados por outras áreas do conhecimento e interpretá-los a partir da perspectiva analítico-comportamental.

Por fim, as analistas do comportamento podem usar dados já levantados por pesquisas feitas na área para promover a discussão feminista, através de análises culturais sobre a violência sexual. É importante, por exemplo, discutir como os estereótipos de gênero abordados nos estudos de Drake et al. (2010) e Cartwright et al. (2016) são formados e mantidos na sociedade, de que forma eles contribuem para a dominação masculina e para existência da violência sexual enquanto uma violência de gênero, e quais contribuições a análise do comportamento tem a trazer para modificar isso. Similarmente, deve-se discutir quais práticas culturais contribuem para a objetificação e desumanização da mulher, abordadas por Hussey et al. (2016), e quais intervenções podem ser feitas na cultura para que elas deixem de existir. Mais do que isso, é importante estabelecer um diálogo com outras áreas do conhecimento e suas formas de investigação. Desta forma, será possível desenvolver interpretações comportamentais sobre o que sociólogas, antropólogas, criminalistas e psicólogas vêm discutindo a respeito de cultura do estupro, produzir novos conhecimentos sobre o tema e contribuir para as tentativas de mudança desta cultura. Um primeiro passo nesse sentido foi dado no estudo de Gonçalves (2017), que utilizou o método de Análise Crítica do Discurso para identificar elementos da cultura do estupro presentes em notícias de jornais e discutir sobre práticas culturais envolvidas nesta cultura, e no artigo de Freitas e Morais (no prelo), mas ainda há um longo caminho a ser percorrido. Neste último, foi realizada uma análise comportamental de práticas culturais e comportamentos presentes na cultura do estupro a partir de considerações feministas. Conceitos da Análise do Comportamento como modelagem, modelação e equivalência de estímulos para discutir a categorização por gênero, os modelos midiáticos de papéis sexuais, o conceito de mitos sobre estupro e como nos comportamos em função deles, as práticas de culpabilização da vítima e acolhimento das mesmas e o processo de justiça relacionado à violência sexual.

Concluímos que falar sobre cultura do estupro não significa naturalizar a violência sexual, pelo contrário, visa descrever as contingências que permitem a naturalização de comportamentos e práticas culturais envolvidas na ocorrência de abusos e agressões sexuais. Neste caso, a utilização do conceito cultura do estupro em pesquisas aponta para o fato das violências sexuais acontecerem em um contexto patriarcal, portanto, prescrições para a erradicação dessa violência incluem intervenções pontuais, mas principalmente o enfraquecimento da cultura patriarcal como um todo.
\vfill
\pagebreak
\section*{Referências Bibliográficas}\sectionmark{Referências Bibliográficas}

\hangindent=25pt
\hangafter=1
\noindent Baptista, M. N., \& Campos, D. C. (2015). \textit{Metodologias de Pesquisa em Ciências: análises Quantitativa e Qualitativa}. Rio de Janeiro: LTC.

\hangindent=25pt
\hangafter=1
\noindent Barnes-Holmes, D.; Barnes-Holmes, Y., Power, P., Hayden, E., Milne, R.., \& Stewart, I. (2006). Do you really know what you believe? Developing the Implicit Relational Assessment Procedure (IRAP) as a direct measure of implicit beliefs. \textit{The Irish Psychologist, 32}(7), 169-177. 

\hangindent=25pt
\hangafter=1
\noindent Barnes-Holmes, D., Barnes-Holmes, Y., Luciano, C. \& McEnteggart, C. (2017). From the IRAP and REC model to a multi-dimensional multi-level framework for analyzing the dynamics of arbitrarily applicable relational responding. \textit{Journal of Contextual Behavioral Science, 6}(4), 434-445.

\hangindent=25pt
\hangafter=1
\noindent Brownmiller, S. (1975/1993). \textit{Against our will: Men, women and rape}. New York: Ballantine Books.

\hangindent=25pt
\hangafter=1
\noindent Bridges, A. J., Wosnitzer, R., Scharrer, E., Sun, C., \& Liberman, R. (2010). Aggression and sexual behavior in best-selling pornography videos: A content analysis update. \textit{Violence Against Women, 16}, 1065-1085.

\hangindent=25pt
\hangafter=1
\noindent Buchwald E., Fletcher, P. R., \& Roth, M. (Eds). (1993/2003). \textit{Transforming a Rape Culture}. Minneapolis: Milkweed Editions.

\hangindent=25pt
\hangafter=1
\noindent Burt, M. R. (1980). Cultural myths and supports for rape. \textit{Journal of personality and social psychology, 38}(2), 217.

\hangindent=25pt
\hangafter=1
\noindent Cartwright, A., Roche, B., Gogarty, M., O'Reilly, A., \& Stewart, I. (2016). Using Modified Function Aquisition Speed Test (FAST) for Assessing Implicit Gender Stereotypes. \textit{Psychological Record, 66}, 223-233.

\hangindent=25pt
\hangafter=1
\noindent Couto, A. G., \& Dittrich, A. (2017). Feminismo e análise do comportamento: caminhos para o diálogo. \textit{Perspectivas em Análise do Comportamento, 8}(2), 147-158.

\hangindent=25pt
\hangafter=1
\noindent Cerqueira, D., Coelho, D., \& Ferreira, H. (2017). Estupro no Brasil: vítimas, autores, fatores situacionais e evolução das notificações no sistema de saúde entre 2011 e 2014. \textit{Revista Brasileira de Segurança Pública. 11}, 24-49. 

\hangindent=25pt
\hangafter=1
\noindent Drake, C. E., Kellum, K. K., Wilson, K. G., Luoma, J. B., Weinstein, J. H., \& Adams, C. H. (2010) Examining the Implicit Relational Assessment Procedure: Four Preliminary. \textit{Psychological Record, 60}(1), 81-100.

\hangindent=25pt
\hangafter=1
\noindent Edwards, S. R., Bradshaw, K. A., \& Hinsz, V. B. (2014). Denying rape but endorsing forceful intercourse: Exploring differences among responders. \textit{Violence and Gender, 1}(4), 188-193.

\hangindent=25pt
\hangafter=1
\noindent Énois Inteligência Jovem, Instituto Vladimir Herzog \& Instituto Patrícia Galvão. (2015) \textit{\#meninapodetudo. Como o machismo e a violência contra a mulher afetam a vida das jovens das classes C, D e E?} Recuperado de \url{https://tinyurl.com/feminismoac30} 

\hangindent=25pt
\hangafter=1
\noindent Fórum Brasileiro de Segurança Pública [FBSP]. (2014). \textit{8º Anuário Brasileiro de Segurança Pública}. Recuperado de \url{https://tinyurl.com/feminismoac31}

\hangindent=25pt
\hangafter=1
\noindent Freitas, J. C. C. \& Morais, A. O. (no prelo). Cultura do estupro: algumas considerações sobre a violência sexual, feminismo e análise do comportamento.

\hangindent=25pt
\hangafter=1
\noindent Freitas, J. C. C. (2017). O IRAP como instrumento para identificação de vieses de gênero: uma revisão de literatura. In: Arantes, A. (deb.). Ninguém nasce odiando uma pessoa: revisões de literatura sobre instrumentos de medida implícita (sessão coordenada). \textit{XXVI Encontro Brasileiro de Psicologia e Medicina Comportamental}. Bauru, SP.

\hangindent=25pt
\hangafter=1
\noindent Instituto de Pesquisa e Economia Aplicada [IPEA]. (2014). \textit{Estupro no Brasil: uma radiografia segundo os dados da Saúde}.

\hangindent=25pt
\hangafter=1
\noindent Garcia, V. A., \& Turini Bolsoni-Silva, A. (2015). Transtorno de Estresse Pós-Traumático e Terapia comportamental: um estudo de caso. \textit{Acta Comportamentalia, 23}(2), 167-183

\hangindent=25pt
\hangafter=1
\noindent Gerger, H., Kley, H., Bohner, G., \& Siebler, F. (2007). The Acceptance of Modern Myths. About Sexual Aggression Scale: Development and Validation in German and English. \textit{Aggresive Behavior, 33}, 422–440.

\hangindent=25pt
\hangafter=1
\noindent Guerin, B. (2016). \textit{How to rethink human behavior: A practical guide to social contextual analysis}. London: Routledge.

\hangindent=25pt
\hangafter=1
\noindent Guerin, B., \& de Oliveira Ortolan, M. (2017). Analyzing domestic violence behaviors in their contexts: violence as a continuation of social strategies by other means. \textit{Behavior and Social Issues, 26}, 5-26.

\hangindent=25pt
\hangafter=1
\noindent Gonçalves, M. C. (2017). \textit{Violência sexual contra mulher: uma análise do discurso de portais de notícias nacionais}. (Trabalho de Conclusão de Curso). Centro Universitário Católico de Quixadá - UniCatólica, Quixadá, CE, Brasil.

\hangindent=25pt
\hangafter=1
\noindent Grubb, A., \& Harrower, J. (2008). Attribution of blame in cases of rape: An analysis of participant gender, type of rape and perceived similarity to the victim. \textit{Aggression and Violent Behavior. 13}, 396–405.

\hangindent=25pt
\hangafter=1
\noindent Hayes, S. C., Barnes-Holmes, D., \& Roche, B. (2001). \textit{Relational frame theory: A post-Skinnerian account of human language and cognition}. Springer Science \& Business Media. 

\hangindent=25pt
\hangafter=1
\noindent Hussey, I., Mhaoileoin, D., Barnes-Holmes, D., Ohtsuki, T., Kishita, N., Hughes, S., Murphy, C. (2016) The IRAP is nonrelative but not acontextual: Changes to the contrast category influence men's dehumanization of women. \textit{Psychological Record, 66}(2), 291-300.

\hangindent=25pt
\hangafter=1
\noindent Kahn, A. S., Jackson, J., Kully, C., Badger, K. \& Halvorsen, J. (2003) Calling it rape: Differences in experiences of women who do or do not label their sexual assault as rape. \textit{Psychology of Women Quarterly, 27}, 233-242.

\hangindent=25pt
\hangafter=1
\noindent Kahn, A. S. \& Mathie, V. A. (1994) Rape Scripts and Rape Acknowledgment. \textit{Psychology of Women Quarterly, 18}, 53-66.

\hangindent=25pt
\hangafter=1
\noindent Koss, Dinero, Seibel \& Cox (1988) Are There Differences In the Victim's Experience? \textit{Psychology of Women Quarterly, 12}, 1-24.

\hangindent=25pt
\hangafter=1
\noindent Layman, M. J., Gidycz, C. A. \& Lynn, S. J. (1996) Unacknowledged Versus Acknowledged Rape Victims: Situational Factors and Posttraumatic Stress. \textit{Journal of Abnormal Psychology, 105}(1), 124-131.

\hangindent=25pt
\hangafter=1
\noindent Morais, A. O. (2017). Cultura do estupro: uma análise comportamental da violência sexual em práticas culturais machistas. In: Castro, M. S. B. (deb.). Violências contra a mulher: cultura do estupro, representação na mídia e relacionamentos abusivos (mesa-redonda). \textit{XXVI Encontro Brasileiro de Psicologia e Medicina Comportamental}. Bauru, SP.

\hangindent=25pt
\hangafter=1
\noindent New York Radical Feminists; Connell, N., \& Wilson, C. (1974). \textit{Rape: the first sourcebook for women}. New American Library. 

\hangindent=25pt
\hangafter=1
\noindent Nunes, L., Hermann, C. A., \& Ratcliffe, K. (2013). Implicit and Explicit Attitudes Toward Rape are Associated With Sexual Aggression. \textit{Journal of Interpersonal Violence, 28}(13), 2657-2675.

\hangindent=25pt
\hangafter=1
\noindent Payne, D. L., Lonsway, K. A., \& Fitzgerald, L. F. (1999). Rape myth acceptance: Exploration of its structure and its measurement using the Illinois rape myth acceptance scale. \textit{Journal of Research in Personality, 33}, 27-68.

\hangindent=25pt
\hangafter=1
\noindent Peterson, Z. D., \& Muehlenhard, C. L. (2004). Was It Rape? The Function of Women’s Rape Myth Acceptance and Definitions of Sex in Labeling Their Own Experiences. \textit{Sex Roles, 51}.

\hangindent=25pt
\hangafter=1
\noindent Roche, B., \& Barnes, D. (1996). Arbitrarily applicable relational responding and sexual categorization: A critical test of the derived difference relation. \textit{Psychological Record, 46}, 451- 475.

\hangindent=25pt
\hangafter=1
\noindent Roche, B., \& Barnes, D. (1998). The experimental analysis of human sexual arousal: Some recent developments. \textit{The Behavior Analyst, 21}, 37-52.

\hangindent=25pt
\hangafter=1
\noindent Roche, B.,\& Barnes-Holmes, D. (2002). Human sexual arousal: A modern behavioral approach. \textit{The Behavior Analyst Today, 3}(2), 145.

\hangindent=25pt
\hangafter=1
\noindent Rost, M., \& Vieira, M. S. (2015) Convenções de gênero e violência vexual: a cultura do estupro no ciberespaço. Contemporanea - comunicação e cultura, \textit{13}(2). 261-276.

\hangindent=25pt
\hangafter=1
\noindent Ryan, K. (1988). Rape and seduction scripts. \textit{Psychology of Women Quarterly, 12}, 237–245.

\hangindent=25pt
\hangafter=1
\noindent Saffioti, H. I. B. (2004). \textit{Gênero, patriarcado e violência}. São Paulo: Fundação Perseu Abramo.

\hangindent=25pt
\hangafter=1
\noindent Sanday, P. R. (1981). The socio-cultural context of rape: A cross-cultural study. \textit{Journal of Social Issues, 37}(4), 5-27.

\hangindent=25pt
\hangafter=1
\noindent Sasson, S., \& Paul, L. A. (2014). Labeling acts of sexual violence: What roles do assault characteristics, attitudes, and life experiences play? \textit{Behavior and Social Issues, 23}, 35-49.

\hangindent=25pt
\hangafter=1
\noindent Smith, M. D. (2004). Encyclopedia of Rape. Westport, Conn.: Greenwood Press.

\hangindent=25pt
\hangafter=1
\noindent Stanley, N., Barter, C., Wood, M., Aghtaie, N., Larkins, C., Lanau, A., \& Överlien, C., (2016). Pornography, sexual coercion and abuse and sexting in young people’s intimate relationships: a European study. \textit{Journal of Interpersonal Violence, 6}. 1–26.

\hangindent=25pt
\hangafter=1
\noindent Tjaden P., \& Thoennes N. (2006). \textit{Extent, nature, and consequences of rape victimization: findings from the National Violence Against Women Survey: special report}. Washington, DC: National Institute of Justice. Recuperado de \url{http://www.nij.gov/pubs-sum/210346.htm}

\hangindent=25pt
\hangafter=1
\noindent Williams, L. C. A. (2002). Abuso sexual infantil. Em: H. J. Guilhardi; M. B. B. P. Madi; P. P. Queiróz \& M. C. Scoz (Orgs.) \textit{Sobre Comportamento e Cognição: Contribuições para a construção da teoria do comportamento, 10}, 144-155. Santo André: ESETec Editores Associados.

\hangindent=25pt
\hangafter=1
\noindent Williams, L. C. A, Hackbarth, C., Blefari, C. A., Padilha, M. G. S., \& Peixoto, C. E. (2014). Investigação de suspeita de abuso sexual infantojuvenil: O Protocolo NICHD. \textit{Temas em Psicologia, 22}(2), 1-18.

\hangindent=25pt
\hangafter=1
\noindent Whatley, M. A. (1996). Victim Characteristics Influencing Attributions of Responsibility to Rape Victims: A Meta-Analysis. \textit{Aggression and Violent Behavior, 1}, (2), 81-95.

