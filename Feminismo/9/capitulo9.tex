\setcounter{footnote}{0}
\setcounter{figure}{0}
\setcounter{table}{0}
\chapter*{Variáveis de gênero que terapeutas devem estar atentas no atendimento a mulheres}\sectionmark{Variáveis de gênero que terapeutas devem estar atentas no atendimento a mulheres}
\addcontentsline{toc}{chapter}{Capítulo 9}
\addcontentsline{toc}{section}{Variáveis de gênero que terapeutas devem estar atentas no atendimento a mulheres}
\addcontentsline{toc}{subsection}{\textbf{Autoras:} \textit{Renata da Conceição da Silva Pinheiro \& Cláudia Kami Bastos Oshiro}}
\begin{flushright}
\begin{small}
Renata da Conceição da Silva Pinheiro\\
Cláudia Kami Bastos Oshiro\blfootnote{Agradecemos à Psic. Ma. Priscila Rolim pelas contribuições durante a construção deste capítulo.}\blfootnote{Por ser uma profissão majoritariamente feminina, os termos psicóloga e terapeuta serão usados no feminino, conforme orientação do Conselho Federal de Psicologia.}\\
\end{small}
\vspace{1cm}
\end{flushright}

``Eu fui lá toda semana, por meses. Em uma semana, contava como me sentia culpada, o quanto eu era uma má namorada e o quanto queria mudar, e começávamos a falar sobre o que eu devia fazer para mudar. Na outra semana, estava furiosa com meu namorado, odiava tudo o que ele fazia, e começávamos a discutir como manejar os comportamentos dele que me irritavam. E assim sucessivamente. Mas eu nunca parei para pensar por que eu sentia tanta raiva, ou que eu tinha razão em ter tanta raiva, ou por que me sentia culpada, e assim permitia que ele abusasse cada vez mais de mim'' [Tulipa\footnote{Nomes fictícios.}]

Clientes com descrições de experiências psicoterápicas insatisfatórias aparecem com alguma frequência na clínica (apesar de supormos que uma boa parte desiste de procurar terapia nesses casos, o que é um problema). No entanto, este não é o caso acima. A cliente referia-se muito bem ao seu terapeuta anterior, com uma única ressalva: ele não conseguia entender. 

Mulheres e homens respondem a uma série de contingências socialmente arranjadas pelo único fato de serem daquele gênero\footnote{Gênero pode ser entendido aqui como uma construção social que organiza as relações entre homens e mulheres em determinado contexto, estruturando relações de poder desiguais (Santos, 2013).}, e muitas vezes pode ser difícil entender como as coisas funcionam para o gênero oposto. Ainda que terapeuta e cliente compartilhem do mesmo gênero, isto não é garantia de que essas contingências e tudo o que delas implicam sejam discriminadas, por dois motivos principais. Primeiro, como aponta Ruiz (1998), essas contingências discriminatórias são imbricadas em práticas culturais amplamente aceitas, tornando-as invisíveis para alguns grupos ou indíviduos – especialmente para aqueles que se beneficiam de tais práticas, mas também para aquelas que são negativamente afetadas por elas – promovendo uma espécie de ``cegueira social'' (Ruiz, 1998, p. 184).

Segundo, temos uma formação como cidadãos e como profissionais psicólogas que não favorece uma compreensão clara desses processos. No campo da análise do comportamento e terapia analítico-comportamental não é diferente, uma vez que são escassos os estudos que discutem feminismo na área (Couto \& Dittrich, 2017) e muitas das análises de casos clínicos focam o nível ontogenético de seleção por consequências, se atendo pouco às variáveis culturais, especialmente as que envolvem desigualdades de gênero.

Apesar de pouco discutidas, é provável que estas desigualdades repercutam em uma grande variedade de problemas apresentados pelas clientes, como depressão, ansiedade, estresse, fobias, dificuldades interpessoais, etc. (Fávero, 2010), e que podem muitas vezes ser ignoradas como variáveis de análise e intervenção. Nesse contexto, a terapeuta está preparada para enxergar essas variáveis muitas vezes não explicitadas em seu processo de formação? Ela ou ele vai ``conseguir entender''?

Este capítulo se propõe a apresentar e discutir, em termos gerais, contingências que comumente aparecem, implícita ou explicitamente, nas demandas de clientes mulheres em processo psicoterápico e que possuem um claro viés de gênero, agrupadas em três seções: invalidação, exigências desiguais e abuso e violência. Essa divisão tem fins didáticos, uma vez que, como ficará claro no decorrer do capítulo, elas se relacionam e se sobrepõem em muitos momentos. As discussões serão ilustradas por trechos referentes a casos reais\footnote{Agradecemos a todas que dividiram um pouco da sua história para a construção deste capítulo.} atendidos em terapia analítico-comportamental. Ao final, algumas sugestões serão fornecidas com o objetivo último de oferecer subsídios para que terapeutas observem e respondam diferencialmente a essas demandas, aumentando a probabilidade de um bom manejo clínico delas. 

\section*{Invalidação}\sectionmark{Invalidação}

``Ele tinha uma série de exigências que me incomodavam, tipo querer que a gente só saísse juntos ou sempre perguntar onde estava e o que estava fazendo pelo \textit{Whatsapp}. Eu não gostava disso, mas ele dizia que isso era companheirismo, ter uma vida a dois. Ele dizia que namorar era assim e que eu não sabia namorar, e eu lembrava que minha mãe sempre reclamava que eu era ‘independente demais’ e acreditava.'' [Tulipa]

``O modo como ele se relaciona com as amigas dele, ou mesmo com minhas amigas, beijando e abraçando, carinhoso demais, principalmente na minha frente, me incomoda. Quando digo isso para ele, ele diz que não tem por que me sentir assim, que são só amigas, que eu posso sair com quem eu quiser que ele não vai nem ligar. E aí eu me sinto horrível por estar reclamando de algo que eu deveria agradecer, que é ter um relacionamento tão maduro assim, mas acho que não sou madura o suficiente. Meus ciúmes e minha baixa autoestima acabam comigo'' [Jasmim]

O primeiro processo a ser discutido provavelmente é o mais sutil. Apesar dos casos acima parecerem opostos – de um lado, um namorado controlador restringindo a liberdade da mulher, de outro, um namorado mais ``liberal'' que deseja e incentiva mais liberdade no relacionamento – existe um ponto fundamental em comum: a invalidação dos sentimentos e percepções da mulher.

Validar significa reconhecer as expressões emocionais do outro e encontrar verdade nelas (Leahy, Tirch \& Napolitano, 2013). Emoções e sentimentos, como Skinner (1974) aponta, são subprodutos das contingências em vigor. Por exemplo, quando alguém se sente triste, a não ser que haja um problema biológico subjacente, há algo no ambiente que evoca essa tristeza, ainda que não seja facilmente identificável. No entanto, muitas vezes não somos treinadas a identificar e estabelecer essa relação funcional entre eventos privados e eventos ambientais, o que é retroalimentado por uma cultura que atribui emoções e sentimentos a características pessoais.

É através da validação que uma criança aprende a discriminar e modular emoções. Em um ambiente invalidante, ou seja, aquele em que a comunicação de experiências privadas costuma ser ignorada, banalizada ou punida, as experiências e expressões emocionais privadas do indivíduo não são consideradas respostas válidas para os acontecimentos (Linehan, 1993).

Por mais que esse processo seja possível para ambos os gêneros, deve ficar claro como essas contingências estão particularmente presentes no universo das mulheres, com raízes na estrutura patriarcal\footnote{Patriarcado pode ser entendido aqui como uma estrutura de poder que situa as mulheres muito abaixo dos homens em todas as áreas de convivência humana (Saffioti, 2015).}. Linehan (1993), quando discute sobre invalidação como uma das possíveis variáveis relacionadas ao desenvolvimento do Transtorno de Personalidade Borderline, transtorno três vezes mais prevalente em mulheres que homens (APA, 2013), aponta a desigualdade de gênero como essencial nesse processo. Por exemplo, mães e pais tendem a acreditar muito mais nos relatos de abuso sexual infantil em meninos que em meninas, tornando o gênero da vítima um forte preditor da qualidade do suporte familiar e do prognóstico da vítima (Elliott \& Carnes, 2001; Pintello \& Zuravin, 2001).

Ao longo da vida, essa contingência não muda muito. A literatura feminista já descreve como \textit{gaslighting}\footnote{O nome gaslight (luzes ou lâmpadas a gás) refere-se a um filme de mesmo nome dirigido por George Cukor em 1944, em que um marido tenta fazer com que sua esposa seja considerada louca para receber sua fortuna (Jiménez \& Varela, 2017).} um fenômeno de manipulação emocional em que uma pessoa tenta, conscientemente ou não, induzir outra a duvidar de suas reações, emoções, percepções, memórias ou crenças, como se fossem não apenas equivocadas ou infundadas como praticamente insanas (Abramson, 2014; Jiménez \& Varela, 2017). Assim, não seria só uma tentativa de desconsiderar o outro, mas também uma tentativa de que o outro também se desconsidere (Abramson, 2014). Essa situação fica bem ilustrada no trecho abaixo:

``Quando eu finalmente resolvi terminar, eu terminei. No outro dia, ele apareceu lá em casa com uma pizza e agiu como se nada tivesse acontecido. Eu perguntei ‘você lembra do que eu te disse ontem?’ e ele disse ‘eu pensei a respeito e cheguei à conclusão de que você não sabe o que é o amor, você me ama e não sabe’.'' [Tulipa]

Vale ressaltar que não só a comunicação de sentimentos e percepções é invalidada, mas sua interpretação é distorcida, sendo atribuída muitas vezes a características pessoais (carente, ciumenta, dramática, etc.), instabilidade emocional (louca, descompensada, histérica, etc.), estados fisiológicos (tensão pré-menstrual, ``falta de sexo'', etc.), ou ainda pelo próprio fato de ser mulher. Isto remete a uma cliente que trouxe para a terapia uma reportagem\footnote{Diferença no cérebro pode influenciar habilidades de homens e mulheres: Mulheres têm mais facilidade com a linguagem; homens com os cálculos. Hormônio também pode interferir no comportamento feminino e masculino. (2012, 26 outubro) G1. Recuperado a partir de \url{https://tinyurl.com/feminismo90}.}que dizia que mulheres eram mais inseguras que homens em função de diferenças no cérebro e nos hormônios, atribuindo a isto a razão da sua própria insegurança.

Essas contingências têm uma série de consequências para a formação do \textit{self}. Moreira et al. (2017), a partir de uma revisão de literatura, descrevem \textit{self} como uma resposta verbal (ou discriminação do próprio comportamento) sob controle de eventos privados relativamente estáveis ao longo do tempo e de contextos ambientais. Se o ambiente não valida minhas experiências privadas e ainda as julga como desadaptadas, eu não aprendo a confiar nelas como sinalizadores válidos (Linehan, 1993), pelo contrário, posso me tornar altamente reativa a elas em função de uma história de punição para expressá-las. Assim, eu posso ter mais dificuldade em identificar quem eu sou, o que eu gosto, minhas preferências pessoais, e responder mais em função de contingências sociais, provocando sensação de insegurança, como no caso da cliente acima. Este contexto se torna especialmente problemático frente ao grande número de exigências que se aplicam a mulher, como será melhor discutido a seguir. 

\section*{Exigências desiguais}\sectionmark{Exigências desiguais}

``Ele dizia que eu tinha relaxado sabe, que não me cuidava, não me arrumava mais, e por isso não tinha mais interesse em mim, que eu acabei com o nosso relacionamento. Realmente isso aconteceu, eu relaxei, parei de me cuidar, mas eu lembro que ele sempre foi muquirana sabe, quando eu fazia alguma coisa ele resmungava que eu estava gastando dinheiro com besteira, mesmo o dinheiro sendo meu. Lembro de uma vez que saí de casa para fazer uma limpeza de pele e menti dizendo que tinha ido para outro lugar''. [Rosa]

Rosa era uma cliente que tinha acabado de sair de um relacionamento abusivo. Ela relatava que ainda ouvia as falas depreciativas dele na cabeça e, por mais que soubesse que não era bem assim, não conseguia se olhar no espelho e não achar que ele tinha razão. A resposta do porquê isso acontecia pode ser bem mais abrangente: porque não foi só ele quem disse isso. Desde crianças, homens e mulheres têm exigências muito diferentes de como devem ser e se comportar, e isto aparece de diferentes formas no conteúdo clínico. 

Não à toa, uma das demandas clínicas mais frequentes envolve, em maior ou menor grau, questões de autoimagem, como a insatisfação com o próprio corpo, o desejo de ir regularmente à academia, a dificuldade em seguir uma dieta, o medo de envelhecer, até transtornos alimentares, como bulimia e anorexia nervosa. Apesar deste tipo de demanda estar crescendo para homens, ainda são as mulheres as mais afetadas. Segundo o Manual de Diagnóstico e Estatística dos Transtornos Mentais (DSM-V), mais de 90\% dos casos de anorexia nervosa e de bulimia ocorrem em mulheres e apontam para sua associação com uma cultura que valoriza a magreza (APA, 2013).

No entanto, ao discutir esta abordagem cultural de transtornos alimentares, Holmes, Drake, Odgers e Wilson (2017) apontam como este fator, apesar de abarcado pelo modelo biopsicossocial, é vagamente definido e não incluído efetivamente enquanto foco de tratamento. Segundo eles, a alimentação desordenada pode não ser necessariamente motivada pela busca do ideal de magreza, mas também por experiências mais amplas que envolvem as expectativas sociais acerca da feminilidade, o que explicaria a grande lacuna existente entre a prevalência em mulheres e homens. Por exemplo, espera-se que a mulher tenha menos apetite que os homens, não só em termos alimentares, mas sexuais ou mesmo econômicos, sentindo-se compelidas a refrear seus desejos ou terem seus comportamentos punidos caso não o façam (Holmes et al., 2017). A isto, Fávero (2010) nomeia de currículo da incorporação da feminilidade, referindo-se aos vários aspectos pelos quais as mulheres são cobradas, que incluem como se portar, como se vestir, como falar, como se relacionar com homens, como lidar com a sexualidade, que podem tanto produzir sofrimento e problemas psicológicos como os já citados, quanto contribuir para a manutenção da desigualdade entre os gêneros, como na alocação de recursos.

Outro aspecto, discutido por Holmes et al. (2017), envolve como o reducionismo dos aspectos culturais ao papel da mídia pode ser compreendido como banalizador e estigmatizante para muitas pacientes com anorexia nervosa, como se só a exposição continuada às propagandas fosse a causa do transtorno. A simplificação dessas explicações leva à importância de se compreender práticas culturais, especialmente as de gênero, de uma maneira mais ampla, para além do papel da mídia ou de regras sociais, mas sim como uma série de contingências, presentes no dia a dia, que modelam classes de respostas específicas consistentes com essas práticas.

Ruiz (2003), ao analisar como essas contingências são arranjadas, aponta o papel do sexo biológico como fonte de controle discriminativo sobre o comportamento de membros da comunidade. Por exemplo, frente aos mesmos estímulos ``pessoa'' e ``louça suja'', minha resposta pode ser A (lavar) ou B (solicitar que a pessoa lave) de acordo com o estímulo condicional sexo da pessoa presente. Ou, frente à mesma qualidade de caligrafia, um professor pode responder diferencialmente de acordo com o sexo da criança. Mudando o foco, a comunidade libera consequências diferenciais para respostas equivalentes de acordo com o sexo de quem se comporta, modelando os padrões de feminino e masculino naquela cultura. Ruiz (2003) aponta como, isoladamente, essas contingências podem parecer insignificantes, mas que cumulativamente contribuem para uma desigualdade de poder entre mulheres e homens. 

Contingências desiguais referentes ao desempenho de atividades domésticas e cuidado com os filhos, por exemplo, tem reflexos significativos na alocação de recursos e na qualidade de vida da mulher. Uma pesquisa do Instituto de Pesquisa Econômica Aplicada (IPEA) apontou que 90\% das mulheres e 53\% dos homens entrevistados declararam realizar essas atividades (IPEA, 2016), mostrando como a responsabilização de mulheres pelo trabalho doméstico não remunerado continua marcado. Além disso, a pesquisa também apontou as mulheres como trabalhando em média 7,5 horas a mais por semana (incluindo trabalho remunerado e não remunerado) que os homens. Não é do escopo deste capítulo discutir as variáveis sociais, econômicas e políticas que propiciaram e mantêm essas contingências, mas cabe a ênfase de que são contingências que controlam diferencialmente as respostas de mulheres e homens.

Outro exemplo envolve o ambiente de trabalho, em que mulheres podem obter consequências aversivas por emitirem as mesmas respostas que seriam reforçadas em homens. Isto fica especialmente ilustrado no caso a seguir:

``Eu era muito séria e firme, não tinha medo de ninguém e não me deixava abater. Então, depois de anos na empresa, meu chefe me deu o cargo de gerente, elogiando minhas características e dizendo que elas iam ser muito úteis nessa função. Fiquei muito feliz, mas aí as coisas foram começando a mudar. As pessoas foram se afastando de mim e se tornando mais hostis, mas até aí tudo bem, eu imaginei que isso poderia acontecer. Mas aí meu chefe começou a reclamar que eu deveria ser mais paciente, mais compreensiva, que esse meu jeito ainda ia render uma ação trabalhista para a empresa. Fiquei meio sem entender, mas comecei a tomar cuidado, a ficar mais alerta, e fui murchando. Acho que isso pode ter contribuído para eu ter me transformado no que sou hoje [refere-se à situação de não sair de casa e do diagnóstico de fobia social e agorafobia]''. [Girassol]

O trecho acima também é útil para destacar um outro ponto, especialmente relevante para o processo terapêutico, e muitas vezes sutil, que diz respeito às características socioemocionais e de relacionamento do currículo da feminilidade, que enfatizam aspectos emocionais, como ser compreensiva, amorosa, cuidadosa, geralmente alocados como características ``femininas''. Para discorrer sobre como essas características são favorecidas, Fávero (2010) utiliza o termo pedagogia do medo, como um modo de criação de meninas cercado de cuidados e precauções, a atribuição e aceitação dos seus medos (e não o incentivo para enfrentá-los), a sensação de alívio e segurança que se estabelece quando ela sai acompanhada do irmão ou com um ``homem no grupo'', dentre outras pequenas práticas que estabelecem função reforçadora condicionada à presença do homem e aversiva à ausência, como no trecho abaixo:

``Eu nunca gostei de dar satisfação para ninguém. Brigava com minha mãe porque ela queria saber meus passos, e meu irmão podia sair e dizer ‘vou ali’, chegar a hora que quiser, e tudo bem. Ela dizia que fazia isso porque o mundo era perigoso para as mulheres.'' [Tulipa]

De fato, o mundo realmente é perigoso para as mulheres, como discutiremos na próxima seção. No entanto, como contrabalancear esse tipo de perspectiva com ensinamentos e práticas que tentem não retroalimentar essa situação, uma vez que perpetuam as diferenças entre mulheres e homens e atribuem à mulher o dever de abdicar de sua liberdade e desocupar o espaço público.

Existem, ainda, outras implicações desta pedagogia do medo, como descreve Fávero (2010): 

\begin{quote}
    [...] um meio que prima pela pedagogia do medo, certamente também estará pronto para reforçá-lo com apoio e, ao mesmo tempo, cobrar esse apoio com ``juros e correções'' e essa cobrança certamente virá em forma de exigências ao cumprimento de papéis femininos, cujo fundamento principal é ser boazinha, o que significa obedecer, compactuar, atender, apoiar, ajudar, respeitar, ou traduzindo numa palavra: agradar (p. 145). 
\end{quote}

Corroborando a discussão sobre invalidação da seção anterior, e do currículo da feminilidade, discutido acima, percebe-se uma série de contingências que modelam e mantêm um padrão de tolerância a várias contingências aversivas, que vão desde a vestir roupas e sapatos desconfortáveis e arrancar os pelos do corpo a ignorar sentimentos e percepções em função da manutenção do relacionamento e da família. Dito em outros termos, ``tornar-se mulher numa sociedade patriarcal significa incorporar a feminilidade, isto é, se dissociar das próprias fomes físicas – comida e sexo, por exemplo – e treinar o corpo a mover-se ou não se mover, de modo apropriado às normas'' (Fávero, 2010, p. 231). 

Ainda que muito do que seja aqui discutido possa levar à ideia errônea de que, nesse contexto, então todas as mulheres apresentariam um padrão semelhante de comportamento, o que desconsideraria totalmente os demais níveis de seleção pelas consequências, o objetivo aqui é apresentar práticas gerais de funcionamento social, muito comuns a grande parte da população, com suas possíveis implicações para padrões de comportamento e relações de poder. De tão pervasivas, podem ser pouco percebidas e, mesmo para mulheres que fazem frente a essas exigências, não deixam de estar presentes, causar algum tipo de desconforto e exigir algum gasto de energia para tal. Além disso, são importantes para compreender quais as contingências que podem favorecer ou manter mulheres em situação de risco para uma série de violências descritas a seguir.

\section*{Abuso e Violência}\sectionmark{Abuso e Violência}

``Eu estava saindo dessa festa e peguei um táxi. Ele pediu para parar só para pegar um café, já que tinha trabalhado a noite toda. Eu disse ‘tudo bem’, e ele parou numa banquinha na calçada, desceu e perguntou se eu queria também, e eu aceitei. E só me lembro de já estar em casa. Eu não quis aceitar, ainda não quero, mas eu sei, pela sensação que tive ao acordar, que tinha acontecido alguma coisa [chora]'' [Jasmim] 

``Ele [esposo] queria transar e eu não queria, então ele começava em uma insistência absurda, que durava horas, então eu transava, mas era horrível, e no final me sentia péssima, chorava no banheiro e entrava em crise, chorava, gritava, me cortava... Eu sentia cada vez mais repulsa dele e de mim. Mas o pior, o pior era dormir e sentir ele me atacando de madrugada, me segurando e me forçando. Também quando eu estava bêbada, eu acordava e sentia o sêmen dele dentro de mim. Os \textit{flashes} vinham, e eu só queria morrer. [chora]'' [Orquídea]

Provavelmente este consiste no tópico mais difícil a ser discutido: as diversas situações de abuso e violência pelas quais mulheres passam ao longo da vida. Segundo a 11ª edição do Anuário Brasileiro de Segurança Pública (FBSP, 2017), só em 2016 houve 1 estupro a cada 11 segundos no Brasil. O próprio anuário aponta como o alvo desse crime é geralmente a mulher (85\% a 88\% dos casos), sendo os agressores geralmente homens (mais de 90\%) (FBSP, 2017).

Apesar da ênfase dada ao crime de estupro a partir dos trechos citados, existem diversas situações, das menos às mais invasivas, agressivas, visíveis, que configuram situações de abuso e violência. Abuso pode ser entendido aqui como o uso excessivo ou imoderado de poder em uma relação. Foi reconhecendo a desigualdade de poderes, ou seja, de acesso e manejo de reforçadores entre homens e mulheres e a suscetibilidade destas a uma série de violências, que foi promulgada a Lei nº 11.340/2006, conhecida como Lei Maria da Penha. Nela, violência contra a mulher é configurada como ``qualquer ação ou omissão baseada no gênero que lhe cause morte, lesão, sofrimento físico, sexual ou psicológico e dando moral ou patrimonial à mulher'' (Lei nº 11.340, 2006, p. 2). Suas diversas formas (física, psicológica, sexual, patrimonial e moral), definições e exemplos podem ser consultados em anexo.

Apesar de úteis, muitas dessas formas levam em consideração a topografia da resposta violenta, em detrimento da sua função. Nesse sentido, Guerin e Ortolan (2017), buscando realizar uma análise contextual dessas respostas, fornecem uma lista com 14 contextos de controle comportamental em casos de violência doméstica, que incluem obter controle sobre recursos, agir de modo a produzir comportamentos de fuga ou esquiva, remover fontes alternativas de recursos e/ou controle, estabelecer contextos de monitoramento, criar um contexto de sigilo, construir uma conformidade pela persuasão, \textit{bullying} ou mesmo através de comportamentos considerados ``positivos'', dentre outros. Assim, segundo eles, uma prevenção e/ou intervenção que vise treinar mulheres a reconhecerem a função do comportamento, no lugar de topografias específicas, podem ser mais eficazes, uma vez que esses comportamentos podem se apresentar de diversas formas (Guerin \& Ortolan, 2017).

Outro ponto chamado atenção pelos autores refere-se a quão inócuas ou até socialmente aceitáveis muitas dessas estratégias podem parecer, especialmente em seus estágios iniciais, como o homem se responsabilizar por sempre dirigir ou cuidar das finanças da casa (Guerin \& Ortolan, 2017), mas que podem servir de base para o desenvolvimento de maior poder e controle do homem na relação, como no exemplo abaixo:

``Eu acho que o artesanato era uma coisa que eu gostava muito de fazer. Mas droga, o A. [esposo] jogou todas as minhas coisas fora, acredita? Na mudança, eu disse para ele pegar meus artesanatos e jogar as borrachas fora, ele fez justamente o contrário, jogou todos os meus artesanatos fora, e trouxe as borrachas. [...] Eu não falei nada, porque ele já está muito estressado, já pagou toda a mudança, então não quis incomodar com mais isso''. [Girassol]

Isoladamente, é possível que o episódio seja compreendido como inócuo, mas inserido em uma análise que sinaliza um reforçamento diferencial sutil para a resposta da esposa de ficar em casa, sob o rótulo de ``superproteção'', esse episódio pode ter a função de restringir o acesso da mulher a atividades reforçadoras.

Muitos dos exemplos dados aqui envolvem relacionamentos entre namorados ou cônjuges\footnote{Os casos e informações aqui apresentados focaram em relacionamentos heteroafetivos, em função das inúmeras variáveis às quais pessoas LGBT também estão sujeitas, como experiências de discriminação ou violência, desvalorização, afastamento interpessoal, ocultação por amigos e familiares, o que exigiria maior aprofundamento.}, porém, essas situações podem se aplicar a qualquer relação em que há abuso ou violência sob controle de gênero. Por exemplo, no caso da cliente Tulipa, descrito na seção anterior, havia um claro controle de gênero no comportamento dos pais de restringirem a liberdade da filha, e muitas vezes esses e outros comportamentos, como invasão de privacidade, excesso de monitoramento, etc., são minimizados e até valorizados como ``excesso de cuidado'', ensinando e favorecendo padrões de relacionamento pouco saudáveis. Decerto, há outras desigualdades que controlam comportamentos abusivos, como desigualdade racial, econômica, de idade e/ou parentesco, que muitas vezes se sobrepõem, tornando difícil isolar a variável relevante ali, como em casos de comportamento homofóbico entre pais e filhos, em que a repressão a comportamentos ditos ``afeminados'' em meninos e ``masculinizados'' em meninas pode estar sob controle, dentre outras variáveis, de estereótipos de gênero.

Outra violência que possui um claro viés de gênero é o abuso sexual infantil, cuja proporção é de três vezes mais meninas abusadas que meninos (Stoltenborgh, Van Ijzendoorn, Euser \& Bajermans-Kranenburg, 2011). Vale salientar o quanto a divisão aqui é didática, uma vez que os três itens discutidos são interligados e se retroalimentam, como, por exemplo, abuso sexual infantil e a invalidação:

\begin{quote}
    O abuso sexual, na forma como ocorre em nossa cultura, talvez seja um dos exemplos mais claros de invalidação extrema durante a infância. No caso típico do abuso sexual, o agressor diz para a vítima que o abuso ou a relação sexual é ``normal'', mas que ela não deve contar a ninguém. O abuso raramente é reconhecido por outros familiares e, se a criança relatar o fato, corre o risco de que não acreditem nela ou a culpem (Tsai e Wagner, 1978). É difícil imaginar uma experiência mais invalidante para uma criança (Linehan, 1993, p.62).
\end{quote}

De uma maneira macro, a comunidade também perpetua essa invalidação, minimizando ou ignorando essas contingências (Salter, 2012). Existe um grau de tolerância social, como discutido por Silva, Gregoli e Ribeiro (2017), manifestada principalmente pela culpabilização da vítima e eufemização e naturalização do comportamento do agressor, que permite que os altos índices de violência contra a mulher se perpetuem. Em uma pesquisa realizada pelo Fórum Brasileiro de Segurança Pública e pelo Instituto Datafolha (FBSP, 2017), por exemplo, apontou que dois terços dos brasileiros que responderam a pesquisa (n=2.073) afirmaram já ter presenciado uma mulher sofrendo algum tipo de violência física ou verbal em 2016.

Isto se relaciona diretamente à chamada cultura do estupro, que diz respeito a uma série de práticas sutis ou explícitas que silenciam ou relativizam a violência sexual contra a mulher (Rosa, 2017). Por exemplo, uma pesquisa do IPEA (2014) encontrou que 58,5\% dos entrevistados (n=3.810) concordam total ou parcialmente com a frase ``Se as mulheres soubessem como se comportar, haveria menos estupros'', o que indica que grande parte da população ainda defende práticas que culpabilizam a vítima, o que pode incluir profissionais de agências de saúde e segurança. Como resultado, há também o grande número de subnotificações por medo de represálias ou constrangimento, além da sensação de impunidade (Santos \& Grelin, 2017), com uma estimativa de que apenas 10\% dos crimes de estupro sejam notificados (FBSP, 2017).

Assim, não é difícil que terapeutas se deparem com clientes que sofreram abuso e violência das quais não fizeram nada para produzir. Além de ser um assunto sobre o qual possa ser muito aversivo se expor, ainda pode haver uma longa história de minimização e culpabilização para isto, o que vai exigir uma boa percepção e manejo da terapeuta em sessão, como no caso citado do início da seção:

``Eu nunca falei isso em voz alta, eu sempre contei outra versão. Ninguém pode saber que fui eu que entrei naquele táxi, ninguém pode saber que fui eu que aceitei aquele café. [chora]'' [Jasmim]

\section*{Como a terapeuta pode ajudar?}\sectionmark{Como a terapeuta pode ajudar?}

Primeiro e principal ponto defendido neste capítulo: não negligencie essas variáveis. Elas existem, exercem controle sobre o comportamento dos indivíduos - inclusive das terapeutas - e precisam ser incluídas nas análises funcionais, nas conceituações de caso, nas intervenções e nas supervisões. As análises precisam incluir a dimensão cultural, de modo a não excluir uma fonte de controle importante do comportamento humano.

O segundo ponto envolve validar as percepções e sentimentos da cliente. Expressões de empatia e validação pela terapeuta são parte fundamental de todo o processo, especialmente para a própria relação terapêutica, cuja construção e manutenção deve perpassar todas as etapas da terapia, devendo subsidiar todas as intervenções sugeridas neste capítulo. Em geral, as pessoas procuram terapia por estarem em sofrimento, com eventos privados experimentados como desagradáveis, e geralmente costumam pensar que esses eventos são desajustados e não têm qualquer relação com a leitura que fazem do ambiente, aparentemente perfeito. Ao investigar um pouco, o terapeuta pode perceber que esse ambiente não é inócuo, e expor isto à cliente é essencial. 

Outra função da validação é favorecer que a cliente confie em seus sentimentos e percepções como sinalizadores válidos do ambiente (Linehan, 1993). Assim, se estabelece o contexto para que se ensine o papel dos sentimentos enquanto produtos de contingências, e não como causas de muitos dos seus problemas, o que serve também para ensinar uma perspectiva externalista à cliente, a ``olhar para fora'' e ver que contingências estão produzindo essas percepções e sentimentos. 

Resgatando o trecho da cliente Jasmim, referente à relação com o namorado ``liberal'', que atribuía aos seus ciúmes e sua baixa autoestima a razão dos seus problemas

``Por mais que não briguemos, eu me sinto muito mal, eles acabam tirando o que pior há de mim. Tenho pensado se eu sou o problema, se eu sou a pessoa tóxica, com base nos outros relacionamentos tóxicos que já vivi [Jasmim já esteve em um relacionamento violento]''

Vale ressaltar que validar não é corroborar que a pessoa se sinta assim, como corroborar que a cliente se sinta uma ``pessoa tóxica'' ou, citando outro caso, uma ``má mãe'', por exemplo, mas sim discutir como pensamentos e sentimentos são produtos da história de aprendizagem da pessoa e reflexos das condições atuais de sua vida, portanto, como faz sentido que ela se sinta assim em um ambiente, pessoal e social, que promove isso. Dessa maneira, se estabelece o contexto para mudar o foco para a análise das relações interpessoais e variáveis culturais que podem estar produzindo esse sentimento. 

Falas de culpa e minimização também devem ser recebidas com atenção pela terapeuta, uma vez que, como muito do comportamento verbal da cliente envolve também o que ela aprendeu do ambiente, costumam ser reflexos das relações discutidas no tópico de invalidação. Quando questionada sobre de onde pode vir ela achar que não deveria se sentir assim, Jasmim respondeu:

``Eu sei de onde vem. De A. (nome do namorado). Ele diz que ele não se importa, ou não se importaria se fosse comigo, e que não tem porque eu me importar. Quando eu disse que não me sentia confortável por ele ficar amigo da minha amiga, ele disse que eu estou errada e que não pararia. [...] Porque ele coloca sempre como se a culpa fosse minha por me importar. ‘Pra mim não importa, então você que tá errada de se importar’. Mas ele não lembra que eu não faço nada disso, nada que possa fazer ele se sentir mal.'' [Jasmim]

Geralmente, essas variáveis não são descritas tão facilmente em terapia, por serem sequer discriminadas pela cliente. Muitos dos trechos aqui apresentados são produtos de um \textit{continuum} de relatos, questionamentos, análises, etc. Cabe à terapeuta a habilidade de fazer perguntas que coloquem em evidência as propriedades relevantes daquela situação que podem estar evocando estes sentimentos, e que também favoreça a discriminação dessas relações pela cliente, ou seja, favoreça o autoconhecimento. Por exemplo, no caso acima, não só a situação de estreita amizade entre namorado e amiga evocava sentimentos aversivos, mas a invalidação do namorado frente à exposição dela. Um ponto que chama atenção nesse caso é o quanto topograficamente o relacionamento parecia ``maduro'', segundo relato da cliente, algo especialmente valorizado pelo namorado, que era envolvido em projetos sociais, defesa de direitos humanos e feminismo, e o quanto ainda existia uma desigualdade que fazia com que, na prática, a opinião de Jasmim fosse desconsiderada em função da dele. 

Uma pergunta que pode surgir é o quanto o gênero pode ser uma variável relevante para cada caso. Uma ferramenta útil nessa tarefa de ``garimpar'' quais as propriedades relevantes de uma situação é a troca de papeis. Por exemplo, no trecho da cliente Girassol, na seção anterior, que não reclamou com o esposo sobre ele ter jogado os artesanatos dela no lixo, quando convidada pela terapeuta a mudar os papeis no sentido de descobrir as variáveis relevantes, respondeu:

``[T: E se você estivesse contribuindo igualmente com as despesas da mudança?] Ainda não falaria. [T: E se você estivesse pagando tudo?] Acho que aí então eu falaria. [T: Engraçado como se vocês ainda estão em situação equivalente, contribuindo igualmente com as despesas, você ainda não falaria.] Eu não me sentiria no direito de falar.'' [Girassol]

Uma das contingências discutidas ao longo do capítulo e que se relacionam aos dois trechos apresentados acima envolve como, em geral, homens tem comportamento de expressar e manter opinião reforçados, enquanto mulheres tem esse mesmo comportamento punido. Por isso, a terapeuta pode se valer dessas informações como relevante para análise e intervenção, como no trecho abaixo:

``Gostaria de colocar mais uma variável na sua análise. Uma ideia que tenho observado refere-se ao quanto a opinião feminina tende a perder quando relacionada a uma opinião masculina. Parece que a gente aprende a questionar constantemente nossa ideia, e ela cede muito facilmente frente aos empecilhos, ou nós mesmos a colocamos em xeque, realidade diferente para os meninos, como se a opinião deles tivesse muito mais peso. Por exemplo, no seu caso, ele nem considera em mudar sua opinião ou sua postura, enquanto você se acha culpada por tudo o que está acontecendo de ruim. Eu acho que isso tem a ver com uma sociedade que pensa e ensina dessa maneira, e trata diferente as opiniões de acordo com o gênero. O que você acha?'' [T]

Apontar essas variáveis e evocar uma discussão clara sobre as desigualdades de gênero em sessão têm uma série de benefícios. Primeiro, tem função evocativa, fazendo com que a cliente descreva outras situações semelhantes e estabeleça novas relações que possam estar controlando seu comportamento. Segundo, tem função de validação, uma vez que a cliente percebe que não se tratava de uma incapacidade pessoal ou algo errado com ela, que faz sentido se sentir assim em um contexto aversivo e de exigências contraditórias. Terceiro, favorece a discriminação de situações funcionalmente semelhantes e, consequentemente, a generalização dos comportamentos aprendidos para novos contextos. Por exemplo, frente à discussão, Girassol também trouxe outras situações que a tinham ``afetado'', mas que até então não percebia que estavam sob controle do seu gênero, como no trecho abaixo: 

``Todo mundo dizia que eu era ambiciosa demais, só porque eu trabalhava muito. Mas eu amo trabalhar, nem era só pelo dinheiro. E mesmo que fosse, meu marido trabalha o mesmo tanto que eu e é elogiado. Mas parece que eu trabalhar incomoda, sabe? Quando eu chegava nos lugares, as pessoas faziam ``plim-plim-plim'', como se fosse um saco de moedas batendo.'' [Girassol]

O mais interessante dessas discussões refere-se ao fato da cliente não ter um histórico de leituras feministas, sugerindo que foi um processo de discriminação favorecido em terapia. Frente a descrições como a de Jasmim e de Girassol, ficar restrita a uma análise molecular, identificando para as respostas de assertividade uma contingência de punição experienciada na vida daquela cliente, por exemplo, faz perder uma parte importante do padrão de funcionamento social implicado aí. Como aponta Ruiz (1998), o desenvolvimento de uma resistência a essas desigualdades envolve o desenvolvimento de dois repertórios distintos: o ``saber como'', que é desenvolvido através do treino direto, mas também o ``saber o que'', que envolve o saber explicar a resposta de resistência e sua relação funcional com as variáveis ambientais (p. 189), o que exige uma discussão clara desses processos. Quando discutimos com a cliente que isto é cultural, e não restrito à relação com os pais, com o namorado, marido ou com o chefe, favorecemos um processo de abstração e ensinamos comportamentos de proteção, como identificar quando elas estão presentes, treinar respostas assertivas, buscar fontes alternativas de apoio, ou mesmo ignorar, especialmente se tratando de um fenômeno pervasivo e de difícil mudança. Assim, a cliente estará muito mais preparada para identificar e responder a todas as situações difíceis que irá enfrentar, como no trecho abaixo, de uma cliente em processo psicoterápico que havia sido vítima de um estupro recente: 

``Eu fui ao hospital e foi horrível. O médico, a enfermeira, cada técnica que tirava meu sangue ou me dava remédio, cada faxineira que passasse cinco minutos na mesma sala que eu perguntava como tinha acontecido, onde eu estava, que horas eram... Aquilo foi me dando um ódio, porque eu sabia que eles não queriam me ajudar, eles queriam me culpar, então eu simplesmente parei de responder.'' [Orquídea] 

Cabe salientar os efeitos iatrogênicos desse tipo de pergunta realizada por profissionais – aliás, por qualquer pessoa –, inclusive as psicólogas. Em terapia, o terapeuta deve estar sempre sob controle da conceituação de caso, e suas perguntas e intervenções devem ter uma função clara para o processo, e que não envolve curiosidades pessoais, que podem se revelar contra-terapêuticas, além de poder trazer um grande prejuízo à relação terapeuta-cliente.

Retomando o trecho de Orquídea, foi interessante observar como, apesar de estar em situação de vulnerabilidade, de ter passado por uma situação de violência e de prever o contexto aversivo que encontraria, a cliente manteve-se no propósito de realizar os exames e tomar as precauções necessárias. Outro trecho pode ser visto abaixo:

``A gente estava escolhendo um filme para assistir, e passamos por um canal que estava passando meu filme favorito, então eu disse ‘olha, esse é meu filme favorito!’ e ele disse ‘Esse? Nossa...’ [cliente descreve sinais de desprezo] e mudou de canal. Eu senti na hora um soco no estômago. Então pensei no que tinha acontecido, demorei alguns minutos me acalmando, e aí perguntei com voz calma ‘Por que você faz isso?’, ele ‘O quê?’, ‘Fica menosprezando o que eu gosto, dá a entender que é ridículo eu gostar desse filme, eu não ia pedir pra gente assistir, só estava comentando’''. [Jasmim]

Esse trecho sintetiza o resultado de muitas das estratégias discutidas ao longo dessa seção, como a identificação e validação dos próprios sentimentos, a busca por variáveis externas que os evocam, e a resposta assertiva. Vale ressaltar que essas sugestões nem esgotam a infinidade de possibilidades, nem devem ser utilizadas indiscriminadamente, pois as intervenções devem sempre responder à conceituação de caso. Por exemplo, ainda que o termo ``pedir'' ainda represente essa desigualdade, direcionar a discussão para esse aspecto provavelmente teria função punitiva. Especificamente para este caso, a postura da terapeuta foi elucidar as possíveis consequências reforçadoras naturais da resposta assertiva, perguntando como a cliente se sentiu ao fazer isso ou enfatizando o quanto isso se aproxima do que ela havia descrito como a namorada que gostaria de ser – menos explosiva e mais assertiva.

A elucidação de valores pessoais é essencial para direcionar muitos dos comportamentos da cliente e da terapeuta em sessão, mas devem ser conduzidas com cuidado nesse contexto onde as demandas sociais são muito fortes e se misturam com os interesses pessoais. Uma observação importante envolve a discriminação, pela terapeuta, dos seus valores pessoais em distinção aos valores da cliente. A terapeuta pode ter suas concepções e valores pessoais, sejam eles quais forem, porém deve tomar cuidado com como eles podem estar controlando a sua resposta em sessão. Por exemplo, uma cliente que deseja se dedicar à vida familiar em detrimento da profissional pode evocar na terapeuta um direcionamento, ainda que sutil, para uma opção valorizada pela terapeuta, o que é um problema, uma vez que os comportamentos desta em sessão devem estar sob controle primordialmente da conceituação de caso. Mesmo um posicionamento feminista deve ser funcionalmente pensado, uma vez que a busca pela igualdade de direitos e poderes entre os gêneros\footnote{Apesar de existirem vários tipos de feminismo (ver capítulo 03), existe uma definição comum que envolve a busca por igualdade de direitos entre os gêneros.} tem o objetivo último do bem-estar dos envolvidos, ou seja, que todos gozem de plenos direitos e tenham igual acesso a recursos para desenvolver as habilidades e ter acesso aos reforçadores que se alinham aos seus valores pessoais. Como aponta Adichie\footnote{Livros da Chimamanda Ngozi Adichie (2014; 2017) são muito úteis para discutir desigualdades de gênero pois são curtos e de fácil leitura.}:

\begin{quote}
    A segunda ferramenta é uma pergunta: a gente pode inverter X e ter os mesmos resultados? Por exemplo: muita gente acredita que, diante da infidelidade do marido, a reação feminista de uma mulher deveria ser deixá-lo. Mas acho que ficar também pode ser uma escolha feminista, dependendo do contexto. Se o Chudi dorme com outra mulher e você o perdoa, será que a mesma coisa aconteceria se você dormisse com outro homem? Se a resposta for ‘sim’, então sua decisão de perdoá-lo pode ser uma escolha feminista, porque não é moldada pela desigualdade de gênero (p. 12).
\end{quote}

Por fim, é importante notar como discutir questões sobre desigualdade de gênero e feminismo ainda pode ser um tabu para terapeutas e clientes, especialmente em função da sua associação com ideias de ódio aos homens ou repúdio à feminilidade (Holmes et al., 2017), apesar dessa realidade estar mudando. Apontar as variáveis que representam a desigualdade entre gêneros é muitas vezes desconfortável, uma vez que significa mexer em contingências que estruturam relações que envolvem mulheres e homens na nossa sociedade e solicitar mudança. À terapeuta, cabe sempre se valer de supervisão clínica e terapia pessoal, de modo a ter clareza de quais variáveis influenciam no seu comportamento, e definir um melhor caminho, levando em consideração o caso clínico e suas limitações pessoais.

\section*{Considerações finais}\sectionmark{Considerações finais}

Ao discutir a abordagem cultural aos transtornos alimentares, Holmes et al. (2017) observou que muitas pacientes já traziam questões de gênero para as(os) profissionais e eram ignoradas, seja por falta de preparo destas(es), seja por diferentes concepções culturais. Foi identificando esta problemática, associada à escassez de literatura sobre o assunto, especialmente na análise do comportamento, que este capítulo se propôs a discutir variáveis de gênero que podem estar envolvidas nas demandas clínicas de clientes em terapia. A ideia é que não só terapeutas estejam mais aptas a reconhecer as influências dessas variáveis, como possam responder diferencialmente a elas. Isto não só amplia as possibilidades de análises e intervenções e sua eficácia, como evita uma série de problemas oriundos da negligência e/ou invalidação pela terapeuta.

Vivemos em uma sociedade desigual em vários aspectos: social, racial, econômica, de gênero, orientação sexual, religião, e terapeutas tanto fazem parte desse ambiente quanto podem perpetuá-lo inadvertidamente, contribuindo para o sofrimento da cliente e, de uma maneira mais ampla, para a perpetuação dessas desigualdades. Perceber que essas desigualdades estão imbrincadas em nossas ações, concepções e valores, e lutar por uma sociedade mais justa e igualitária é também nosso papel, dentro e fora de terapia.
\vfill
\newpage
\section*{Referências Bibliográficas}\sectionmark{Referências Bibliográficas}

\hangindent=25pt
\hangafter=1
\noindent Abramson, K. (2014). Turning up the lights on gaslighting. \textit{Philosophical Perspectives, 28}, 1-30.

\hangindent=25pt
\hangafter=1
\noindent Adichie, C. N. (2014). \textit{Sejamos todos feministas}. São Paulo: Companhia das Letras.

\hangindent=25pt
\hangafter=1
\noindent Adichie, C. N. (2017). \textit{Para educar crianças feministas: um manifesto.} São Paulo: Companhia das Letras. 

\hangindent=25pt
\hangafter=1
\noindent American Psychiatry Association (2013). \textit{Diagnostic and Statistical Manual of Mental disorders}, 5ª ed. Washington: American Psychiatric Association.

\hangindent=25pt
\hangafter=1
\noindent Brasil. Lei n 11.340, de 7 de agosto de 2006. Cria mecanismos para coibir a violência doméstica e familiar contra a mulher. Recuperado de: \url{https://tinyurl.com/feminismo91}. 

\hangindent=25pt
\hangafter=1
\noindent Couto, A. G., \& Dittrich, A. (2017). Feminismo e análise do comportamento: Caminhos para o diálogo. \textit{Revista Perspectivas em Análise do Comportamento, 8}(2), 147-158.

\hangindent=25pt
\hangafter=1
\noindent Elliott, A. N., \& Carnes, C. N. (2001). Reactions of Non offending parents to the sexual abuse of their child: A review of the literature. \textit{Child Maltreatment, 6}, 4, 314-331.

\hangindent=25pt
\hangafter=1
\noindent Fávero, M. H. (2010). \textit{Psicologia do Gênero: Psicobiografia, sociocultural e transformações.} Curitiba: Ed. UFPR.

\hangindent=25pt
\hangafter=1
\noindent Fórum Brasileiro de Segurança Pública (2017a). Anuário Brasileiro de Segurança Pública, 11ª ed. Recuperado de: \url{https://tinyurl.com/feminismo92}. 

\hangindent=25pt
\hangafter=1
\noindent Fórum Brasileiro de Segurança Pública (2017b). \textit{Visível e Invisível: a vitimização de mulheres no Brasil.} Recuperado de: \url{https://tinyurl.com/y827aa6g}

\hangindent=25pt
\hangafter=1
\noindent Guerin, B., \& Ortolan, M., O. (2017). Analyzing domestic violence behaviors in their contexts: Violence as a continuation of social strategies by other means. \textit{Behavior and Social Issues, 26}, 5-26.

\hangindent=25pt
\hangafter=1
\noindent Holmes, S., Drake, S., Odgers, K., \& Wilson, J. (2017). Feminist approaches to Anorexia Nervosa: a qualitative study of a treatment group. \textit{Journal of Eating Disorders, 5}(36), 1-15.

\hangindent=25pt
\hangafter=1
\noindent Instituto de Pesquisa Econômica Aplicada (2014). \textit{Tolerância social à violência contra as mulheres.} Recuperado de: \url{https://tinyurl.com/feminismo93}. 

\hangindent=25pt
\hangafter=1
\noindent Instituto de Pesquisa Econômica Aplicada (2016). \textit{Retrato das desigualdades de gênero e raça – 1995 a 2015.} Recuperado de: \url{https://tinyurl.com/feminismo94}.

\hangindent=25pt
\hangafter=1
\noindent Jiménez, J. S. G., \& Varela, M. R. F. (2017). Gaslighting: La invisible violencia psicológica. \textit{UARICHA Revista de Psicología¸ 14}(32), 53-60.

\hangindent=25pt
\hangafter=1
\noindent Leahy, R. L., Tirch, D., \& Napolitano, L. A. (2013). \textit{Regulação emocional em psicoterapia: um guia para o terapeuta cognitivo-compor\-tamental.} Porto Alegre: Artmed.

\hangindent=25pt
\hangafter=1
\noindent Linehan, M. (2010). \textit{Terapia cognitivo-comportamental para transtorno de personalidade borderline.} Porto Alegre: Artmed. 

\hangindent=25pt
\hangafter=1
\noindent Moreira, F. R., Silva, E. F., Lima, G. O., Assaz, D. A., Oshiro, C. K. B, \& Meyer, S. B. Comparação entre os conceitos de self na FAP, na ACT e na obra de Skinner. \textit{Revista Brasileira de Terapia Comportamental e Cognitiva, 19}(3), 220-237.

\hangindent=25pt
\hangafter=1
\noindent Pintello, D., \& Zuravin, S. (2001). Intrafamilial child sexual abuse: Predictors of postdisclosure maternal belief and protective action. \textit{Child Maltreatment, 6}, 344-352.

\hangindent=25pt
\hangafter=1
\noindent Rosa, C. T. (2017). A perícia nos casos de estupro: compreensão, desafios e perspectivas. In: Fórum Brasileiro de Segurança Pública. Anuário Brasileiro de Segurança Pública (pp. 44-45), 11ª ed. Recuperado de: \url{https://tinyurl.com/feminismo95}.

\hangindent=25pt
\hangafter=1
\noindent Ruiz, M. R. (1998). Personal agency in feminist theory: evicting the illusive dweller. \textit{The behavior analyst, 21}(2), 179-192.

\hangindent=25pt
\hangafter=1
\noindent Ruiz, M. R. (2003). Inconspicuous sources of behavioral control: The case of gendered practices. \textit{The Behaviorist Analyst Today, 4}(1), 12-16. 

\hangindent=25pt
\hangafter=1
\noindent Saffioti, H. (2015). G\textit{ênero, patriarcado e violência.} 2ª ed. São Paulo: Expressão Popular: Fundação Perseu Abramo.

\hangindent=25pt
\hangafter=1
\noindent Salter, M. (2012). Invalidation: A Neglected Dimension of Gender‐based Violence and Inequality. \textit{International Journal for Crime and Justice, 1}(1), 3-13.

\hangindent=25pt
\hangafter=1
\noindent Santos, H. M. (2013). A importância de discutir gênero na Psicologia. In: Andrade, D. S. V., \& Santos, H. M. (Org.). \textit{Gênero na Psicologia: articulações e discussões.} (p. 19-33). Salvador: CRP-03.

\hangindent=25pt
\hangafter=1
\noindent Santos, M., O., P., \& Grelin, D., M. (2017). Violências invisíveis: o não óbvio em evidência. In: Fórum Brasileiro de Segurança Pública. Visível e Invisível: a vitimização de mulheres no Brasil (pp. 35-39). Recuperado de: \url{https://tinyurl.com/feminismo96}

\hangindent=25pt
\hangafter=1
\noindent Secretaria de Políticas para as Mulheres (2015). Viver sem violência é direito de toda mulher: entenda a Lei Maria da Penha. Recuperado de: \url{https://tinyurl.com/feminismo97}.

\hangindent=25pt
\hangafter=1
\noindent Silva, R., V., Gregoli, R., \& Ribeiro, H., M. (2017). Resultado de pesquisa expõe tolerância social à violência contra as mulheres em espaços públicos. In: \textit{Fórum Brasileiro de Segurança Pública. Visível e Invisível: a vitimização de mulheres no Brasil} (pp. 25-28). Recuperado de: \url{https://tinyurl.com/feminismo98}.

\hangindent=25pt
\hangafter=1
\noindent Skinner, B. F. (1974). \textit{About behaviorism.} New York: Alfred A. Knopf.

\hangindent=25pt
\hangafter=1
\noindent Stoltenborgh, M., Van Ijzendoorn, M. H., Euser, E., \& Bajermans-Kranenburg, M. J. (2011). A Global Perspective on Child Sexual Abuse: Meta-Analysis of Prevalence Around the World. \textit{Child Maltreatment, 16}(2), 79-101.
