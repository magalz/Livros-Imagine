\chapter{O empoderamento das mulheres sob uma perspectiva analítico-comportamental}\sectionmark{O empoderamento das mulheres sob uma perspectiva analítico-comportamental}\blfootnote{O trabalho aqui publicado é fruto da dissertação de mestrado da autora. Agradeço imensamente ao prof. Alexandre Dittrich, orientador deste trabalho, e às professoras Carolina Laurenti e Camila Muchon de Melo, cujas contribuições em banca foram de grande valia para desenvolver os presentes argumentos.}
\begin{flushright}
\begin{scriptsize}
    Aline Guimarães Couto
\end{scriptsize}
\vspace{1cm}
\end{flushright}

A palavra \textit{empoderamento} é repetida constantemente nos meios feministas. É uma resposta verbal frequentemente reforçada em uma ampla gama de contextos relacionados ao movimento, se convertendo em solução frequente a um problema difícil de resolver: como levar a categoria \textit{mulher} à tão sonhada liberdade? Como superar a opressão e a desigualdade?

Apesar da sonoridade da palavra e do quanto ouvimos ela se repetir, como muitos termos utilizados pelo movimento feminista, \textit{empoderamento} tem uma definição difícil. O uso em contextos ligados a movimentos sociais, em geral, tem suas raízes atribuídas ao uso proposto por Paulo Freire em suas obras sobre a educação e seu papel transformador da realidade social, com foco não apenas em questões individuais, mas em questões coletivas – relacionando, portanto, o empoderamento com o processo pelo qual grupos excluídos são emancipados (Baquero, 2012).

Tomando as mulheres como um dos grupos que sofrem opressão na sociedade, o empoderamento passou a ser discutido pela teoria feminista, guardando proximidade com a concepção freiriana em alguns momentos e se afastando em outros. Parece ser tanto um produto, um objetivo último, quanto um processo que leva as mulheres à autonomia (Sardenberg, 2012). Segundo a autora, a palavra \textit{empoderamento} se refere basicamente a três aspectos da ação feminina: questionamento da ideologia patriarcal, transformação das estruturas e instituições que reforçam e mantêm essa ideologia e a criação das condições para que ambos sejam feitos pelas mulheres, em especial pelas mais pobres. 

Ele se refere, em última análise, aos meios concretos pelos quais cada mulher pode se libertar de um contexto opressor ou à libertação desse contexto em si, carregando em sua definição uma tensão entre aspectos individuais e coletivos – sendo uma noção criticada por vezes, inclusive por feministas, por apresentar, em alguns contextos, um viés excessivamente focado nas liberdades individuais em detrimento da liberdade das mulheres enquanto classe de indivíduos (Sardenberg, 2008). 

Não existe consenso sobre o termo, e ele é usado por vezes indiscriminadamente para falar de ações que não necessariamente têm objetivos últimos libertadores, segundo análises de algumas correntes feministas (Sardenberg, 2008). Um exemplo particularmente discutido é o empoderamento sexual. Tomando como base o anseio feminista de que todas as mulheres possam exercer sua sexualidade como desejarem, a mídia e a sociedade retratam a ``mulher empoderada sexualmente'' como a mulher que consome produtos que a deixam \textit{sexy}, desejável, sem se importar com concepções moralistas – mas o modelo a ser imitado aí é o de uma mulher heterossexual, que nunca recusa uma relação sexual e que se veste e se porta de acordo com um ideal de sexualidade bastante específico; em última análise, algo vantajoso para o gênero masculino (Gill, 2008).

Muitas críticas também são feitas na literatura feminista sobre o uso do \textit{empoderamento} sendo apropriado fortemente por organizações que o associam quase que exclusivamente a programas de microcrédito em nações empobrecidas, reduzindo o empoderamento feminino aos seus aspectos econômicos. Embora a pobreza tenha um papel central na vivência das mulheres nesses países, cerceando, por exemplo, sua liberdade em deixar companheiros por dependência financeira, os programas parecem aplicar lógicas que, a longo prazo, perpetuam a posição social das mulheres, ao emprestar somas que permitem apenas pequenos negócios, muitas vezes ligados a estereótipos femininos como salões de cabeleireiro ou venda de artesanato – algo que, em última análise, mantém mulheres distantes do trabalho em empresas onde podem galgar posições de destaque (c. f. Mayoux, 1999); também diminuem a importância da educação formal em tais iniciativas. A maioria dos estudos que relacionam empoderamento das mulheres e microcrédito parece levar em conta apenas uma ou outra medida, ignorando conjuntos de variáveis mais amplos e efeitos a longo prazo (Norwood, 2014; Addae, 2015).

Entre autores da psicologia, os usos da palavra empoderamento na literatura refletem as tendências discutidas no campo feminista. Uma crítica contundente a estes usos é realizada por Stephanie Riger, psicóloga social americana. Riger (1993) destaca dois aspectos comuns na discussão sobre empoderamento: é encarado frequentemente de forma individualista, dando pouca importância a contextos sociais e enfoque em aspectos relacionados à sensação de controle ou à percepção de uma pessoa sobre a forma com que ela vive; e é visto de forma em que o poder é associado a noções masculinas de competição e controle, em vez de noções comuns às mais identificadas com o feminino de cooperação e comunidade. A autora ainda afirma que parte desses usos pode se dever a uma visão dualista do ser humano, em que a realidade é vista como ``criada'' de acordo com o que alguém pensa; uma pessoa que passou por um processo de empoderamento enxergaria suas capacidades de forma diferente e isso a faz ter poder em relação ao mundo. Para Riger (1993), tal noção isola as ações do ser humano dos contextos sociais nos quais elas ocorrem, imprescindíveis para uma discussão sobre poder, além de psicologizar um fenômeno complexo, reduzindo o mesmo a aspectos internos de um indivíduo.

Tais críticas vêm ao encontro de noções behavioristas radicais sobre o comportamento humano, impossível de ser visto isolado do ambiente em que ocorre. É famosa a asserção de Skinner (1957) que abre o livro \textit{Verbal Behavior}, considerada pelo autor sua obra mais importante, que sumariza a importância do ambiente na filosofia behaviorista radical:

\begin{quote}
    Os homens agem sobre o mundo, mudam-o, e são mudados pelas consequências de suas ações. Certos processos, que o organismo humano compartilha com outras espécies, alteram o comportamento e este conquista formas mais seguras e úteis de troca com um ambiente em particular. Quando o comportamento apropriado foi estabelecido, suas consequências funcionam de acordo com processos similares para mantê-lo em vigor. Se o ambiente por acaso muda, velhas formas de se comportar desaparecem, enquanto novas consequências constroem novas formas (Skinner, 1957, p. 1, tradução livre)\footnote{``Men act upon the world, and change it, and are changed in turn by the consequences of their action. Certain processes, which the human organism shares with other species, alter behavior so that it achieves a safer and more useful interchange with a particular environment. When appropriate behavior has been established, its consequences work through similar processes to keep it in force. If by chance the environment changes, old forms of behavior disappear, while new consequences build new forms''. (p. 1)}.
\end{quote}

Se considerarmos que o autor, apesar de citar aqui nominalmente apenas os homens, teve a bondade de considerar as mulheres entre os organismos citados, então, muito do que se fala sobre empoderamento pode ser levado em conta quando falamos das possibilidades transformadoras da sociedade por meio das ações humanas.

Ao mesmo tempo, a falta de literatura da análise do comportamento sobre questões relacionadas ao feminismo dificulta uma reflexão mais aprofundada sobre o termo. Relacionando-se à discussão sobre o conceito de poder e sua importância no feminismo, discutida no presente livro (Arantes e Nicolodi, 2018), este capítulo busca discutir noções presentes em artigos que versam sobre o empoderamento da mulher, de um ponto de vista analítico-comportamental. 

\section{Considerações sobre a metodologia empregada}

A polissemia do termo empoderamento e o uso frequente nos dias atuais pelo movimento feminista levanta aqui a seguinte questão: do que o feminismo fala quando fala de empoderamento da mulher? Para entender o fenômeno, realizamos aqui uma análise de produtos do comportamento verbal presentes na literatura científica feminista. 

As considerações aqui apresentadas se baseiam em uma revisão de literatura cuja metodologia está detalhada na dissertação de Couto (2017). A revisão consistiu em buscar, em periódicos científicos que publicam textos de temática e viés feminista de diversas áreas, menções ao termo empoderamento dos anos de 2006 a 2016, com a seleção de trechos pertencentes aos textos que pudessem fornecer pistas sobre o que controla a emissão da resposta verbal \textit{empoderamento} em tais textos. Os trechos selecionados permitiram categorizar o empoderamento conforme duas noções principais: \textit{empoderamento com base em estados internos e empoderamento com base no contracontrole}.

\section{Empoderamento com base em estados internos}

Na literatura feminista consultada, podemos observar o \textit{empoderamento com base em estados internos} como a classificação que, em geral, considera comportamentos relacionados à expressão verbal de sentimentos das mulheres estudadas como característicos do processo de empoderamento. Alguns dos trechos encontrados na literatura realizam descrições verbais das mulheres como empoderadas em determinado contexto a partir do momento em que elas adquirem consciência do controle exercido por práticas culturais machistas.

Em alguns casos, essa literatura permite presumir mudanças nos contextos em que as mulheres viviam que poderiam ser responsáveis pelos sentimentos descritos como relacionados ao empoderamento. Mesmo assim, a palavra surge sob controle dos sentimentos que as mulheres relatam, a exemplo de Nasrabadi (2014), que discute a participação feminina no movimento estudantil à época da revolução iraniana:

\begin{quote}
    Dentro dos parâmetros de ``igualdade de gênero'', algumas mulheres experimentaram novos \textbf{sentimentos de empoderamento e pertencimento}. Soosan chegou aos Estados Unidos em 1978 e rapidamente se incluiu no grupo de esquerda de estudantes iranianos. Após a revolução, ela ficou quase um ano na prisão no Irã e escapou por pouco da execução. Soosan relembra sua participação no ISA [\textit{Iranian Students Association}] como ``realmente uma das melhores épocas da minha vida''. Ela sorri quando relembra ``aquela paixão de fazer algo'' coletivamente. ‘Nós éramos tão iguais’, ela disse. ``Eu ou um homem podia ser líder de notícias.'' (Nasrabadi, 2014, p. 133, tradução livre, grifo meu)\footnote{``Within the parameters of gender sameness, some women experienced novel \textbf{feelings of empowerment and belonging}. Soosan came to the United States in 1978 and quickly became immersed in the Iranian student Left in Berkeley. After the revolution, she would spend almost a year in prison in Iran and narrowly escape execution. Soosan remembered her participation in the ISA as ‘really one of my best times that I had in my life.’ She smiled when she recalled ‘that passion of doing something’ collectively. ‘We were so equal,’ she said. ‘I could be news leader or a man could be’.'' (p. 133)}
\end{quote}

Em outros textos, a relação com o empoderamento é descrita pelas autoras como parte dos sentimentos que as mulheres descrevem ter em contextos que, de alguma forma, desafiam as normas relacionadas ao comportamento que é esperado na sociedade pelas mulheres. Medved (2009), em um artigo no qual relata entrevistas com mulheres que são provedoras financeiras das suas famílias (\textit{breadwinning mothers}), destaca sentimentos de felicidade destas mulheres por fornecer bons exemplos às suas filhas como característicos do empoderamento, atrelados à construção da identidade de boas mães:

\begin{quote}
    Em outras passagens, esperanças por mudanças futuras em papeis de gênero foram também colocadas como resultado de atos das mães e das identidades como provedoras: ‘Oh, Deus, minhas garotas têm uma figura paterna tão forte em casa... ... isso vai fazer tanta diferença em suas vidas... ... Você não coloca um preço nisso, quero dizer, eu sei que elas estão conseguindo, serão mulheres fortes porque eu sou uma mulher forte que acredita que elas podem fazer qualquer coisa que quiserem.’ \textbf{forte senso de \textit{self}} como uma mulher provedora é construído como tendo um efeito direto nas vidas das crianças neste excerto. \textbf{Esta passagem pode ser lida como um aspecto empoderador da subjetividade de uma mãe provedora}. (Medved, 2009, p. 151, tradução livre, grifo meu)\footnote{``In other passages, hopes for future gender-role changes were also positioned as the result of mothers’ acts and identities of breadwinning: ‘Oh, gosh, my girls have such a strong father figure at home. . . . That is going to make such a difference in their lives. . . . You just can’t put a price on that. I mean, I know they’re getting, they’re going to be strong girls because I’m a strong mom that believes they can do whatever they want to.’ \textbf{A strong sense of \textit{self}} as a breadwinning mother is constructed as having a direct affect on children’s lives in this excerpt. \textbf{And this passage can be read as an empowering aspect of a breadwinning mother’s subjectivity}.'' (p. 151)}
\end{quote}

Embora em alguns momentos surja na literatura consultada o empoderamento como diretamente relacionado aos sentimentos das mulheres em determinados contextos, muitas vezes as autoras também teciam críticas a esse ``sentimento de empoderamento'' considerado como libertação de práticas culturais do patriarcado. É o que ocorre, por exemplo, em Showden (2009), que discute os ``novos feminismos'' em que se valoriza a reinterpretação de mulheres como não apenas vítimas, mas também como livres e empoderadas, tendo uma visão mais positiva de práticas combatidas como degradantes às mulheres pelos movimentos feministas anteriores e, para isso, enfocando os sentimentos de liberdade experimentados por estas mulheres:

\begin{quote}
    Assim, um problema significativo com a posição pós-feminista sobre poder feminino é que ela confunde uma certa reivindicação de feminilidade com uma declaração feminista sobre agência. \textbf{o uso de batom e minissaias possa parecer empoderador e livremente escolhido, essa liberdade e empoderamento são muitas vezes - pelo menos até certo ponto - ilusórias, dada a incapacidade do indivíduo de controlar a leitura de suas ações}\footnote{``Thus, a significant problem with the postfeminist girl power position is that it confuses a determined reclamation of femininity with a feminist statement on agency. \textbf{While wearing lipstick and miniskirts might feel empowering and freely chosen, such freedom and empowerment are often—at least to some degree—illusory given the individual’s inability to control the reading of her actions}.'' (p. 177)}. (Showden, 2009, p. 177, tradução livre, grifo meu) 
\end{quote}

Tal noção de empoderamento também aparece em Lieu (2013), que discute a percepção de mulheres vietnamitas sobre a participação em concursos de beleza, falando também do estímulo ao consumo aplicado às mulheres e de como este subjaz a lógica neoliberal de liberdade como escolha individual:

\begin{quote}
    Além disso, ideias pessoais sobre liberdade e individualismo dentro da lógica do neoliberalismo podem ser estendidas para discussões de gênero, particularmente na era do pós-feminismo. De acordo com Yvonne Taskler e Diane Negra, a cultura pós-feminista ‘trabalha em parte para incorporar, assumir ou naturalizar aspectos do feminismo; crucialmente, também trabalha para mercantilizar o feminismo através da figura da mulher como consumidora empoderada’. \textbf{Alinhada ao neoliberalismo, a cultura pós-feminista enfatiza a escolha (oportunidades profissionais e educacionais) e a liberdade individual particularmente através do empoderamento físico e sexual}. [...] Enquanto os desfiles eram comercializados como eventos comunitários, os organizadores os promoveram como locais afáveis, onde cada concorrente poderia se sentir bem sobre si mesma enquanto forjava amizades com outras mulheres. No entanto, a realidade dessas competições é que apenas uma mulher pode emergir como a rainha da beleza. A única vencedora seria recompensada com bens materiais, além de ganhar capital simbólico como porta-voz da comunidade. [...] Apesar dessas contradições, tanto os organizadores de concursos quanto os próprios participantes empregaram a linguagem neoliberal de escolha, oportunidade e empoderamento feminino para defender o processo competitivo pelo qual os corpos femininos seriam exibidos e julgados. Em consonância com a retórica pós-feminista, os atos coletivos de objetificação e mercantilização não foram contestados, pois as mulheres jovens ‘escolheram’ entrar nos concursos com grandes esperanças de um resultado de sucesso. (Lieu, 2013, pp. 28-29, tradução livre, grifo meu) \footnote{``Moreover, personal ideas about freedom and individualism within the logic of neoliberalism can be extended to discussions of gender particularly in the era of postfeminism. According to Yvonne Taskler and Diane Negra, postfeminist culture ‘works in part to incorporate, assume, or naturalize aspects of feminism; crucially, it also works to commodify feminism via the figure of the woman as empowered consumer.’ \textbf{Aligned with neoliberalism, postfeminist culture emphasizes choice (professional and educational opportunities) and individual freedom particularly through physical and sexual empowerment}. […] While the pageants were marketed as communal events, organizers promoted them as affable sites where each contestant could feel good about herself as she forged friendships with other women. However, the reality of these competitions is that only one woman can emerge as the beauty queen. The sole winner would be rewarded with material goods, as well as gain symbolic capital as a spokesperson for the community. […] Despite these contradictions both pageant organizers and contestants themselves deployed the neoliberal language of choice, opportunity, and female empowerment to defend the competitive process whereby female bodies would be displayed and judged. In line with postfeminist rhetoric the collective acts of objectification and commodification went unchallenged as young women ‘chose’ to enter the pageants with high hopes of a successful outcome.'' (pp. 28-29)}
\end{quote}

Sentido semelhante é conferido ao comportamento das mulheres nepalesas participantes de concursos de beleza citadas por Crawford et al. (2008):

\begin{quote}
    As mulheres que participam de concursos de beleza se apresentam como ‘inteligentes, orientadas por metas, independentes, resolutas e comprometidas com o individualismo’ (Banet-Weiser, 1999: 88). \textbf{Ao fazer o que vêem como uma escolha individual e assertiva de participar, elas acreditam que estão ganhando uma oportunidade de aumentar o equilíbrio, a autoconfiança e a autoestima, e se tornarem fortalecidas, independentes, liberadas e 'modernas' (Ahmed Ghosh, 2003)}. (Crawford et. al., 2008, p. 64-65, tradução livre, grifo meu) \footnote{``Women who enter beauty pageants present themselves as ‘intelligent, goal-oriented, independent, feisty, and committed to individualism’ (Banet-Weiser, 1999: 88). \textbf{By making what they see as an individual, assertive choice to participate, they believe they are gaining an opportunity to enhance poise, self-confidence, and self-esteem, and to become empowered, independent, liberated, and ‘modern’} (Ahmed-Ghosh, 2003).'' (p. 64-65)}
\end{quote}

Em trechos como os presentes em Doetsch-Kidder (2012), o empoderamento enquanto sentimento também aparece como parte de um processo de escolha; as mulheres, argumenta-se, podem exercer tal escolha frente aos desafios colocados ao seu gênero:

\begin{quote}
    Culpar não cria mudança. Como Thich Nhat Hanh escreve: ‘Somente o amor e a compreensão podem ajudar as pessoas a mudarem’. Anzaldua articula o vínculo criado culpando os outros pela opressão: ‘Bloqueados, imobilizados, não podemos avançar, não podemos nos mover para trás. Nos renegamos’. \textbf{aponta a escolha para nós: nos sentirmos empoderadas ou nos sentirmos vitimadas}. [...] Dizer que alguém pode escolher ser empoderada não nega a realidade da opressão estrutural e outros limites ao que se pode fazer no mundo. Keating chama isso de ‘o paradoxo da agência pessoal e da determinação estrutural’ e observa que Anzaldua escreve de dentro dessa contradição, declarando sua incapacidade de resolvê-la. \textbf{A escolha de se sentir empoderada é expansiva, a localização da criatividade e uma fonte de amor. Por meio de críticas amorosas, aceitamos a responsabilidade por nosso papel nos conflitos e nosso poder de responder, construir, criar e transformar. Nós nos recusamos a circunscrever pessoas como vítimas e opressores}. (Doetsch-Kidder, 2012, p. 457, tradução livre, grifo meu)\footnote{``Blaming does not create change. As Thich Nhat Hanh writes, ‘Only love and understanding can help people change.’ Anzaldua articulates the bind created by blaming others for oppression: ‘Blocked, immobilized, we can't move forward, can't move backwards. We abnegate.’ \textbf{points out the choice that we all have to feel empowered or to feel victimized}. [...] To say that one can choose to be empowered does not deny the reality of structural oppression and other limits to what one can do in the world Keating calls this ‘the paradox of personal agency and structural determinacy’ and notes that Anzaldua writes from within this contradiction, declaring her inability to resolve it. \textbf{The choice to feel empowered is expansive, the location of creativity, and a source of love. Through loving criticism, we accept responsibility for our role in conflicts and our power to respond, construct, create, and transform. We refuse to circumscribe people as victims and oppressors}.'' (p. 457)}
\end{quote}

O empoderamento como algo libertador a partir dos sentimentos das mulheres aparece também em Bowman (2013), que examinou a relação das mulheres com a prática da masturbação. Ao refletir que o comportamento de uma mulher de masturbar-se é malvisto na sociedade – em outras palavras, punido ou passível de punição –, a autora coloca os sentimentos de poder das mulheres como característicos da libertação que a prática traria, mesmo reconhecendo que a definição de empoderamento é motivo de discussão na literatura feminista:

\begin{quote}
    É importante definir precisamente o construto do empoderamento sexual, especialmente considerando que as teóricas feministas continuam a lutar com sua definição (Lamb, 2010; Lamb \& Peterson, 2011; McClelland \& Fine, 2008; Peterson, 2010; Tolman, 2012). O empoderamento é melhor entendido como uma experiência interna de agência e poder (isto é, sentir ou experimentar empoderamento)? Ou é uma medida concreta da capacidade ou poder de uma pessoa para alterar os arranjos sociais e políticos (isto é, ter poder)? Alguns teóricos distinguem ``poder para'' (um senso interno de autoeficácia ou autoestima) de ``poder sobre'' (controle real sobre a tomada de decisões e recursos; Hollander \& Offermann, 1990; Riger, 1993; Yoder\& Kahn, 1992), enquanto outros simplesmente se referiram ao primeiro como subjetivo e o segundo como empoderamento objetivo (Peterson, 2010). \textbf{Embora continue a haver uma falta de consenso sobre qual forma de poder constitui o empoderamento sexual, meu estudo tentou compreender as próprias experiências das mulheres sobre o fortalecimento sexual como um aspecto da masturbação, independentemente de essas crenças se traduzirem em mudanças observáveis nas relações de poder}. (Bowman, 2013, p. 364, tradução livre, grifo meu)\footnote{``It is important to precisely define the construct of sexual empowerment, especially considering that feminist theorists continue to struggle with its definition (Lamb, 2010; Lamb \& Peterson, 2011; McClelland \& Fine, 2008; Peterson, 2010; Tolman, 2012). Is empowerment best understood as an internal experience of agency and power (i.e., feeling or experiencing empowerment)? Or is it a concrete measure of a person’s ability or power to alter social and political arrangements (i.e., being empowered)? Some theorists distinguish ‘power to’ (an internal sense of self-efficacy or selfesteem) from ‘power over’ (actual control over decision making and resources; Hollander \& Offermann, 1990; Riger, 1993; Yoder \& Kahn, 1992), whereas others have simply referred to the former as subjective and the latter as objective empowerment (Peterson, 2010). \textbf{Though there continues to be a lack of consensus regarding which form of power constitutes sexual empowerment, my study attempted to understand women’s own experiences of sexual empowerment as an aspect of masturbation, regardless of whether these beliefs translate to observable shifts in power relations}.'' (p. 364)}
\end{quote}

Outros trechos, como em Godbee \& Novotny (2013), relacionam o empoderamento à construção de autoestima e autoconfiança, diretamente ligados ao conceito feminista de \textit{agência}, que discutirei adiante. Neste artigo, as autoras falam do empoderamento que ocorre em turmas de escrita para mulheres, baseadas em coorientação:

\begin{quote}
    Os momentos poderosos que vemos, então, não são efêmeros, mas duradouros. Quando Charisse [entrevistada na pesquisa] aproveita a experiência da sala de aula de Andrea [outra entrevistada], cria, em colaboração, um registro da conversa e das alegações que surgem, e as anotações subsequentes se tornam o roteiro que leva Andrea ao longo de seu processo de escrita. \textbf{Além disso, a escrita serve como um meio de autoempoderamento associado à confiança construída e agência afirmada durante a conferência. [...] Se concordamos que a co-orientação feminista desempenha um papel importante na promoção do senso de valor (isto é, autoempoderamento, agência, solidariedade)}, então indivíduos podem reconhecê-la como importante para si e para outras na academia e dedicar tempo a ela (mesmo incluindo-a em outras tarefas que demandam tempo), ao invés de ser afastados por todas as outras demandas.'' (Godbee \& Novotny, 2013, pp. 190-191, tradução livre, grifo meu) \footnote{``The powerful moments we see, then, are not ephemeral, but lasting. When Charisse taps into Andrea’s classroom experience, they collaboratively create a record of the conversation and claims that arise, and the subsequent notes become the roadmap leading Andrea through her writing process. \textbf{Further, the writing serves as a means of self-empowerment associated with the confidence built and agency asserted during the conference. [...] If we agree that feminist comentoring plays an important role in fostering one’s sense of value (i.e., self-empowerment, agency, solidarity)}, then individuals can recognize it as important to their own and others’ positions in academia and put time toward it (even folding it into other time-demanding tasks), rather than being pulled away by all the other demands on time.'' (pp. 190-191)}
\end{quote}

Outros trechos encontrados na literatura consultada caracterizam o empoderamento como um fenômeno relacionado à construção de consciência - ou seja, à descrição das mulheres das variáveis que afetavam seu comportamento e à possibilidade de ação que tal descrição poderia acarretar. Um exemplo dessa caracterização é encontrado em Maneschy, Siqueira e Álvares (2012), ao descrever o comportamento classificado como empoderado das mulheres participantes de uma associação de pescadoras brasileira:

\begin{quote}
    No Brasil, a Articulação Nacional de Pescadoras é um grande exemplo. É notável em um ramo que, conforme as representações convencionais e hegemônicas, é associado aos pescadores, hábeis e corajosos homens a enfrentar o mar distante e seus perigos. \textbf{Desse modo, as pescadoras em movimento criam suas próprias versões de empoderamento e conscientizam-se de sua presença objetiva em curso no processo da pesca, desestabilizando noções como as de que são ``ajudantes'' ou ``dependentes''; enfim, de que elas não estão nesse setor em suas próprias capacidades}. (Maneschy, Siqueira e Álvares, 2012, p. 724, grifo meu)
\end{quote}

A relação entre agência e conscientização é explorada por Ruiz (1998) como um dos pontos de tensão das teorias feministas que poderia ser discutida à luz do behaviorismo radical. A autora argumenta que, embora as feministas considerem o comportamento humano como produto de um contexto, não refletindo quaisquer essências do gênero masculino ou feminino, alguns internalismos persistem na discussão das instâncias ``causadoras'' do comportamento. A agência, conforme descrita pelo feminismo, é vista como o ``agir com consciência'' – ou seja, a partir da discriminação das variáveis que afetam o comportamento, uma pessoa pode mudar o curso de suas ações, agindo de forma diferente. No entanto, a forma como tal processo é descrito tende a obscurecer as variáveis em questão, uma vez que coloca o \textit{locus} da ação no indivíduo, sem descrever que variáveis contribuem para a mulher em questão passar a descrever as contingências que a afetam. No caso dos trechos discutidos acima, pode-se questionar de que maneira as mulheres aprenderam a descrever seus comportamentos como empoderados a partir do que sentem. A crítica de Lieu (2013) à lógica neoliberal, por exemplo, sugere que no contexto em que as mulheres aprendem que a escolha livre está ligada ao consumo de ideais de beleza, os sentimentos que elas expressam são tomados como característicos dessa escolha.

Os sentimentos também são tomados como fonte do empoderamento descrito por essas mulheres, como surge em Bowman (2013), citada acima. Tomar sentimentos como característicos da liberdade experimentada pelas mulheres nos remete à discussão de Skinner (1971) sobre o enfoque nos sentimentos para explicar o comportamento dos indivíduos.

\begin{quote}
    Pode-se dizer que algumas teorias tradicionais definem a liberdade como a ausência de controle aversivo, mas a ênfase tem sido em como essa condição é sentida. Pode-se dizer que outras teorias tradicionais definem a liberdade como a condição de uma pessoa quando ela está se comportando sob controle não aversivo, mas a ênfase tem sido em um estado de espírito associado a fazer o que se quer. [...] Uma pessoa escapa ou destrói o poder de um controlador para se sentir livre, e uma vez que ele se sente livre e pode fazer o que ele deseja, nenhuma ação adicional é recomendada e nenhuma é prescrita pela literatura de liberdade, exceto talvez vigilância eterna para que o controle não seja retomado. (Skinner, 1971, pp. 36-37, tradução livre)\footnote{``Some traditional theories could conceivably be said to define freedom as the absence of aversive control, but the emphasis has been on how that condition feels. Other traditional theories could conceivably be said to define freedom as a person's condition when he is behaving under non-aversive control, but the emphasis has been upon a state of mind associated with doing what one wants. [...] A person escapes from or destroys the power of a controller in order to feel free, and once he feels free and can do what he desires, no further action is recommended and none is prescribed by the literature of freedom, except perhaps eternal vigilance lest control be resumed.'' (pp. 36-37)}
\end{quote}

O autor chama a atenção para a possibilidade de exercer controle sobre os indivíduos mesmo quando estes não se sentem controlados. Isso diminui a probabilidade do contracontrole e permite que controladores utilizem menos os métodos de controle aversivo, modificando esquemas de reforçamento para que os controlados façam mais com um menor emprego da força. Partindo daí, os sentimentos dos controlados são pouco confiáveis como critério para julgar a liberdade que os mesmos têm naquele contexto. Skinner (1974) lembra ainda que sentimentos positivos surgidos em relações de exploração não percebidas pelos controlados tendem a manter tais relações intactas:

\begin{quote}
    O fato importante não é que nos sentimos livres quando somos positivamente reforçados, mas que não tendemos a escapar ou contra-atacar. Sentir-se livre é uma característica importante de um tipo de controle que se distingue pelo fato de não gerar contracontrole. A luta pela liberdade parece se mover em direção a um mundo em que as pessoas fazem o que gostam ou o que querem fazer, em que gozam do direito de serem deixadas sozinhas, na qual foram ‘redimidas da tirania de deuses e governos. Pelo crescimento de seu livre arbítrio em perfeita força e autoconfiança.’ [...] É um mundo no qual o controle do comportamento humano é errado, no qual ‘o desejo de mudar outra pessoa é essencialmente hostil’. Infelizmente a sensação de estar livre não é uma indicação confiável de que chegamos a esse mundo. (Skinner, 1974, pp. 77-78, tradução livre)\footnote{``The important fact is not that we feel free when we have been positively reinforced but that we do not tend to escape or counterattack. Feeling free is an important hallmark of a kind of control distinguished by the fact that it does not breed countercontrol. The struggle for freedom has seemed to move toward a world in which people do as they like or what they want to do, in which they enjoy the right to be left alone, in which they have been ‘redeemed from the tyranny of gods and governments by the growth of their free will into perfect strength and self-confidence.’ [...] It is a world in which the control of human behavior is wrong, in which ‘the desire to change another person is essentially hostile.’ Unfortunately the feeling of being free is not a reliable indication that we have reached such a world.'' (pp. 77-78)}
\end{quote}

Por outro lado, Skinner (1953) também destaca que os sentimentos podem ser pistas importantes das contingências que estão em vigor, sendo evocados sentimentos positivos (felicidade, júbilo, alívio) quando há reforçamento, seja ele positivo ou negativo, e sentimentos negativos (raiva, agressividade, frustração) quando há punição, seja ela positiva ou negativa. Isso nos permite deduzir que as mulheres que se sentem empoderadas, segundo a literatura consultada, estão emitindo comportamentos que são reforçados pelas pessoas do seu convívio – outras mulheres e outros homens. Pensar que mulheres que se engajam em atividades de cuidado estético, como nos casos citados referentes a concursos de beleza ou cuidado dos filhos, estejam recebendo reforçamento social por tais atividades, seja em forma de elogios, afeto, ou mesmo por evitar a aversividade ligada ao não envolvimento com práticas culturais tidas como femininas, torna possível enxergar tais sentimentos como parte de contingências onde a sensação de felicidade ou alívio é uma descrição genuína.

O problema surge a partir da possibilidade de existirem consequências aversivas atrasadas empregadas no controle. Skinner (1971) discute tais consequências atrasadas, ou armadilhas do controle, a partir dos esquemas de reforçamento empregados pelos controladores. Nos casos retratados acima, há que se observar a possibilidade de consequências atrasadas, por exemplo, nas situações em que o empoderamento tem relação com aspectos físicos ou sexuais. Mulheres que se comportam de forma a conformar-se a idais de beleza, sensualidade e/ou maternidade – exercendo tarefas em que se embelezam, comportam-se de forma a estar disponíveis sexualmente ou dedicam-se aos cuidados com a prole – estão se adequando às práticas culturais esperadas para as mulheres, e caso tais comportamentos não sejam emitidos, as mulheres são punidas por não serem consideradas bonitas ou boas o suficiente como mães. O adequar-se a práticas culturais misóginas que geram reforçadores imediatos mantém tais comportamentos no repertório feminino, mas não contribui para a mudança da hierarquia entre gêneros a longo prazo.

Outros questionamentos podem ser levantados a partir da relação entre consciência e empoderamento, destacada em alguns dos trechos aqui apresentados. A discriminação e descrição verbal de variáveis presentes no ambiente, que caracterizaria a consciência em termos analítico-comportamentais, é um repertório comportamental adquirido e requer ser explicado, enquanto tal, de acordo com as suas próprias origens. A consciência das variáveis relacionadas a práticas culturais decorrentes da misoginia pode ser considerada parte do desenvolvimento das estratégias de mudança e da detecção da necessidade de aprendizagem de novos repertórios comportamentais, e, mais ainda, da intervenção necessária por parte de planejadores culturais, o que denota a importância do desenvolvimento de comunidades verbais que ajudem as mulheres a descrever tais fatores, conforme salientado por Ruiz (1998).

A autoconsciência, colocada como parte do processo de empoderamento nos trechos citados acima, como no relato anterior de Maneschy, Siqueira e Álvares (2012) sobre a articulação de pescadoras, pode surgir do contato com comunidades verbais que descrevam as variáveis que afetam o comportamento das mulheres – o que pode ajudar a explicar por que, entre mulheres, comportamentos que antes eram vistos como típicos de homens sejam, então, caracterizados como empoderados. No entanto, apenas a descrição verbal de variáveis não necessariamente altera outros repertórios comportamentais dessas mulheres. Carvalho Neto, Alves e Baptista (2007) discutem a consciência como fator de prevenção e cuidado contra a violência, destacando os problemas de colocar a consciência como necessária para a mudança:

\begin{quote}
    De um ponto de vista analítico-comportamental, a consciência não seria um determinante autônomo interno da ação dos indivíduos. Não poderia ser a causa do aumento da violência (ausência) e nem de sua solução (presença). Mais do que isso, a consciência estaria entre os produtos da própria violência, entendida como uma forma de interação coercitiva entre o indivíduo e a sociedade. Seria apenas mais um dos repertórios de esquiva (autocontrole) gerados por certas práticas culturais (Sidman, 1989/1995 e Skinner, 1957/1992). Poderia ser entendida também como "ser capaz de descrever o que se está fazendo" e "porquê", identificando as variáveis de controle (racionalidade) (Skinner, 1974/1976). Nesse sentido, a consciência seria um repertório comportamental a ser explicado e não uma explicação última para as demais ações. [...] a responsabilidade pelos problemas acabaria por recair sobre as próprias pessoas mais diretamente afetadas por eles, ou melhor, recairiam sobre as capacidades internas inferidas que supostamente estariam ausentes nessas pessoas. (Carvalho Neto, Alves e Baptista, 2007, pp. 39-40)
\end{quote}

Assim como não poderíamos ver a consciência como solução para a questão da violência em um indivíduo, também podemos pensar que o empoderamento, visto de acordo com os critérios sugeridos por Sardenberg (2012) citados na primeira seção deste capítulo, não se esgota na construção do repertório comportamental da consciência. Este é parte de uma relação que envolve a transformação do poder, ou seja, da mudança de posição entre controladores e controlados. As noções encontradas na literatura classificadas de acordo com a categoria de empoderamento baseado no contracontrole permitirão uma análise mais aprofundada dessa transformação, a seguir.

\section{Empoderamento com base no contracontrole}

Na literatura consultada, o \textit{empoderamento com base no contracontrole} em geral se relaciona a descrições de alterações nas contingências vigentes desvantajosas para as mulheres - seja em termos do reforçamento ou punição recebida naqueles contextos por comportamentos emitidos por elas, seja em relação à percepção do controle por práticas culturais decorrentes da misoginia pelas mulheres e da descrição de tais práticas de controle como objeto de posterior intervenção pelos grupos estudados, com o intuito de modificá-las.

O contracontrole é citado por Skinner (1974) como a resposta dos organismos ao controle poderoso exercido principalmente por uma agência de controle ou por outros indivíduos. Este controle pode ser exercido ao dispor contingências imediatamente aversivas ou de exploração ao longo do tempo. Quando indivíduos respondem de forma a atacar ou modificar as estruturas do controle, seja por meio de ação organizada como protestos, greves ou revoluções ou nas ações contra um membro controlador da sociedade, exercitam o contracontrole. Sá (2016) refina a definição de Skinner de contracontrole social ao descrevê-lo como:

\begin{quote}
    qualquer classe de respostas emitidas por indivíduos (isolados ou em grupo) que tenham o efeito de prevenir, eliminar ou atenuar as consequências aversivas e/ou exploratórias (a curto, médio ou longo prazo) produzidas para tais indivíduos por qualquer dada instância de controle social institucionalizada (legal ou consuetudinariamente) ou em vias de institucionalização. (Sá, 2016, pp. 55-56)
\end{quote}

Skinner (1974) pondera que o contracontrole é mais nitidamente visível nas relações sociais em que há emprego de consequências aversivas imediatas. Nas relações em que há consequências atrasadas, frequentemente não ocorre contracontrole, pois, conforme discutido na seção anterior, as contingências que produzem os sentimentos de liberdade surgidos em tais relações diminuem a probabilidade de contracontrolar. Nessas relações de exploração, podemos incluir as condições de desigualdade entre homens e mulheres socialmente construídas, que perduram também pelo emprego de consequências aversivas atrasadas – mulheres que se comportam de acordo com práticas culturais misóginas recebem reforçamento positivo imediato. A relação de poder entre controlados e controladores foi objeto também de uma revisão do conceito de contracontrole promovida por Ricetti \& Dittrich (2016), que, frente à variedade de definições presentes na literatura behaviorista radical, ponderam que, ao considerar a utilidade do conceito na análise de fenômenos sociais, analistas do comportamento devem atentar à modificação de relações de poder na direção de uma distribuição mais igualitária do mesmo.

Dito isto, o empoderamento feminino se configura como uma destas possbilidades de modificação na distribuição de poder entre indivíduos, conforme descrições presentes nos trechos da literatura revisada. A definição de empoderamento como um fenômeno diretamente relacionado à mudança de relações de poder está presente em parte dos textos analisados, a exemplo de Mariano (2008):

\begin{quote}
    A importância da perspectiva de gênero está relacionada à democratização das relações sociais entre homens e mulheres, partindo do entendimento de que estas são relações de poder, conforme Joan Scott (1990), as quais estruturam sistemas de desigualdades sociais. \textbf{Quando orientadas por essa concepção, as proposições de projetos e políticas públicas implicam vislumbrar impactos nessa estrutura de poder, visando, com isso, promover o empoderamento das mulheres, de forma a abalar e superar as relações de subordinação} (Mariano, 2001). (Mariano, 2008, p. 161, grifo meu).
\end{quote}

Outros textos colocam o empoderamento como característico de comportamentos que mudam ou refletem mudanças nas relações de poder, discutindo a mudança nas relações entre indivíduos, a exemplo de Hung (2012), que descreve as alterações no repertório comportamental de mulheres participantes de workshops voltados a recém-divorciadas, algo culturalmente malvisto na China, país onde foi realizado o estudo:

\begin{quote}
    Nos workshops de empoderamento, as mulheres também foram convidadas a compartilhar o que viam como seus pontos fortes. \textbf{Suas listas incluíam a capacidade de lidar com seus relacionamentos com seus ex-maridos e com a família paterna, sobreviver a dificuldades no processo de divórcio e lidar com as emoções negativas envolvidas.} As participantes também foram ajudadas ao longo do processo a perceber suas forças para tomar a decisão de se divorciar e incentivadas a se esforçar para melhorar suas situações registrando-se nas oficinas, aprendendo e ajudando umas às outras. (Hung, 2012, p. 293, tradução livre, grifo meu).\footnote{``In the empowerment workshops, the women were also invited to share what they saw as their strengths. \textbf{Their lists included the ability to handle their relationships with their ex-husbands and the paternal family, to survive hardships in the divorce process, and to be able to handle the negative emotions that were involved}. The participants were also helped throughout the process to realize their strengths in making the decision to divorce and encouraged to make efforts to improve their situations by registering with the workshops and learning from and helping each other.'' (p. 293)}
\end{quote}

É interessante notar que, mesmo discutindo os sentimentos das mulheres em relação ao divórcio, algo importante nas contingências em que as mulheres se encontravam, o estudo de Hung (2012) valorizou as estratégias pelas quais as mulheres aprenderam a lidar com tais sentimentos, desde a participação nos workshops e os esforços para estarem ali presentes até o próprio compartilhamento de experiências pelas mulheres. Tal característica se verifica também em outros textos onde o empoderamento se aproximou da categoria aqui discutida: a menção aos sentimentos das mulheres ocorre, mas a descrição das variáveis que levam a tais sentimentos permite inferir a função de tais sentimentos em uma contingência de aversividade presente ou de percepção do controle a que se submetem as mulheres nos contextos apresentados.

A caracterização do empoderamento enquanto processo pelo qual mulheres adquirem repertórios comportamentais que, de outra forma, não gerariam reforçamento dado o gênero a que pertencem é bastante presente em um contexto encontrado com frequência nos artigos consultados neste trabalho: a superação da vulnerabilidade econômica por programas de microcrédito ou transferência de renda a mulheres, citado na seção inicial deste capítulo. O empoderamento relacionado a tal contexto se refere à situação de pobreza em que se encontram mulheres e suas famílias e aos impactos da transferência de recursos financeiros em contextos que afetam mulheres de forma específica – como, por exemplo, a decisão por manter ou não um relacionamento violento com um parceiro ou parceira íntima. No entanto, a transferência de renda por si só não seria capaz de alterar parte dos comportamentos aprendidos por mulheres que se relacionam a práticas culturais machistas. Um exemplo de tal discussão está presente em Krenz, Gilbert \& Mandayam (2014), nos seguintes trechos que mencionam o empoderamento como um componente do microcrédito como transferência de renda:

\begin{quote}
    Desde meados da década de 1980, a preocupação com o empoderamento das mulheres cresceu no campo do desenvolvimento internacional (Batliwala, 2007). \textbf{A estudiosa Naila Kabeer (1999) caracteriza o empoderamento de forma ampla como o ‘processo pelo qual aqueles a quem foi negada a capacidade de fazer escolhas estratégicas de vida adquirem tal habilidade’} (p. 435). (p. 310, grifo meu)
\end{quote}

\begin{quote}
    [...] Embora muitas pesquisas tenham questionado os efeitos empoderadores do microcrédito, a provisão de serviços financeiros para mulheres e famílias pobres tem sido amplamente promovida pelas agências doadoras como um caminho relativamente simples para o empoderamento e a redução da pobreza (Mayoux, 2003). \textbf{A suposta correlação entre o microcrédito e o empoderamento baseia-se na premissa de que, com acesso a recursos financeiros, as mulheres estão melhor equipadas para atender às necessidades práticas, contribuir para os recursos domésticos e desafiar a desigualdade de gênero} (Mayoux, 2003). Em um nível básico, a prática do microcrédito envolve estender pequenos empréstimos a tomadoras pobres que, de outra forma, não teriam acesso ao crédito. Os empréstimos são desembolsados para indivíduos ou grupos, com a expectativa de que o dinheiro será investido em atividades empresariais, gerar renda e oportunidades de emprego, e ajudar a diminuir a pobreza no nível individual e comunitário (Isserles, 2003). (p. 312, grifo meu)
\end{quote}

\begin{quote}
    [...] \textbf{Ao insistir em um modelo de empoderamento baseado em grupos, Annapurna} (grupo participante do programa de microcrédito estudado) \textbf{cria espaços sociais novos e seguros para observação, interação e desenvolvimento pessoal. Espaços que promovem a inclusão de grupos e relações não hierárquicas entre indivíduos pobres e menos pobres parecem ser particularmente bem-sucedidos na melhoria dos sentimentos de autoestima}. As reuniões maiores em toda a organização também servem para tranquilizar as clientes de que não estão sozinhas em suas lutas. (p. 316, grifo meu) 
\end{quote}

\begin{quote}
    [...] Alguns casais provavelmente praticaram uma tomada de decisão mais equitativa, mesmo antes de ingressar em Annapurna. Por várias razões, alguns maridos podem ser mais propensos a apoiar a participação de suas esposas nas atividades de microfinanças. Isso é consistente com a pesquisa de Ahmed (2008), que demonstra a maneira pela qual \textbf{modelos divergentes de masculinidade influenciam as atitudes dos homens em relação ao empoderamento de gênero e a participação de suas esposas em esquemas de microcrédito} (p. 152). (p. 321, grifo meu). (Krenz, Gilbert \& Mandayam, 2014, traduções livres).\footnote{``Since the mid-1980s, concern for women’s empowerment has grown within the international development field (Batliwala, 2007). \textbf{Development scholar Naila Kabeer (1999) characterizes empowerment broadly as the ‘‘process through which those who have been denied the ability to make strategic life choices acquire such an ability’’} (p. 435). (p. 310, grifo meu) [...] Though much research has brought into question the empowering effects of microcredit, the provision of financial services to poor women and families has been widely promoted by donor agencies as a relatively straightforward pathway to empowerment and poverty reduction (Mayoux, 2003). The assumed correlation between microcredit and empowerment is based on the premise that, with access to financial resources, women are better equipped to meet practical needs, contribute to household resources, and challenge gender inequity (Mayoux, 2003). On a basic level, the practice of microcredit involves extending small loans to poor borrowers who otherwise would not be able to access credit. Loans are disbursed to individuals or groups, with the expectation that the money will be invested in entrepreneurial activities, generate income and employment opportunities, and help to lessen poverty on the individual and community level (Isserles, 2003). (p. 312, grifo meu) [...] \textbf{By insisting on a group-based model of empowerment, Annapurna} [grupo participante do programa de microcrédito estudado] \textbf{creates new and safe social spaces for observation, interaction, and personal development. Spaces that foster group inclusion and nonhierarchical relationships between poor and less poor individuals seem to be particularly successful in improving feelings of self-worth.} The larger organization-wide meetings also serve to reassure clients that they are not alone in their struggles. (p. 316, grifo meu) [...] Some couples likely practiced more equitable decision making even before joining Annapurna. For various reasons, some husbands might be more likely to support their wives’ participation in microfinance activities. This is consistent with Ahmed’s (2008) research, which demonstrates the way in which \textbf{divergent models of masculinity influence men’s attitudes toward gender empowerment and their wives’ participation in microcredit schemes} (p. 152).''} 
\end{quote}

Mudanças nos comportamentos dos homens que convivem com mulheres que passam pelo processo de empoderamento, no entanto, não se verificam sempre nos textos consultados, sugerindo que algum nível de controle aversivo ainda é empregado contra as mulheres que passam a agir de modo a desafiar o poder masculino. Tal constatação é vista em Amorim, Fiuza e Pinto (2015), que discutem o empoderamento de trabalhadoras rurais participantes e não participantes de sindicatos:

\begin{quote}
    \textbf{A noção conceitual de empoderamento traz, assim, consigo essa perspectiva de mudança nas relações sociais das mulheres com os homens}. Outros autores também abordam essa possibilidade de estabelecimento de relações conflitivas na família e na comunidade advindas da conquista do empoderamento por parte das mulheres (Antunes, 2006; Cortez \& Souza, 2008). [...] Na pesquisa de Antunes (2006) com o movimento das babaçueiras do Maranhão a autora analisou se ocorreu a transferência do empoderamento coletivo, alcançado na esfera pública, na luta pelo direito ao livre acesso ao babaçu, para o âmbito individual, na esfera privada. Ela constatou em seu trabalho a existência em uma mesma mulher de sua faceta de líder empoderada e de esposa desempoderada, demonstrando que essas mulheres deixaram todo o poder alcançado na esfera coletiva do lado de fora (Amorim, Fiuza \& Pinto, 2015, p. 206, grifo meu).

    [...] Nesse mesmo sentido, Cortez e Souza (2008) apontam o empoderamento de mulheres e a repercussão que isso tem causado nos índices de violência conjugal. \textbf{Os autores destacam as implicações do empoderamento de mulheres em suas relações conjugais. Aspectos como trabalho assalariado, questionamentos sobre a vida sexual e maior participação no âmbito público são sinalizadores do empoderamento das mulheres e se tornam ``ameaçadores'' à tradicional dominação masculina. Dessa forma, os homens tentam proteger sua masculinidade através da violência praticada contra a mulher o que também representa um mecanismo de suprimir manifestações femininas de poder} (p. 207).
\end{quote}

Tais aspectos podem indicar que, embora a educação de mulheres entre mulheres promova uma alteração nas estruturas de poder entre gêneros na sociedade, tal alteração não é completamente pacífica, envolvendo, mais do que a percepção do controle, também o enfrentamento do controle aversivo exercido por parceiros e homens do seu convívio. O aprendizado de repertórios comportamentais nesses casos, portanto, mais do que gerado apenas por reforçamento positivo, pode envolver também o estabelecimento de contingências aversivas, nas quais a educação é exercida como forma de contracontrole.

A noção de que o empoderamento envolve o aprendizado de repertórios comportamentais ligados à educação para a participação em meios dominados por homens também surge nos trechos presentes em Blair et. al. (2011), ao descrever a experiência de um acampamento digital voltado para jovens mulheres:

\begin{quote}
    Em vez de presumir, no entanto, que a única coisa que educadoras feministas precisam fazer para facilitar a alfabetização e o empoderamento resultante é fornecer experiências de aprendizado direcionadas para meninas, reconhecemos que tais experiências focadas podem ajudar os participantes a desenvolver uma compreensão compartilhada das possibilidades e restrições tecnológicas em suas próprias vidas e, então, articular essas experiências através de processos de composição multimodal, de forma a movê-las da posição de usuárias de espaços tecnológicos para planejadoras deles. (p. 47, tradução livre) 

    [...] Projetar tarefas e currículos de maneiras que valorizem a aprendizagem como um processo não só perturba estruturas hierárquicas que privilegiam os produtos finais, mas também amplia as oportunidades de pensar com e através de espaços tecnológicos, \textbf{o que é especialmente importante se procurarmos empoderar meninas e mulheres para se relacionarem com/através de tecnologias que não são mediadas por estereótipos de gênero.''} (p. 57, tradução livre, grifo meu)\footnote{``Rather than presume, however, that the only thing feminist educators need to do to facilitate literacy and resulting empowerment is to provide targeted learning experiences for girls, we recognize that such focused experiences can help participants develop a shared understanding of the technological possibilities and constraints in their own lives and then to articulate those experiences through multimodal composing processes in ways that move them from the position of users of technological spaces to designers of them. (p. 47) […] Designing assignments and curriculum in ways that value learning as a process not only disrupts hierarchical structures that privilege final products but also broadens the opportunities for thinking with and through technological spaces,\textbf{which is especially important if we seek to empower girls and women to form relationships with and through technology that are not mediated by gendered stereotypes}''. (p. 57)}
\end{quote}

Galié (2013), ao tratar de uma população de fazendeiras na Síria, retrata a participação de mulheres em um meio tipicamente masculino e coloca a questão das relações de poder modificadas pelo empoderamento feminino, ao discutir a visão de poder comumente descrita pela literatura feminista:

\begin{quote}
    O empoderamento das mulheres tornou-se um objetivo de desenvolvimento frequentemente citado. No desenvolvimento agrícola, o empoderamento é considerado essencial para que agricultores salvaguardem seus interesses de sustento e a agrobiodiversidade baseada em sementes. O empoderamento também é considerado para permitir que pequenos agricultores de áreas marginais participem da pesquisa como parceiros mais iguais, junto com cientistas, aumentando assim a eficácia da pesquisa agrícola. \textbf{O empoderamento dos agricultores mais marginalizados e das mulheres rurais, em particular, é considerado importante para fornecer aos grupos mais vulneráveis os meios para expressar suas necessidades e desejos e tomar medidas para que possam influenciar o desenvolvimento rural e agrícola para a melhoria da nutrição e comida segura}. O vencedor do Nobel, Amartya Sem, demonstra em seu livro Pobreza e Fome como a fome vem do desempoderamento, da marginalização e da pobreza. (grifo meu, tradução livre, p. 58) 

    [...] Parte da literatura do empoderamento olha para o poder como uma luta entre indivíduos com interesses conflitantes para ganhar o poder que os outros detêm, em um jogo de soma zero. Ao olhar para o ‘co-poder’, outros chamam a atenção para o poder produzido pelos relacionamentos e pela ação coletiva para abordar as preocupações comuns dos grupos. A ação coletiva - a ação voluntária tomada por um grupo para alcançar interesses comuns - foi analisada como uma estratégia poderosa para garantir as necessidades e interesses dos membros do grupo (p. 81, tradução livre).\footnote{``Empowerment of women has become a frequently cited goal of development. In agricultural development, empowerment is considered essential in order for farmers to safeguard their livelihood interests and seed-based agro-biodiversity. Empowerment is also considered to enable small farmers from marginal areas to participate in research as more equal partners alongside scientists, thereby increasing the effectiveness of agricultural research. E\textbf{mpowerment of the most marginal farmers, and rural women in particular, is considered important to provide these most vulnerable groups with the means to voice their needs and desires and to take action so that they can influence rural and agricultural development for the improvement of nutrition and food security}. Nobel Prize winner Amartya Sen demonstrates in his book Poverty and Famines how hunger stems from disempowerment, marginalization, and poverty. (p. 58) [...] Part of the empowerment literature looks at power as a struggle between individuals with conflicting interests to gain the power held by others, in a zero-sum game. By looking at copower, others draw attention to the power produced by relationships and by collective action to address the common concerns of groups. Collective action— the voluntary action taken by a group to achieve common interests— has been analyzed as a powerful strategy for securing the needs and interests of group members.'' (p. 81)}
\end{quote}

Frente aos trechos destacados nesta seção, as considerações tecidas pela literatura feminista sobre o empoderamento da mulher estão relacionadas, em grande parte, à alteração de contextos para que a desigualdade de poder entre homens e mulheres seja mitigada. Tal alteração depende tanto do aprendizado de repertórios de contracontrole frente a situações aversivas como do aprendizado referente a situações em que as mulheres não participam de forma comum nas culturas a que pertencem, principalmente por serem mulheres.

\section{Considerações finais}

Frente à caracterização do empoderamento levantada nas seções anteriores, podemos estabelecer uma tentativa de definição comportamental do empoderamento da mulher. Podemos afirmar que o empoderamento feminino é o processo pelo qual as mulheres adquirem novos repertórios comportamentais que, de alguma forma, se relacionam à mudança de contextos aversivos dependentes do seu gênero. Tal aprendizado se dá, especialmente, entre outras mulheres, que, de acordo com suas próprias histórias de vida, dispõem reforçadores para o comportamento das mulheres que aprendem outras formas de se comportar, que não estejam de acordo com as práticas culturais machistas prescritas pela sociedade em geral.

Como parte das contingências em que mulheres passam a obter reforçadores anteriormente não acessíveis, podem ocorrer sentimentos de prazer, felicidade, etc., paralelamente ao poder adquirido. No entanto, conforme apontado, este é apenas um dos efeitos do reforço positivo, ao lado do efeito fortalecedor da resposta - e ele, por si só, não caracteriza um ganho de poder. Nada impede que se atente aos sentimentos das mulheres como parte das contingências promotoras do empoderamento, já que estes são parte do processo e irão ocorrer junto às mudanças nas relações de poder, incluindo as contingências de contracontrole; este, porém, não deve ser um critério único, que prescinda de um cuidadoso exame das modificações nas práticas culturais estabelecidas entre homens e mulheres.

É importante também notar que o empoderamento feminino se relacionou a uma grande variedade de contextos referentes às mulheres na literatura pesquisada. Os artigos estudados se referiram ao empoderamento relacionado à modificação de contextos em que as mulheres passaram por violências e traumas diversos (Lewinson, Thomas \& White, 2014), situações de vulnerabilidade física, psicológica e/ou material (Leitão-Martins, 2006; Krenz, Gilbert \& Mandrayam, 2014; Kim, 2012), participação na política institucional ou na formação de coletivos de mobilização (Melo, 2011; Rai, 2007; Gulbrandsen \& Walsh, 2012), dentre outros. A superação de eventos aversivos e/ou de práticas culturais referentes às mulheres não prescinde de descrever quais são estes eventos e práticas culturais e como eles se aplicam às mulheres na sua totalidade ou a subgrupos dentre as mulheres. Para planejar intervenções referentes a contextos onde o empoderamento feminino é importante, torna-se necessário ter clareza de qual contexto precisa ser superado e como ele se relaciona com aquele subgrupo de mulheres e com as mulheres enquanto categoria.

O estudo aqui discutido traz limitações acerca do método empregado, no entanto, já que a grande variedade entre os contextos apresentados pela literatura analisada dificulta uma análise mais aprofundada de cada um e das propostas possíveis a partir da Análise do Comportamento, levando em conta suas especificidades. As categorias propostas se prestam a uma reunião de tais contextos segundo critérios funcionais, no entanto, não pretendem esgotar a discussão sobre formas de entender o empoderamento em termos dos diversos contextos em que este é citado, visto que podem existir tantos quantos contextos existem. Além desta limitação, a análise do comportamento verbal proposta neste estudo contém as limitações inerentes à dedução das variáveis das quais este tipo de comportamento é função: sem acesso ao ambiente em que tais comportamentos são emitidos, resta especular as condições das quais o comportamento verbal registrado nos textos analisados é função. Outra dificuldade encontrada se refere à variabilidade entre as matrizes epistemológicas presentes nos artigos consultados como objeto principal deste trabalho; as asserções sobre o empoderamento feminino, em geral, subjaziam noções filosóficas diferentes sobre o entendimento do que é a mulher e das formas com que o feminismo luta para emancipá-la.

Não obstante, as categorias aqui discutidas do pensamento sobre o empoderamento permitem aproximações entre o entendimento da filosofia behaviorista radical e das teorias feministas acerca do comportamento dos seres humanos e, principalmente, buscam fomentar as possibilidades de diálogo entre as duas áreas do conhecimento. A aproximação com a literatura feminista aqui apresentada pode ser útil para a comunidade analítico-comportamental, uma vez que sumariza aspectos importantes do que é considerado o empoderamento feminino pelo movimento feminista e dos recursos e limitações do conceito para o planejamento de intervenções comportamentais que tenham como objetivo contribuir para a igualdade entre homens e mulheres. Também permite, em última análise, aproximar a comunidade analítico-comportamental de um campo de conhecimento já há muito em construção e que, vista a tradição da ciência de ignorar as contribuições de mulheres e a denúncia realizada pelo feminismo dessa prática (c. f. Schiebinger, 2001), bem como o isolamento da análise do comportamento em relação a outras áreas, não deve aqui ser repetida.

\section*{Referências Bibliográficas}\sectionmark{Referências Bibliográficas}

\hangindent=25pt
\hangafter=1
\noindent Addae, J. (2015). Effect of microfinance on women's empowerment: A review of the literature. Africa Development and Resources Research Institute (ADRRI) Journal, 13, 1-15.

\hangindent=25pt
\hangafter=1
\noindent Amorim, E. O., Fiuza, A. L. de C., \& Pinto, N. M. de A. (2015). Mulher e trabalho no meio rural: Como alcançar o empoderamento? Caderno Espaço Feminino, 28, 195-213.

\hangindent=25pt
\hangafter=1
\noindent Baquero, R. V. A. (2012). Empoderamento: Instrumento de emancipação social? – Uma discussão conceitual. Revista Debates, 6, 173-187.

\hangindent=25pt
\hangafter=1
\noindent Blair, K., Fredlund, K., Hauman, K., Hurford, E., Kastner, S., \& Witte, A. (2011). Cyberfeminists at play: Lessons on literacy and activism from a girls’ computer camp. Feminist Teacher, 22, 43-59.

\hangindent=25pt
\hangafter=1
\noindent Bowman, C. P. (2013). Women’s masturbation: Experiences of sexual empowerment in a primarily sex-positive sample. Psychology of Women Quarterly, 38, 363-378.

\hangindent=25pt
\hangafter=1
\noindent Carvalho Neto, M. B. de, Alves, A. C. A., \& Baptista, M. Q. G. (2007). A ``consciência'' como um suposto antídoto para a violência. Revista Brasileira de Terapia Comportamental e Cognitiva, 9, 27-44.

\hangindent=25pt
\hangafter=1
\noindent Couto, A. G. (2017). Uma análise behaviorista radical da discussão feminista sobre o empoderamento da mulher (Dissertação de mestrado). Universidade Federal do Paraná – UFPR, Curitiba, PR, Brasil.

\hangindent=25pt
\hangafter=1
\noindent Crawford, M., Kerwin, G., Gurung, A., Khati, D., Jha, P., \& Regmi, A. C. (2008). Globalizing beauty: Attitudes toward beauty pageants among Nepali women. Feminism \& Psychology, 18, 61–86.

\hangindent=25pt
\hangafter=1
\noindent Doetsch-Kidder, S. (2012). Loving criticism: A spiritual philosophy of social change. Feminist Studies, 38, 444-473.

\hangindent=25pt
\hangafter=1
\noindent Galié, A. (2013). Empowering women farmers: The case of participatory plant breeding in tem Syrian households. Frontiers: A Journal of Women’s Studies, 34, 58-92.

\hangindent=25pt
\hangafter=1
\noindent Gill, R. (2008). Empowerment/sexism: Figuring female sexual agency in contemporary advertising. Feminism \& Psychology, 18, 35-60.

\hangindent=25pt
\hangafter=1
\noindent Godbee, B. \& Novotny, J. C. (2014). Asserting the right to belong: Feminist co-mentoring among graduate student women. Feminist Teacher, 23, 177-195.

\hangindent=25pt
\hangafter=1
\noindent Gulbrandsen, C. L., \& Walsh, C. A. (2012). It starts with me: Women mediate power within feminist activism. Affilia: Journal of Women and Social Work, 27, 275-288.

\hangindent=25pt
\hangafter=1
\noindent Hung, S. L. (2012). An empowerment model on reconstituting the meanings of divorce. Affilia: Journal of Women and Social Work, 27, 289-299.

\hangindent=25pt
\hangafter=1
\noindent Kim, S. M. (2012). Evaluations of women-centered U.S. microenterprise development programs. Affilia: Journal of Women and Social Work, 27, 71-83.

\hangindent=25pt
\hangafter=1
\noindent Krenz, K., Gilbert, D. K., \& Mandayam, G. (2014). Exploring women’s empowerment through ``Credit-Plus'' microfinance in India. Affilia: Journal of Women and Social Work, 29, 310-325.

\hangindent=25pt
\hangafter=1
\noindent Leitão-Martins, M. T. de S. (2006). Apesar de... demos a volta por cima: Um estudo sobre o empoderamento de mulheres idosas. Revista Ártemis, 4, s.p.

\hangindent=25pt
\hangafter=1
\noindent Lewinson, T.; Thomas, M. L.; \& White, S. (2014). Traumatic transitions: Homeless women’s narratives of abuse, loss and fear. Affilia: Journal of Women and Social Work, 29(2), 192-205.

\hangindent=25pt
\hangafter=1
\noindent Lieu, N. T. (2013). Beauty queens behaving badly: gender, global competition, and the making of post-refugee neoliberal vietnamese subjects. Frontiers: A Journal of Women’s Studies, 34, 25-57.

\hangindent=25pt
\hangafter=1
\noindent Maneschy, M. C., Siqueira, D., \& Álvares, M. L. M. (2012). Pescadoras: Subordinação de gênero e empoderamento. Revista Estudos Feministas, 20(3), 713-737.

\hangindent=25pt
\hangafter=1
\noindent Mariano, S. A. (2008). Traduções político-culturais de gênero na política de assistência social: Paradoxos e potencialidades para o empoderamento das mulheres no programa Bolsa Família. Revista Gênero, 9, 155-187.

\hangindent=25pt
\hangafter=1
\noindent Mayoux, L. (1999). From access to empowerment: gender issues in micro-finance. CSD NGO Women’s Caucus Position Paper for CSD-8. Recuperado de: \url{https://tinyurl.com/feminismo61}

\hangindent=25pt
\hangafter=1
\noindent Medved, C. E. (2009). Constructing breadwinning-mother identities: Moral, personal, and political positioning. Women’s Studies Quarterly, 37, 140-156.

\hangindent=25pt
\hangafter=1
\noindent Melo, H. P. de. (2011). Uma avaliação do desempenho brasileiro no Global Gender Gap Index do Fórum Econômico Mundial. Caderno Espaço Feminino, 24, 537-552.

\hangindent=25pt
\hangafter=1
\noindent Nasrabadi, M. (2014). ``Women can do anything men can do'': Gender and the affects of solidarity in the U.S. Iranian student movement, 1961–1979. Women’s Studies Quarterly, 42, 127-145.

\hangindent=25pt
\hangafter=1
\noindent Norwood, C. (2014). Women's empowerment and microcredit: A case study from rural Ghana. Journal of Development Studies, 4, 1–22.

\hangindent=25pt
\hangafter=1
\noindent Rai, S. M. (2007). Deliberative democracy and the politics of redistribution: The case of the Indian panchayats. Hypatia: a Journal of Feminist Philosophy, 22(4), 64-80.

\hangindent=25pt
\hangafter=1
\noindent Ricetti, E., \& Dittrich, A. (2016). Definições de contracontrole identificadas na Análise do Comportamento e suas consequências para a análise de fenômenos sociais. (Trabalho de Conclusão de Curso). Universidade Federal do Paraná – UFPR, Curitiba, PR, Brasil.

\hangindent=25pt
\hangafter=1
\noindent Riger, S. (1993). What’s wrong with empowerment. American Journal of Community Psychology, 21, 279-292.

\hangindent=25pt
\hangafter=1
\noindent Ruiz, M. R. (1998). Personal agency in feminist theory: Evicting the illusive dweller. The Behavior Analyst, 21, 179–192.

\hangindent=25pt
\hangafter=1
\noindent Ruiz, M. R. (1998b). Women and welfare reform: How well can we fare without education? Behavior and Social Issues, 8, 153-158.

\hangindent=25pt
\hangafter=1
\noindent Sá, C. P. de. (2016). J. G. Holland, contracontrole social e socialização do behaviorismo radical. Revista Brasileira de Terapia Comportamental e Cognitiva, 18, 52-60.

\hangindent=25pt
\hangafter=1
\noindent Sardenberg, C. (2008). Liberal vs. liberating empowerment: A Latin American 	feminist perspective on conceptualising women's empowerment. IDS Bulletin, 39, 18-27.

\hangindent=25pt
\hangafter=1
\noindent Sardenberg, C. (2012). Conceituando ``empoderamento'' na perspectiva feminista. (Comunicação oral, I Seminário Internacional: Trilhas do Empoderamento de Mulheres – Projeto TEMPO – 2006). Recuperado de: \url{https://tinyurl.com/feminismo60}

\hangindent=25pt
\hangafter=1
\noindent Schiebinger, L. (2001). O feminismo mudou a ciência? Bauru: EDUSC.

\hangindent=25pt
\hangafter=1
\noindent Showden, C. R. (2009). What’s political about the new feminisms? Frontiers: A Journal of Women’s Studies, 30, 166-198.

\hangindent=25pt
\hangafter=1
\noindent Skinner, B. F. (1953). Science and human behavior. Cambridge: The B. F. Skinner Foundation.

\hangindent=25pt
\hangafter=1
\noindent Skinner, B. F. (1957). Verbal behavior. New York: Appleton-Century-Crofts.

\hangindent=25pt
\hangafter=1
\noindent Skinner, B. F. (1971). Beyond freedom and dignity. Middlesex, Reino Unido: Penguin Books. 

\hangindent=25pt
\hangafter=1
\noindent Skinner, B. F. (1974). About behaviorism. New York: Penguin Books.

\hangindent=25pt
\hangafter=1
\noindent Skinner, B. F. (1976). Walden Two. Cambridge: Hackett Publishing. (Trabalho original publicado em 1948).
