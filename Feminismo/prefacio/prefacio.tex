\chapter*{Prefácio}
\begin{center}
    \textbf{\large Debates sobre Feminismo e Análise do Comportamento}\sectionmark{Prefácio: Debates sobre Feminismo e Análise do Comportamento}
\end{center}

Quando pensamos a organização desse livro, na ocasião da 47ª Reunião Anual da Sociedade Brasileira de Psicologia (SBP), em outubro de 2017, não imaginávamos a proporção que ele tomaria. O objetivo era reunir um material voltado para a discussão da desigualdade de gênero na análise do comportamento, especialmente quando a ausência dessa discussão em muitos textos com os quais nos formamos analistas do comportamento mantinha isso invisível. Nesse caminho, descobrimos e redescobrimos muitos trabalhos e pesquisadoras e o resultado foi impressionante. Nós somos muitas! Foi inspirador ler os trabalhos e descobrir como a análise do comportamento pode contribuir para a área de estudos de gênero e, também, se beneficiar dela. Daí se formou o segundo objetivo deste livro: visibilizar essas pesquisadoras e essa produção. Para isso, no primeiro capítulo, as autoras Gabriela Jheniffer Teixeira da Silva e Ana Arantes expõem o processo de invisibilização da mulher na ciência, especificamente na Análise do Comportamento, e resgatam um pouco dessa história representada – ou não representada – na literatura da área.

No segundo capítulo, Táhcita Medrado Mizael escreve sobre algumas formas de articulação entre a análise do comportamento e o feminismo interseccional. No terceiro capítulo, Laís Nicolodi e Ana Arantes operacionalizam os conceitos de poder e patriarcado a partir do referencial analítico-comportamental. Em seguida, no quarto capítulo, Amanda Oliveira de Morais e Júlia Castro de Carvalho Freitas discorrem sobre cultura do estupro e os diferentes métodos de investigação deste fenômeno dentro e fora da psicologia e da análise do comportamento. 

No quinto capítulo, Madeleine Reinert Marcelino e Ana Arantes utilizam o conceito de atitudes implícitas e pesquisas sobre este tema para evidenciar como práticas machistas e opressoras podem ser fortalecidas na história de vida dos indivíduos. No sexto capítulo, Aline Guimarães Couto traz diferentes noções sobre o empoderamento da mulher e os discute sob um ponto de vista analítico-comportamental.

No sétimo capítulo, Júlia Cavalcanti Ferraz, Hellen Luane Silva Peixinho, Christian Vichi e Angelo A. S. Sampaio, utilizam os conceitos de metacontingência, macrocontingência e macrocomportamento para analisar a prática de assédio sexual verbal. No oitavo capítulo, Izadora Ribeiro Perkoski faz uma análise sobre os aspectos culturais e intervenções comportamentais que possibilitem o aumento da participação feminina na área da computação. 

No nono capítulo, Renata da Conceição da Silva Pinheiro e Cláudia Kami Bastos Oshiro apresentam e discutem algumas contingências que comumente aparecem nas demandas de clientes mulheres em processo psicoterápico e que possuem um claro viés de gênero. Seguindo na mesma linha clínica, no décimo capítulo, Analu Ianik Costa discute aspectos teóricos sobre relacionamentos abusivos e possibilidades de intervenção, aplicando-os em um estudo de caso.

Gostaríamos de agradecer, além das autoras, às revisoras e revisores que deram \textit{feedback} nos trabalhos. Acreditamos que esse livro chega num momento muito oportuno, acadêmica e politicamente. O interesse crescente pela área, o desenvolvimento de estudos mais culturais e socioverbais e a expressão de políticas que se mostram contrárias a igualdade de gênero proposta pelo feminismo mostram como o conteúdo deste livro pode ser útil e se tornar uma ferramenta de estudo e de luta. Dessa forma, compartilhamos o conteúdo desse livro com vocês, esperando que eles as e os inspirem como inspiraram a nós. Convidamos vocês também a olharem, a seguir, as fotos e minicurrículos de todas as autoras que juntas construíram esse material conosco. Seguimos juntas.
\vfill