\chapter{Pontes entre o feminismo interseccional e a análise do comportamento}\sectionmark{Pontes entre o feminismo interseccional e a análise do comportamento}
\begin{flushright}
\begin{small}
    Táhcita Medrado Mizael    
\end{small}
\vspace{1cm}
\end{flushright}

O feminismo pode ser conceituado, grosso modo, como um movimento de luta pela conquista de direitos iguais entre os homens e as mulheres. Ele costuma ser dividido em três grandes ondas, a primeira da metade do século XIX até os anos 1960, a segunda entre a década de 1960 e os anos 1980, e por fim, a terceira, da década de 1980 até os anos atuais (e.g., Nogueira, 2017). Em resumo, as pautas das lutas na primeira onda envolveram o desejo de emancipação das mulheres, as quais eram dependentes e subordinadas a seus maridos, além da reivindicação de direitos que apenas os homens tinham, como o direito de votar. 

Na segunda onda, com o início da participação das mulheres brancas de classe média e alta no mercado de trabalho, uma pauta bastante presente foi o reconhecimento da opressão sobre as mulheres, especialmente na família nuclear e no trabalho. Em 1949, com a publicação de ``O segundo sexo'', de Simone de Beauvoir, muitas mulheres utilizaram o livro como inspiração para as lutas ao reconhecerem que elas eram consideradas menos cidadãs que seus parceiros homens. Nesse contexto, a frase ``não se nasce mulher, torna-se'' ficou internacionalmente famosa, sendo utilizada como base para a divisão entre o sexo designado no nascimento – vinculado ao aparato biológico e, portanto, considerado natural, dos homens e mulheres – e o gênero (masculino e feminino) – caráter social das diferenças sexuais entre homens e mulheres, algo considerado construído e não natural (e.g. Piscitelli, 2002; Scott, 1995).

Um slogan nessa fase que ficou muito famoso foi ``O pessoal é político'', utilizado para evidenciar que qualquer tipo de violência ocorrida no âmbito familiar não deveria ser ocultado; pelo contrário, as violências deveriam ser expostas para que a vítima pudesse ser ajudada e o perpetrador, responsabilizado. Outras pautas presentes foram a defesa do direito à contracepção e ao aborto, questionamentos envolvendo a sexualidade, como a existência da heterossexualidade compulsória (Rich, 1980/2010) e a objeção ao tratamento das mulheres como objetos na publicidade, na pornografia e também nas artes (e.g., Nogueira, 2017; Schiebinger, 2008). 

O leitor mais crítico pode, desde o início do texto, ter se perguntado sobre quais mulheres e quais homens foram referidos até o momento, uma vez que homens e mulheres negras eram obrigados a trabalhar em condições sub-humanas e sem poder exercer uma série de direitos que as pessoas brancas tinham, como o direito de estudar. Além disso, para uma variedade de pessoas (como as mulheres lésbicas, homens negros, indivíduos portadores de deficiências, pessoas que moram no campo, etc.), seus marcadores sociais (características físicas e/ou simbólicas) as colocam em diferentes posições. Sendo assim, suas reivindicações seriam diferentes, por terem uma vida distinta da maioria das mulheres que ``protagonizaram'' (por terem condições de fazê-lo) os feminismos mais conhecidos. 

Foi na segunda onda também que o feminismo negro surgiu com mais força, denunciando que as demandas e reivindicações existentes até o momento no feminismo eram pautadas na experiência de mulheres brancas, ocidentais e da classe média. Um relato muito importante, ainda no período da primeira onda, foi o de Sojourner Truth, na Convenção dos Direitos das Mulheres ocorrida em 1851, que denunciava naquela época os perigos da essencialização da categoria ``mulher'': 

\begin{quote}
    Aquele homem lá diz que uma mulher precisa ser ajudada ao entrar em carruagens, e levantada sobre as valas, e ficar nos melhores lugares onde quer que vá. Ninguém me ajuda em lugar nenhum! E eu não sou uma mulher? Olhem para mim! Olhem para o meu braço. Eu arei, eu plantei e eu recolhi tudo para os celeiros. E nenhum homem pode me auxiliar. E eu não sou uma mulher? Eu poderia trabalhar tanto e comer tanto quanto qualquer homem (...) e suportar o chicote tão bem quanto! E eu não sou uma mulher? Eu dei à luz a crianças e vi a maior parte delas ser vendida como escravas. E quando eu chorei com o sofrimento de uma mãe, ninguém além de Jesus me ouviu. E eu não sou uma mulher? (Brah \& Phoenix, 2004, p. 77).
\end{quote}

A terceira onda do feminismo é marcada por uma diversidade e pluralidade de pensamentos e posicionamentos, com destaque para 1) o feminismo pós-moderno e pós-estruturalista (e.g., Butler, 2003), que questiona a existência de uma identidade coerente e estável; 2) a presença de posições teóricas que problematizam concepções feministas essencialistas (e.g, Butler, 2001; Louro, 2008) e 3) críticas com relação a separação de sexo e gênero, com o primeiro considerado uma diferença natural e o segundo, artificial/construído (e.g., Nicholson, 2000), uma vez que não existe sexo pré-discursivo (Butler, 2003), além de críticas com relação a concepções monolíticas da subjetividade. Assim, a desconstrução, a diversidade e a fragmentação identitária são características bastante presentes na terceira onda.

Assim, o feminismo, ou melhor, os feminismos não são movimentos monolíticos ou homogêneos. Sua divisão em ondas tampouco é acurada, uma vez que pode dar a impressão errônea de que as ideias surgiram em um mesmo momento e foram substituídas por novas demandas, sem que as demandas antigas subexistissem ou que houvesse pontos discordantes nos discursos das feministas. Além disso, é importante frisar que as demandas reivindicadas nessas três ondas consistiam, sumariamente, das demandas de mulheres brancas da classe média, muitas delas estadunidenses (Piscitelli, 2002; Nogueira, 2017).

\section{Epistemologia feminista e críticas à concepção de ciência moderna}

Pelo menos desde a década de 1970, muitas feministas, especialmente as acadêmicas, começaram a criticar as concepções de ciência correntes. Elas perceberam uma relação entre os conhecimentos científicos produzidos e problemas centrais do movimento, como a subordinação das mulheres em diversos âmbitos. As feministas criticaram fortemente o uso do masculino como universal, a exclusão das mulheres como pesquisadoras e a subordinação destas como objeto de pesquisa, o enviesamento androcêntrico e as lacunas e generalizações nas pesquisas (Schiebinger, 2008). Como exemplo tem-se as pesquisas sobre ansiedade e depressão. A maioria dos estudos sobre modelos animais é feito com ratos machos, e os ensaios clínicos com homens. Os resultados, contudo, são generalizados para toda a população. Além disso, durante muitos anos houve uma naturalização das diferenças entre os sexos como produtos biológicos (e.g., a inteligência, habilidades viso-espaciais e a coordenação motora foram consideradas habilidades que somente os homens tinham, apesar de haver dados mostrando que essas habilidades são aprendidas). No entanto, hoje sabe-se que meninos muitas vezes desenvolvem mais as habilidades de raciocínio viso-espacial e coordenação motora fina, por exemplo, pela exposição a brinquedos os quais as meninas são proibidas e/ou punidas ao brincar (e.g., Porto, 2016; Schiebinger, 2008).

Até mesmo campanhas de saúde veiculadas televisivamente são focadas no masculino. Pense em quais são os sintomas de infarto do miocárdio. Dor no peito, náuseas, suor frio e desmaios, certo? Na verdade, nas mulheres muitos sintomas do infarto do miocárdio são diferentes e envolvem, por exemplo, falta de ar, cansaço inexplicável e arritmia (\textit{National Institute of Health}, 2012). Além disso, a representação do óvulo e espermatozoide como agentes passivo e ativo, respectivamente, em livros didáticos, vídeos e outros tipos de mídia também é incorreta (e.g., Keller, 2006).

Com base nas diversas críticas à concepção de ciência corrente na época, algumas epistemologias feministas foram desenvolvidas, com posições que variavam desde as mais liberais até as mais radicais. Nas posições mais liberais, há certa concordância com os pressupostos da ciência moderna e o foco é em pesquisas sobre questões que dizem mais respeito às mulheres. Assim, as feministas que adotam as posições mais liberais consideram que a ciência é neutra e desinteressada, que conhecedor e conhecido estão separados, que a ciência pode ser objetiva, etc., mas que é necessário focar os estudos nas experiências das mulheres. Por outro lado, as posições mais radicais vão questionar inclusive a objetividade e a racionalidade como bases da metodologia científica, evidenciando que a ciência está imbrincada na política e na ideologia, ou seja, que é impossível um pesquisador ser neutro e seus interesses e/ou pesquisas, desinteressadas. (Nogueira, 2017).

Assim, por exemplo, no empiricismo feminista, parte-se de uma concepção na qual é possível corrigir os vieses androcêntricos e sexistas se as regras da pesquisa científica forem cumpridas à risca. Para as teorias de \textit{standpoint} feministas (e.g., Harding, 1986), por outro lado, além de seguir as normas da pesquisa científica de maneira estrita, é necessário que se pesquisem sobre as mulheres, suas experiências e concepções sobre os acontecimentos. Por fim, uma terceira epistemologia feminista vai advogar a recusa a qualquer discurso universalizante, fomentando a existência de conhecimentos situados, ou seja, que uma pesquisa realizada em determinado local e contexto pode ter (e provavelmente terá) resultados distintos da mesma pesquisa realizada em outro local e/ou contexto. As duas primeiras abordagens, portanto, possuem uma concepção universal e generalizante da mulher, essencializando-a, que é justamente o que a terceira abordagem busca evitar.

\section{Interseccionalidades}

O termo interseccionalidade tem sido utilizado em uma variedade de contextos, sendo considerado um método de pesquisa, uma teoria, uma abordagem, um paradigma, um conceito, uma metáfora analítica, entre outros (e.g., Davis, 2008). Historicamente, o estudo das interseccionalidades foi fortemente influenciado pelo feminismo negro, ou seja, um dos movimentos feministas nos quais as mulheres negras perceberam que a intersecção entre diferentes marcadores sociais, especialmente raça e gênero, dificultava sua identificação com as lutas feministas (principalmente nas décadas de 1960 a 1980) e com as lutas pelos direitos civis. O relato de Falcón (2009) ilustra isso:

\begin{quote}
    As ativistas mulheres de cor dessa época eram frequentemente forçadas a optar por um dos lados entre as lutas feministas e as dos direitos civis. Fazer a escolha era difícil para mulheres de cor porque suas experiências não eram apenas baseadas em raça \textit{ou} gênero, mas em raça e gênero. Frustradas com feministas brancas que fracassaram em integrar o antirracismo em seu ativismo e com homens de cor que fracassaram na luta contra seu sexismo, as mulheres de cor começaram a se organizar e vocalizar suas questões [específicas] (Falcón, 2009, p. 467, itálicos adicionados).
\end{quote}

Assim, o termo \textit{feminismo interseccional} tem sido utilizado para denominar feminismos nos quais as interseccionalidades são levadas em consideração nas análises, ou seja, que ser mulher pode produzir formas de opressão, mas que essa característica não é (ou não deveria ser) considerada a única ou a mais importante forma de opressão. 

O primeiro registro do termo interseccionalidade foi feito pela pesquisadora estadunidense Kimberlé Crenshaw em 1989, para se referir à abordagem que leva em consideração a interação entre diversas formas de subordinação. A Teoria da Interseccionalidade (TI) parte de um questionamento relacionado aos feminismos mais ``tradicionais'', que abordavam, entre outras coisas, as diferenças de gênero, no sentido de masculino/feminino. Nesse sentido, em vez de pesquisar as diferenças entre homens e mulheres, a TI ressalta a importância de se pesquisar também as diferenças entre as próprias mulheres. O objetivo de tal empreitada é reduzir os essencialismos, ao descentrar os discursos dominantes focados, primordialmente, nas diferenças entre homens e mulheres brancos e de classe média de sociedades ocidentais (Henning, 2015; Nogueira, 2017).

Para a TI, todo conhecimento é considerado 1) socialmente construído e 2) parcial e limitado histórica ou politicamente. Além disso, a TI enfatiza a importância de coexistirem uma variedade de posições, indicando, portanto, que os posicionamentos de determinado indivíduo dependem de sua história de vida, dos contextos aos quais ele foi exposto e das experiências que teve. Desse modo, diferentes marcadores sociais são dotados de diferentes valorações em diferentes contextos, o que expõe o caráter relacional de aspectos e características humanas. Na TI, o foco de análise é a interação ou intersecção entre as várias categorias ou identidades a qual uma pessoa pertence (marcadores sociais), como raça, sexo designado no nascimento, identidade de gênero, orientação sexual, deficiências, classe, idade/geração, território, nacionalidade, corporalidade, etc. (Aguião, 2015; Nogueira, 2017; Piscitelli, 2008). Para Crenshaw (2002):

\begin{quote}
    A interseccionalidade é uma conceituação do problema que busca capturar as consequências estruturais e dinâmicas da interação entre dois ou mais eixos de subordinação. Ela trata especificamente da forma pela qual o racismo, o patriarcalismo, a opressão de classe e outros sistemas discriminatórios criam desigualdades básicas que estruturam as posições relativas de mulheres, raças, etnias, classes e outras (p. 177).
\end{quote}

Assim, diferentemente de concepções mais tradicionais, onde predomina o raciocínio de que os marcadores sociais se somam, essa teoria\footnote{De acordo com Adriana Piscitelli (2008), existem duas abordagens predominantes sobre as interseccionalidades: a abordagem sistêmica e a abordagem construcionista. Em termos gerais, na primeira abordagem, há um foco sobre o caráter repressivo dos marcadores sociais na possibilidade de agência (capacidade de identificar relações de controle – geralmente aversivo – e agir, de modo a reduzir o controle aversivo) das mulheres, e o segundo possui uma concepção de dinamismo maior, de modo que a articulação entre os diferentes marcadores sociais pode gerar contextos de opressão, mas também de agência para as mulheres. Este texto utiliza a segunda abordagem do termo.} mostra que a articulação destes marcadores cria condições de opressão e privilégio, dependendo dos contextos onde os indivíduos estão e da própria articulação entre os marcadores. 

Diante da pergunta ``quem tem mais desvantagem: homens ou mulheres?'', a resposta pode variar a depender dos outros marcadores incluídos na análise. São homens e mulheres brancos? Eles vivem na cidade, no campo? São jovens, idosos? E se for um homem negro e uma lésbica branca? Existem pesquisas mostrando, inclusive, que certas palavras que deveriam ser utilizadas para descrever homens e mulheres, como ``\textit{Black}'' (negro), e até palavras que descrevem uma classe de pessoas (``\textit{women}'', mulheres), independente de outros atributos evocam respostas muito especificas. No caso, \textit{Black} evoca a classe ``homens negros'' apenas, e \textit{women}, ``mulheres brancas'' (e.g., Goff, Jackson, Di Leone, Culotta, \& DiTomasso, 2014; Steinbugler, Press, \& Dias, 2006; Warner, 2008). Nas palavras de Warner (2008), ``um estereótipo diferente é eliciado quando gênero e raça são considerados juntos, do que quando raça ou gênero são considerados sozinhos'' (Warner, 2008, p. 457).

\section{É possível uma articulação entre o feminismo (interseccional) e a análise do comportamento?}

O feminismo interseccional pode ser conceituado como a análise das formas de entrelaçamento entre diferentes marcadores sociais contextualizados histórica e culturalmente que podem produzir desigualdades, mas também formas de resistência e/ou privilégios. Ele ``se opõe à ideia de partir de diferenças tidas como relevantes \textit{à priori}'' (Henning, 2015, p. 110)

Pressupostos feministas são coerentes com a abordagem analítico-comportamental e contribuições da análise do comportamento (AC) para os feminismos foram realizadas, principalmente pela pesquisadora Maria Ruiz (Ruiz, 1995, 1998, 2003, 2009; 2013; Ruiz \& Roche, 2007), mas também por outros (e.g., Couto \& Dittrich, 2017; Fideles \& Vandenberghe, 2014; Silva \& Laurenti, 2016). Em seguida, serão apontadas algumas das críticas aos modelos de ciência feitos por feministas, seguidas por explicações analítico-comportamentais (parte da discussão a seguir pode ser vista em alguns dos estudos supracitados): 


\begin{enumerate}
    \item Visão contextual e subjetiva do conhecimento científico: 
\end{enumerate}   
Uma crítica comum no discurso feminista em geral, ou seja, de várias abordagens feministas, como apontado por Ruiz (1995), é a suposição de que o conhecimento científico é objetivo e neutro e de que conhecedor e conhecido estão separados. Desde a década de 1970, alguns feminismos defendem uma ciência na qual o cientista não pode ser separado do objeto a ser conhecido, uma vez que este faz parte do mesmo ambiente. Nesse sentido, o conhecimento científico não pode ser neutro nem objetivo, uma vez que as realidades são construídas socialmente (e.g., Schiebinger, 2008). A AC possui uma visão que concorda (pelo menos parcialmente) com o pressuposto de que todo conhecimento é contextual e subjetivo. Nas palavras de Skinner (1974):

\begin{quote}
    Seria absurdo para o behaviorista afirmar que está de alguma forma isento de sua análise. Ele não pode sair do fluxo causal e observar o comportamento de algum ponto de vista especial [...]. No próprio ato de analisar o comportamento humano, ele está se comportando – como no próprio ato de analisar o pensamento, o filósofo está pensando (p. 234; tradução da autora).
\end{quote}

\begin{enumerate}[resume]
    \item Visão contextualista de mundo
\end{enumerate} 

Outra crítica presente no discurso feminista apontada por Ruiz\linebreak (1995) e especialmente dirigida à psicologia, é que esta é individualizante, uma vez que os problemas psicológicos, apesar de serem pautados em contextos sócio-políticos, são considerados pela psicologia como questões individuais. Exemplos disso são a violência doméstica, onde é comum ver análises da vítima e do agressor sem levar em consideração os contextos mais amplos no qual a agressão ocorreu, o ``indivíduo ansioso'' ou ``depressivo'', diagnosticado e tratado individualmente, sem considerar porque a pessoa desenvolveu tais transtornos; e o preconceito racial, onde o indivíduo ``racista'' é considerado o problema a ser tratado, como se o racismo viesse ou fosse uma propriedade de instâncias internas e não fosse aprendido no ambiente no qual o indivíduo está inserido. 

Diferentemente de algumas abordagens psicológicas consideradas individualizantes, e em concordância com o discurso feminista que\linebreak preza por análises contextuais, a AC possui uma visão contextualista de mundo, onde os contextos históricos e imediatos são fundamentais para a análise. Em uma pesquisa sobre violência doméstica de Bernard Guerin e Marcela Ortolan (2017), por exemplo, os autores apontam que a análise de um ou mais episódios de violência contra a mulher requer não só a análise do comportamento do par, mas também do ambiente mais abrangente. Isto envolve, segundo os autores, os contextos políticos, históricos, sociais e econômicos atuais, os quais oferecem, muitas vezes, condições para que diversos tipos de violência sejam reforçados e/ou naturalizados. 

\begin{enumerate}[resume]
    \item Agência como uma forma de controle recíproco sobre as contingências ambientais
\end{enumerate}

No discurso feminista, agência é um termo utilizado para se referir, grosso modo, à capacidade de identificar relações de controle (geralmente aversivo) e buscar formas de eliminar ou reduzir o controle aversivo. Assim, uma terceira crítica das feministas parte do entendimento de que a agência é algo externo ao indivíduo (e não uma relação organismo-ambiente, como a AC a entende) e que, portanto, é impossível modificar práticas opressivas se ``não tenho'' ou ``não possuo'' agência. Entretanto, para a AC, o conceito de agência diz respeito a um controle recíproco sobre contingências ambientais (Ruiz, 1998), de modo que o ambiente causa mudanças em nosso comportamento, mas nós também modificamos o ambiente: ``os homens agem sobre o mundo, modificam-no e, por sua vez, são modificados pelas consequências de suas ações'' (Skinner, 1957/1978, p. 15). Nesse sentido, é possível que a concepção de agência utilizada nos feminismos seja coerente com a proposta analítico-comportamental, sendo vista como uma forma de autoconhecimento que instrumentaliza o indivíduo para a ação.

Após verificar, portanto, que vários pressupostos dos feminismos são coerentes com a proposta analítico-comportamental, de que maneira nossas ferramentas ou formas de análise podem auxiliar no estudo das diferenças entre homens e mulheres (e entre as próprias mulheres e os próprios homens)?

Ruiz (2003) faz uma análise do que ela chama de práticas culturais generificadas, mostrando que o sexo designado no nascimento dos indivíduos serve como estímulo discriminativo para tais práticas. Assim, por exemplo, temos evidência de que professores de ensino fundamental reforçam classes de respostas de seus alunos diferencialmente, dependendo do sexo da criança: se o indivíduo é um menino, a qualidade do trabalho é reforçada. Se for menina, por outro lado, a aparência do trabalho é reforçada. Tal reforço diferencial é dado em uma série de contextos e com relação a uma variedade de respostas, como a maneira de se sentar, o que é considerado bagunça, que tipos de interesses são reforçados (matemática para os meninos e artes para as meninas), etc. 

Outro exemplo se refere ao comportamento assertivo. Segundo Del Prette e Del Prette (2005), na ``base do conceito de assertividade encontra-se a noção de igualdade de direitos e deveres, de legitimidade dos comportamentos voltados para a reivindicação e defesa desses direitos, de respeito e dignidade da pessoa humana.'' (p.175). Entretanto, é comum ver no dia-a-dia que uma mesma classe de respostas emitidas por homens e mulheres que poderia levar o rótulo de assertividade é considerada assertiva somente quando emitida por homens; as mesmas respostas emitidas por mulheres são consideradas ``agressivas''.

Assim, em resumo, para Ruiz (2003):

\begin{quote}
    Quando nós falamos de práticas culturais generificadas, nós estamos falando de formas de controle social relacionadas ao poder e a relações de dominância que levam diretamente ao nível de acesso que um indivíduo ou grupo de indivíduos podem ter a fontes de reforçamento ou alocação de recursos (p. 15; tradução da autora).
\end{quote}
\vfill
\pagebreak
E o feminismo interseccional? Apesar de várias dessas contribuições serem possíveis em vários tipos de feminismos, de que maneira a AC poderia contribuir especificamente no feminismo interseccional?

\begin{enumerate}
    \item Concepção de que todo conhecimento é socialmente construído:
\end{enumerate}

A TI e o feminismo interseccional adotam a concepção de que as formas de conhecimento são socialmente construídas. Para a AC, a concepção de conhecimento presente nas propostas construcionistas sociais se refere ao ``saber que'' (\textit{knowing that}), ou seja, ``conhecer significa comportar-se com o comportamento verbal apropriado'' (Guerin, 1992/2009, p. 5). Guerin (1992/2009) relaciona a AC e algumas abordagens construcionistas utilizando duas abordagens principais modernas: a de Gergen \& Davis (1985) e a de Moscovici (1988).

Para Gergen e Davis (1985) existem quatro concepções presentes no construcionismo social: 1) a de que ``nossas relações com o mundo nem sempre correspondem ao mundo real'' (Guerin, 1992/2009, p. 3), com a implicação de que mesmo que essas relações e objetos sejam inventados/criados, tais relações podem influenciar o comportamento aberto e encoberto dos indivíduos; 2) a concepção de que a forma com a qual explicamos o mundo (pelo uso da linguagem) também se configura como um produto social; 3) a concepção de que a manutenção de qualquer conhecimento não depende de sua validade empírica, mas de sua relação com o ambiente social e não social; e 4) de que conhecimentos construídos socialmente são inseparáveis de nossas vidas sociais e se relacionam com várias outras atividades sociais. Além disso, para Gergen e Davis (1985), conhecimento não é uma propriedade ou um objeto existente na cabeça das pessoas, mas coisas que as pessoas fazem juntas, o que também é coerente com a proposta analítico-comportamental.

Outros pesquisadores da AC também têm evidenciado pontos de contato entre a AC e o construcionismo social. Roche e Barnes-Holmes (2003), por exemplo, relataram semelhanças (e diferenças) entre a proposta construcionista social e a proposta analítico-comportamental: 1) a natureza do conhecimento: ambas as abordagens consideram que o conhecimento científico possui origem social e seu entendimento se baseia no estudo de práticas da comunidade verbal; 2) a linguagem como peça-chave para a ação humana: a concepção de que a linguagem é uma convenção social e o interesse em análises funcionais da linguagem, e 3) ênfase no contexto e história para analisar os eventos, para citar apenas três. 

A proposta analítico-comportamental compartilha a noção de que as formas pelas quais compreendemos o mundo são o produto do ambiente social ao qual os indivíduos pertencem. Descrições do mundo e do próprio \textit{self}, são, portanto, resultado de um processo de aprendizagem de relações complexas formadas no contato com o mundo, de maneira que nossas concepções sobre o mundo, sobre quem somos e o que sentimos não correspondem a uma realidade objetiva, mas sim a uma série de aprendizagens sociais. 

Do mesmo modo, a linguagem, ou melhor, os significados que damos aos vários aspectos da linguagem (símbolos, gestos, verbalizações, etc.) são determinados e compartilhados por uma comunidade verbal social, ou seja, uma mesma palavra (símbolo ou gesto) pode ter diferentes significados em diferentes culturas, ou ainda dentro de uma mesma cultura (ou comunidade verbal). Assim, é possível considerar a AC como uma abordagem construcionista social que converge com a concepção presente nas abordagens interseccionais (e, consequentemente, no feminismo interseccional) de que conhecimentos são socialmente construídos. 

\begin{enumerate}[resume]
    \item Identificação de relações de controle
\end{enumerate}

A descrição operacional das contingências, prática básica da AC, pode ser muito útil na identificação e descrição acurada das complexas relações envolvidas nos eventos de interesse do universo do feminismo interseccional. Resgatando a pergunta ``quem sofre mais violência no Brasil: Homens ou mulheres?'', possíveis relações de controle seriam: o primeiro indivíduo pode responder sob controle da violência que ocorre nas ruas. A segunda pode responder sob controle da violência doméstica, e uma terceira relação de controle estaria baseada nos dados de violência policial. Portanto, o tipo de violência (física, psicológica, etc.), o contexto no qual a pessoa sofreu a violência, quem a perpetrou, entre outros, são variáveis que podem controlar a emissão da resposta e, portanto, criar diferentes verdades ou concepções sobre o assunto.

A identificação de relações de controle também se relaciona com outro aspecto central da TI e do feminismo interseccional, que é a oposição à ideia de partir de diferenças consideradas relevantes à priori, ou seja, antes da análise. Os contextos vão ``informar'' quais diferenças, marcadores sociais e/ou outros estímulos controlaram as respostas que determinado indivíduo emitiu em determinado evento. 

\begin{enumerate}[resume]
    \item Ênfase no aspecto relacional e na história dos indivíduos
\end{enumerate}

Na seção sobre ``Interseccionalidades'', foi ressaltada a descoberta, em algumas pesquisas, de que certas palavras que são utilizadas para descrever homens e mulheres podem evocar respostas específicas, como no caso de ``\textit{Black}'', geralmente assumido como ``homem negro'', e ``woman'' como ``mulher branca'' (e.g., Warner, 2008). Essa descoberta pode ser analisada, em nosso referencial, a partir de um modelo recente, elaborado no contexto da teoria das molduras relacionais\footnote{A teoria das molduras relacionais é uma abordagem moderna da linguagem e cognição humanas. De acordo com essa teoria, a base da cognição e linguagem humana está na habilidade de aprender diferentes tipos de relações entre estímulos, de modo que a aprendizagem de algumas relações seja suficiente para que um indivíduo derive outras relações, ou seja, aprenda algumas relações sem que seja instruído para fazê-lo. Para saber mais, consulte Hayes et al. (2001).} (\textit{Relational Frame Theory}, RFT, em inglês; Hayes, Barnes-Holmes \& Roche, 2001).

O nome desse modelo é DAARE, \textit{Differential Arbitrarily Applicable Relational Responding Effects} (efeitos do responder relacional diferencial arbitrariamente aplicado) e ele surgiu em um contexto de pesquisa na qual os experimentadores tiveram dificuldade em explicar porque alguns estímulos se relacionavam mais entre si do que outros. Especificamente, em uma tarefa na qual os participantes tinham que relacionar cores com cores, cores com formas e formas com formas, os experimentadores verificaram que os participantes tinham mais facilidade em relacionar ``cor-cor'' do que ``forma-forma'' (Finn, Barnes-Holmes, Hussey, \& Graddy, 2016). O que os pesquisadores hipotetizaram, em um estudo subsequente (Finn, Barnes-Holmes, \& McEnteggart, 2018), é que talvez esses resultados ocorreram porque, na história verbal com essas palavras, a frequência de emissão das palavras utilizadas na pesquisa que se referiam a cores era maior que a frequência de emissão de palavras que denotavam formas. Para este modelo, então, essa frequência diferencial no uso das palavras referentes às cores e às formas evoca mais respostas de orientação às primeiras, em comparação com as segundas, de modo que há uma maior coerência, isto é, uma maior consistência entre o padrão de responder relacional e a história comportamental que deu origem a esse padrão nas relações ``cor-cor'' do que nas relações ``forma-forma''. 

Neste contexto, pesquisas podem se utilizar desse modelo para discutir resultados já encontrados na literatura, por exemplo: sobre vieses raciais e de gênero, na criação de novas pesquisas que investiguem esses efeitos e aplicando-os a questões como estereótipos de gênero e a intersecção gênero-raça. O estudo de Barnes-Holmes, Murphy, Barnes-Holmes e Stewart (2010), por exemplo, investigou a existência de vieses raciais implícitos em participantes adultos. Os participantes tinham que responder a relações consideradas consistentes ou inconsistentes com suas histórias de vida, respondendo verdadeiro ou falso às relações ``homem branco carregando armas-seguro'', ``homem branco carregando armas-perigoso'', ``homem negro carregando armas-seguro'' e ``homem negro carregando armas-perigoso''. Veja que, além da história verbal com esses estímulos, existe também a história do responder a ``verdadeiro'' ou ``falso'', com o primeiro ocorrendo em uma frequência maior que o segundo (a frequência de emissão da resposta ``verdadeiro'' tende a ser maior que a frequência de emissão da resposta ``falso''). 

Nesse estudo, a hipótese era que, de acordo com as respectivas histórias de vidas, os participantes responderiam com uma rapidez semelhante às relações entre homens negros e perigo e a homens brancos e segurança. Entretanto, os resultados mostraram que os participantes tiveram mais facilidade em relacionar ``branco-seguro'' do que ``negro-perigoso''. De acordo com o modelo DAARE, e como interpretado por Barnes-Holmes, Harte e McEnteggart (no prelo), se a foto do homem branco e a palavra ``seguro'' tiverem funções avaliativas positivas para os participantes, ao passo que a foto do homem negro e a palavra ``perigoso'' tivessem funções avaliativas negativas, é possível interpretar tais resultados em termos de maior ou menor coerência entre ``branco-seguro-verdadeiro'' e ``negro-perigoso-verdadeiro'', com o primeiro sendo mais frequente, ou seja, mais coerente com suas histórias verbais. Portanto, tal efeito ``pode ter surgido, em parte, a partir das diferenças na coerência entre os dois tipos de tentativas, ao invés de puramente a partir de respostas racialmente enviesadas'' (Barnes-Holmes et al., no prelo, p. 19).

Uma interpretação possível disso é que relações entre palavras tem o potencial de gerar certos padrões de resposta mais frequentes que outros, como o ``branco-seguro'' ser mais frequente que o ``negro-\linebreak perigoso'' na linguagem dos participantes da pesquisa supracitada. \linebreak Nesse caso, extrapolando a análise, a forma como notícias, novelas, músicas e até piadas são veiculadas/contadas pode auxiliar no estabelecimento e/ou manutenção de relações indesejadas (preconceituosas), reforçando relações já existentes (estereótipos) sobre grupos. Essa possibilidade deve ser examinada experimentalmente, e os estudos na área da equivalência e de pareamento de estímulos (e.g., Amd, de Almeida, de Rose, Silveira, \& Pompermaier, 2017; Barnes, Leader, \& Smeets, 1996; Leader, Barnes, \& Smeets, 1996; Sidman, 1994; Sidman \& Tailby, 1982) mostram que isso é possível.

Com relação às possibilidades de pesquisa, estudos poderiam verificar se esse efeito é demonstrado em pesquisas sobre estereótipos de gênero, utilizando como estímulos, por exemplo, ``homem'', ``mulher'', ``sensível'' e um adjetivo neutro. De acordo com o modelo DAARE, os resultados mostrariam que é mais fácil relacionar ``mulher-sensível-verdadeiro'' do que ``homem-sensível-verdadeiro''. Além disso, poderiam ser delineados estudos que utilizem, como estímulos, compostos como ``homem branco'', ``homem negro'', ``mulher branca'' e ``mulher negra'' e atributos relacionados a um desses grupos, investigando os efeitos da intersecção entre dois ou mais marcadores sociais no responder relacional.

\section{O que analistas do comportamento feministas podem fazer em seus campos de atuação?}

Para finalizar, seguem algumas dicas de atuação que podem ter grandes implicações teóricas e práticas, caso analistas do comportamento se identifiquem com os pressupostos feministas supracitados:

\begin{enumerate}
    \item Explicitar as razões pelas quais um/a pesquisador/a quer investigar ou investigou certo tema de pesquisa, pensando especialmente nas implicações disso em termos de consequências científicas e sociais para o grupo estudado;
    \item Descrever contingências que revelem estruturas e/ou contextos de opressão existentes e como tais estruturas/contextos deixam vários grupos de pessoas à margem. Exemplos seriam descrições sobre como a aceitação de comportamentos como o ciúme pode ter a função de controlar a parceira; efeitos negativos de fomentar uma cultura em que os homens devem cuidar das finanças do casal e etc. (o trabalho de Guerin \& Ortolan, 2017 é um ótimo exemplo disso);
    \item Promover práticas inclusivas (aspecto interseccional importantíssimo) pensando nos mais diversos tipos de público que tenham interesse e/ou necessidade nos serviços prestados pelos analistas do comportamento. Assim, ao oferecer um curso sobre ``como lidar com o luto'', por exemplo, o profissional provavelmente vai fazer recomendações que são voltadas para um determinado tipo de público (e.g., uma pessoa com condições de pagar um psicólogo, e/ou com uma rede de apoio disponível). Nesse sentido, é importante que o profissional tente pensar em outras populações (e.g., pessoas pobres, indivíduo que não possui rede de apoio, etc.) para que as recomendações sejam mais abrangentes (quando necessário) e, principalmente, que o profissional seja humilde a ponto de revelar desconhecimento sobre determinados temas, mas que, ao mesmo tempo, se mostre disponível para auxiliar em uma demanda que é da sua área, mas se refere a condições ainda não pensadas;
    \item Analisar e criar estratégias eficazes de contracontrole: de acordo com Baum (2017), o ``contracontrole atua para corrigir a inequidade/desigualdade, diminuindo o desequilíbrio no poder'' (p. 202; tradução da autora). Assim, estratégias de contracontrole (e.g., as lutas nos movimentos feministas) podem ser consideradas o que costumeiramente é chamado de ``resistência'' nos feminismos.
\end{enumerate}

É importante considerar que, por mais que você seja um/a profissional atualizado, com conhecimentos sobre grupos sociais estigmatizados, isso não significa que você esteja imune aos vários tipos de preconceitos que nós aprendemos durante nossas vidas. Nas palavras de Ruiz (2003): ``a despeito de nossos esforços autodeclarados de objetividade, nossas observações, descrições e análises funcionais não estão imunes a nossas suposições delimitadas pela cultura (\textit{culture-bound}), incluindo aquelas com relação ao sexo e às práticas generificadas (p. 15; tradução da autora)''.

O objetivo deste trabalho foi evidenciar alguns pontos de contato entre a proposta analítico-comportamental e o feminismo interseccional. Este trabalho não se propôs esgotar as discussões no que se refere à articulação entre essas duas áreas; pelo contrário, seu propósito foi servir como um ponto de partida para que as discussões sobre os pontos de contato (e possíveis discordâncias) sejam cada vez mais abordadas, assim como fomentar o uso da abordagem interseccional nas pesquisas de AC, de modo geral.
\vfill
\pagebreak
\section*{Referências Bibliográficas}\sectionmark{Referências Bibliográficas}

\hangindent=25pt
\hangafter=1
\noindent Aguião, S. (2015). A produção de identidades e o reconhecimento de sujeitos e direitos: Algumas possibilidades da perspectiva interseccional e da articulação de marcadores sociais da diferença. Material suplementar da disciplina ``Sexualidade'' do curso de especialização em gênero e sexualidade da Universidade Estadual do Rio de Janeiro. 

\hangindent=25pt
\hangafter=1
\noindent Amd, M., de Almeida, J. H., de Rose, J. C., Silveira, C. C., \& Pompermaier, H. M. (2017). Effects of orientation and differential reinforcement on transitive stimulus control. Behavioural Processes, 144, 58-65. doi: \url{10.1016/j.beproc.2017.08.014}

\hangindent=25pt
\hangafter=1
\noindent Barnes, D., Leader, G., \& Smeets, P.M. (1996). Establishing equivalence relations using a respondent-type training procedure. The Psychological Record, 46, 685-706.

\hangindent=25pt
\hangafter=1
\noindent Barnes-Holmes, D., Harte, C., \& McEnteggart, C. (no prelo). Implicit cognition and social behaviour. In Rehfeldt, R. A., Tarbox, J., \& Fryling, M. (Eds.) Applied Behavior Analysis of Language and Cognition. New Harbinger: Oakland, CA.

\hangindent=25pt
\hangafter=1
\noindent Barnes-Holmes, D., Murphy, A., Barnes-Holmes, Y., \& Stewart, I. (2010). The Implicit Relational Assessment Procedure: Exploring the impact of private versus public contexts and the response latency criterion on pro-white and anti-black stereotyping among white Irish individuals. The Psychological Record, 60, 57-66.

\hangindent=25pt
\hangafter=1
\noindent Baum, W. H. (2017). Understanding Behaviorism: Behavior, culture and evolution. 3ª. ed. United Kingdon: Wiley. 

\hangindent=25pt
\hangafter=1
\noindent Brah, A., \& Phoenix, A. (2004). Ain’t I A Woman? Revisiting intersectionality. Journal of International Women’s Studies, 5(3), 75-86.

\hangindent=25pt
\hangafter=1
\noindent Butler, J. (2001). Corpos que pesam: Sobre os limites discursivos do sexo. In: Louro, G. L. (Org.). O corpo educado:Pedagogias da sexualidade (p. 151-172). Belo Horizonte: Autêntica.

\hangindent=25pt
\hangafter=1
\noindent Butler, J. (2003). Problemas de gênero:Feminismo e subversão da identidade. Rio de Janeiro: Civilização Brasileira. 

\hangindent=25pt
\hangafter=1
\noindent Couto, A. G., \& Dittrich, A. (2017). Feminismo e análise do comportamento: Caminhos para o diálogo. Revista Perspectivas em Análise do Comportamento, 8(2), 147-158. 

\hangindent=25pt
\hangafter=1
\noindent Crenshaw, K. W. (1989). Demarginalizing the intersection of race and sex: A Black feminist critique of antidiscrimination doctrine, feminist theory, and antiracist politics. University of Chicago Legal Forum, 8(1), 139-167. 

\hangindent=25pt
\hangafter=1
\noindent Crenshaw, K. W. (2002). Documento para o encontro de especialistas em aspectos da discriminação racial relativos ao gênero. Estudos Feministas, 10(1), 171-188.

\hangindent=25pt
\hangafter=1
\noindent Davis, K. (2008). Intersectionality as buzzword, a sociology of science perspective on what makes a feminist theory successful. Feminist Theory, 9(1), 67-85. 

\hangindent=25pt
\hangafter=1
\noindent Del Prette, Z. A. P., \& Del Prette, A. (2005). Psicologia das habilidades sociais na infância: Teoria e prática. Petrópolis, RJ: Vozes.

\hangindent=25pt
\hangafter=1
\noindent Falcón, S. M. (2009). Black Feminist Thought. In: O’Brien, J. (Ed.). Encyclopedia of Gender and Society. CA: SAGE Publications. 

\hangindent=25pt
\hangafter=1
\noindent Fideles, M. N. D., \& Vandenberghe, L. (2014). Psicoterapia Analítica Funcional feminista: Possibilidades de um encontro. Psicologia: Teoria e Prática, 16(3), 18-29.

\hangindent=25pt
\hangafter=1
\noindent Finn, M., Barnes-Holmes, D., Hussey, I., \& Graddy, J. (2016). Exploring the behavioral dynamics of the Implicit Relational Assessment Procedure: The impact of three types of introductory rules. The Psychological Record, 66, 309–321.\\doi: \url{10.1007/s40732-016-0173-4}

\hangindent=25pt
\hangafter=1
\noindent Gergen, K. J. (1985). The social constructionist movement in modern psychology. American Psychologist, 40, 266-275.

\hangindent=25pt
\hangafter=1
\noindent Gergen, K. J. (1994). Realities and relationships: Soundings in social construction. Cambridge: Harvard University Press.

\hangindent=25pt
\hangafter=1
\noindent Gergen, K. J., \& Davis, K. E. (1985). The social construction of the person. New York: Springer-Verlag. 

\hangindent=25pt
\hangafter=1
\noindent Goff, P. A., Jackson, M.C., Di Leone, B.A.L., Culotta, C.M., \& DiTomasso, N.A. (2014). The essence of innocence: Consequences of dehumanizing Black children. Journal of Personality and Social Psychology, 106(4), 526-545. doi: \url{10.1037/a0035663}

\hangindent=25pt
\hangafter=1
\noindent Guerin, B. (2009). Análise do comportamento e a construção social do conhecimento [Behavior analysis and the social construction of knowledge]. Revista Brasileira de Análise do Comportamento, 5(1), 117-137. (Original publicado em 1992).

\hangindent=25pt
\hangafter=1
\noindent Guerin, B., \& de Oliveira Ortolan, M. (2017). Analyzing domestic violence behaviors in their contexts: Violence as a continuation of social strategies by other means. Behavior and Social Issues, 26, 5-26.

\hangindent=25pt
\hangafter=1
\noindent Harding, S. (1986). The science question in feminism. Ithaca and London: Cornell University Press. 

\hangindent=25pt
\hangafter=1
\noindent Hayes, S. C., Barnes-Holmes, D., \& Roche, B. (Eds.). (2001). Relational Frame Theory: A post-Skinnerian account of human language and cognition. New York: Plenum Press.

\hangindent=25pt
\hangafter=1
\noindent Henning, C. E. (2015). Interseccionalidade e pensamento feminista: As contribuições históricas e os debates contemporâneos acerca do entrelaçamento de marcadores sociais da diferença. Mediações, 20(2), 97-128. 

\hangindent=25pt
\hangafter=1
\noindent Keller, E. F. (2006). Qual foi o impacto do feminismo na ciência? Cadernos Pagu, 27, 13-3. 

\hangindent=25pt
\hangafter=1
\noindent Leader, G., Barnes, D., \& Smeets, P. M. (1996). Establishing equivalence relations using a respondent-type training procedure. The Psychological Record, 46, 685-706.

\hangindent=25pt
\hangafter=1
\noindent Louro, G.L. (2008). Um corpo estranho: Ensaios sobre sexualidade e teoria queer.1ª. Ed. Belo Horizonte: Autêntica.

\hangindent=25pt
\hangafter=1
\noindent Moscovici, S. (1988). Notes towards a description of social representation. European Journal of Social Psychology, 18, 211-250. 

\hangindent=25pt
\hangafter=1
\noindent National Institute of Health (2012). Subtle and dangerous: Symptoms of heart disease in women. U.S. Department of Health and Human Services. Recuperado de: \url{https://tinyurl.com/feminismoac2}

\hangindent=25pt
\hangafter=1
\noindent Nicholson, L. (2000). Interpretando o gênero. Revista Estudos Feministas, 8(2), p. 9-41. Recuperado de: \url{https://periodicos.ufsc.br/index.php/ref/article/view/11917/11167}

\hangindent=25pt
\hangafter=1
\noindent Nogueira, C. (2001). Contribuições do construcionismo social a uma nova psicologia do gênero. Cadernos de Pesquisa da Fundação Carlos Chagas, 112, 37-153.

\hangindent=25pt
\hangafter=1
\noindent Nogueira, C. (2017). Interseccionalidade e Psicologia Feminista. Bahia: Ed. Devires. 

\hangindent=25pt
\hangafter=1
\noindent Piscitelli, A. (2002). Recriando a (categoria) mulher? In: Algranti, L. (Org.). A prática feminista e o conceito de gênero (p. 7-42). Campinas: IFCH-Unicamp.

\hangindent=25pt
\hangafter=1
\noindent Piscitelli, A. (2008) Interseccionalidades, categorias de articulação e experiências de migrantes brasileiras. Sociedade e Cultura, 11(2), 263-274.

\hangindent=25pt
\hangafter=1
\noindent Porto, T. H. (2016). Does participant’s gender interfere with behavior studies? Operants, Quarter IV, p. 40-42. Recuperado de: \url{https://tinyurl.com/feminismoac21}
Rich, A. (2010). Heterossexualidade compulsória e existência lésbica. Bagoas: Estudos gays, gêneros e sexualidades, 4(5), p. 17-44. (Original publicado em 1980). Recuperado de: \url{http://www.cchla.ufrn.br/bagoas/v04n05art01_rich.pdf}

\hangindent=25pt
\hangafter=1
\noindent Roche, B., \& Barnes-Holmes, D. (2003). Behavior analysis and social constructionism: Some points of contact and departure. The Behavior Analyst, 26(2), 215-231. 

\hangindent=25pt
\hangafter=1
\noindent Ruiz, M. R. (1995). B. F. Skinner’s radical behaviorism: Historical misconstructions and grounds for feminist reconstructions. Behavior and Social Issues, 5(2), 29-44. 

\hangindent=25pt
\hangafter=1
\noindent Ruiz, M. R. (1998). Personal agency in feminist the¬ory: Evicting the illusive dweller. The Behavior Analyst, 21, 179–192.

\hangindent=25pt
\hangafter=1
\noindent Ruiz, M. R. (2003). Inconspicuous sources of behavioral control: The case of gendered practices. The Behaviorist Analyst Today, 4(1), 12-16. 

\hangindent=25pt
\hangafter=1
\noindent Ruiz, M. R. (2009). Beyond the mirrored spa¬ce: Time and resistance in feminist theory. Behavior and Philosophy, 37, 141–147.

\hangindent=25pt
\hangafter=1
\noindent Ruiz, M. R. (2013). Values and morality: Science, faith, and feminist pragmatism. The Behavior Analyst, 36(2), 251-254. 

\hangindent=25pt
\hangafter=1
\noindent Ruiz, M. R., \& Roche, B. (2007). Values and the scientific culture of behavior analysis. The Behavior Analyst, 30, 1–16.

\hangindent=25pt
\hangafter=1
\noindent Schiebinger, L. (2001). O feminismo mudou a ciência. Bauru: Edusc.

\hangindent=25pt
\hangafter=1
\noindent Scott, J. (1995). Gênero: Uma categoria útil de análise histórica. Revista Educação e Realidade,20(2), p. 71-99, 1995. Recuperado de: \url{https://tinyurl.com/feminismoac23}

\hangindent=25pt
\hangafter=1
\noindent Sidman, M. (1994). Equivalence relations and behavior: A research story. Boston, MA: Authors Cooperative.

\hangindent=25pt
\hangafter=1
\noindent Sidman, M., \& Tailby, W. (1982). Conditional discrimination vs. matching to sample: An expansion of the test paradigm. Journal of the Experimental Analysis of Behavior, 37(1), 5-22. doi: \url{10.1901/jeab.1982.37-5}

\hangindent=25pt
\hangafter=1
\noindent Silva, E. C. \& Laurenti, C. (2016). B.F. Skinner e Simone de Beauvoir: ``A mulher'' à luz do modelo de seleção pelas consequências. Perspectivas em Análise do Comportamento, 7(2), 197-211.

\hangindent=25pt
\hangafter=1
\noindent Skinner, B. F. (1974). About behaviorism. New York: Alfred A. Knopf.

\hangindent=25pt
\hangafter=1
\noindent Skinner, B.F. (1978). Comportamento Verbal. São Paulo: Cultrix/EDUSP. (Original publicado em 1957).

\hangindent=25pt
\hangafter=1
\noindent Steinbugler, A. C., Press, J. E., \& Dias, J. J. (2006). Gender, race, and affirmative action: Operationalizing intersectionality in survey research. Gender and Society, 20(6), 805-825.

\hangindent=25pt
\hangafter=1
\noindent Warner, L. R. (2008). A best practices guide to intersectional approaches in psychological research. Sex Roles, 59, 454-463. doi: \url{10.1007/s11199-008-9504-5}
